\section{Deployment Lessons Learned}
\label{evaluation-sec-lessons}

Sensor network deployments, particularly in remote areas, involve significant
cost in terms of time and equipment. Failures of hardware and software can
have a negative impact on the uptake of this technology by domain science
experts. Our experiences at Reventador volcano produced a set of valuable
lessons for future sensor network deployments. Several important lessons led
directly to Lance (Chapter~\ref{chapter-lance}) and IDEA
(Chapter~\ref{chapter-idea}), new architectural approaches described in later
chapters.

\begin{itemize}

\item \textbf{Working with developmental code:} The software failures that we
observed during our deployment (FTSP, Deluge) are common to code contributed
by research projects. Frequently it takes a great deal of time between when
the code is made publicly available and all of the errors and problems are
removed.

In the case of FTSP, we tried to do the right thing by careful
experimentation before we deployed our system, but were still surprised by
its poor performance in the field. In the case of Deluge, we were guilty of
incorporating it into the system extremely late, and did not have time to
test it as rigorously as other components. In addition, our lack of
experience with the Deluge code meant that, when it failed, we had no choice
but to remove it and try to reproduce some of the important functionality it
was providing elsewhere.

\item \textbf{Ground truth and self-validation mechanisms:}
During our deployment, we did not initially consider colocating several
of our wireless sensors with existing data loggers in order to establish
ground truth. This would have clearly aided our analysis, though we were
fortunate to locate one of our sensors near (but not immediately adjacent to)
the RVEN station. In addition, self-validation mechanisms are needed to
provide detailed information on the health and accuracy of the data recorded
by the network. The periodic ``heartbeat'' messages that we built into our
system proved essential to remotely tracking system operation.

More generally, it is critical to design the evaluation process well before
the system being studied is designed and deployed. Deployments are expensive
and deployment time is valuable, and if the system is not properly
instrumented it can be difficult to assess its performance after the
deployment has ended.

\item \textbf{Coping with infrastructure and protocol failures:} As discussed
previously, we were surprised to find that the sensor nodes themselves were
the most reliable components of the system. Even without classifying the
3-day network outage as an infrastructure failure, this downtime was far
exceeded by outages caused by power failures at the base station. We did not
devote enough attention to assuring the reliability of the base station and
radio modem infrastructure, assuming it would be a trivial matter of plugging
into wall power. This single point of failure was more fragile than expected.

Additionally, several pieces of deployed software, including Deluge and FTSP,
exhibited failures in the field that we had not expected given our laboratory
experiments. These failures both speak for and show the limitations of
careful, pre-deployment testing. While we were fortunate to be able to
correct protocol errors in the field and during post-processing, the risk of
uncorrectable problems argues for more rigorous testing and analysis. At the
same time, the unpredictability of field deployments places demands on the
flexibility and adaptability of the researchers involved.

\item \textbf{Building confidence inside cross-domain scientific
collaborations:} It is important when working with domain scientists to
understand their expectations and plan carefully to meet them. There can be
tensions between the desire of computer science researchers to develop more
interesting, sophisticated and complex systems, and the needs of domain
science, which relies upon thoroughly validated instrumentation. Developing
new instrumentation means a nontrivial probability of failure, which can be
frustrating to scientists used to working with reliable equipment. Pushing
more complexity into the sensor network can improve lifetime and performance,
but the resulting system must be carefully validated before deployment to
ensure that the resulting data is scientifically accurate.

Effective collaboration involves both meeting and managing scientific
expectations. Particularly when using low-power devices, it can be difficult
to duplicate the properties of existing high-power instrumentation.  In our
case, we were unable to provide continuous signals for analysis.
Understanding the scientists' goals makes it more likely that you can
convince them that your system can meet them, despite the fact that it is
probably providing different information with different properties than they
are used to. Simply duplicating existing instrumentation is not a path to the
macroscope. Scientists also have to devise new techniques for harnessing
low-power, unreliable, embedded devices fitted with low-resolution sensors.

Good communication between computer and domain scientists is also critical.
During our deployment, the seismologists were eager to see the collected
signals, which were initially in an unprocessed format with timing errors as
described earlier. From the CS perspective, the early data provided evidence
of successful data collection, but from the geophysics perspective it
highlighted failures in the time synchronization protocol. It took a great
deal of effort after the deployment to build confidence in the validity of
our data.

\item \textbf{Incorporating flexibility for experimentation and system
adaptation:} Designing a system that is flexible and provides a great deal of
diagnostic output can be a great way to facilitate deployment
experimentation. It also has the side benefit of making it more likely that
researchers can work around unforseen failures.

Our system included two features that improved flexibility and visibility and
that we found very helpful. First was the command-and-control interface
discussed earlier. We used the ability to reboot nodes remotely to help work
around the FTSP failures, and the ability to change node state could have
helped if other protocols had not worked as expected. For example, had the
routing protocol failed or produced poor performance we could have chosen
each node's parent manually. Second, the periodic status messages provided
critical information about the functioning of the system. Despite the rapid
rate at which they were being sent (every 10~s), their overhead did not have
an appreciable effect on the system's performance, and the ability to quickly
view the effects of changes to the network was extremely helpful when
debugging.

Given more time it would have been interesting to experiment with other
system components. We could have varied the metrics used by the routing tree,
compared various event detection algorithms, or tweaked parameters used by
the bulk data-transfer protocol.

\end{itemize}

Many features of Lance, described in the next chapter, emerged directly out
of some of the challenges we encountered at Reventador. Lance replaces the
binary event detector with a value-based ordering of all signals. This
reduces the dependence of the performance of the overall system on event
detection parameters and thresholds, and allows us to adapt to changing
volcanic behavior. Data collection, which was shown to be a significant
bottleneck in terms of bandwidth and energy consumption, is also considered
directly when examining signals to download. Finally, our negative experience
with Deluge caused us to avoid relying on adaptations that required
reprogramming nodes, leading to the policy-module framework included in Lance
which facilitates significant changes to network behavior through changes
made only at the base station.
