This all started in the summer of 2003. A freshly-minted Harvard College
graduate without any better plans, I replied to a forwarded email from a new
Harvard Computer Science Professor who was looking for research assistants.
That was Matt Welsh. Matt saw something in me that I didn't yet see in
myself, and stuck with me through the years, through my stubborness,
intransigence, bad habits, and willful disobedience. Along the way I've
thought about leaving for law school, or get-rich-quick schemes like
Facebook, but I'm glad I'm going to keep building and studying real systems,
since that's been more and more fun as we've continued working together. I
know I'll think about Matt a lot as I start my own group, and I can only hope
to do as well as he has.

Margo Seltzer deserves credit for helping me discover the thing that I'm
perhaps starting to get good at. Taking Operating Systems from her in 2001 is
the highlight of my college experience, turning me in the direction I
continue to travel. She, Matt, and Michael Mitzenmacher all provided valuable
commentary on this dissertation, and improved it.

Building volcano monitoring sensor networks requires people who know
something about volcanos, and this project really came about because we met
two great ones: Jonathan Lees and Jeff Johnson. They offered us a huge amount
of support in the field, patiently answered many questions, and really
embraced the technology we offered. I am extremely thankful for them, as well
as the wonderful staff at the IGEPN in Ecuador, and the Ecuadoreans we've had
the chance to work with --- Mario Ruiz and Omar Marcillo.

I have had the pleasure of spending a lot of time with Harvard undergraduates
over the past seven years, through my affiliations with CS161 and Eliot
House. I am thankful for all of my former students, especially those that
came to work with us: Pat Swieskowski and Stephen Dawson-Haggerty. And
without the community at Eliot House I probably never would have completed
this degree. I thank Lino Pertile and Mike Canfield for giving me a chance to
be a tutor, and all my former Eliot students and friends for their love and
support. Floreat domus de Eliot!

None of the projects described in these pages was done alone, and I have been
blessed with fun, intelligent, and forgiving colleagues at Harvard. Thaddeus
Fulford-Jones designed our volcano sensor board, experienced the joy of
sharing an office with me and has become a good friend. Konrad Lorincz joined
our deployment team in 2005 and did a huge portion of our volcano monitoring
system. Jason Waterman has been a joy to work with and contributed greatly to
the IDEA project. Rohan Murty tolerated my many attempts to annoy him before
I discovered what a special person he is. Mark Hempstead has always been
willing to offer an electrical engineer's biased take on things. Bor-rong
Chen and Geoff Mainland have been supportive of my work, offering a great
deal of constructive criticism. Lex Stein was always available to offer
advice, most of it helpful, the rest amusing. People like Joanne Bourgeois,
Gioia Sweetland, Tristen Hubbard, Susan Wieczorek, James MacArthur, Lars
Kellogg-Stedman and William Walker make it a pleasure to work at the School
of Engineering and Applied Science. Additionally, I thank all of the members
of the SYRAH and Welsh groups not mentioned by name, travelers on the
Maxwell-Dworkin Spaceship, and the interns in MSR 112/3001 that kept me sane
in the summer of 2007.

\clearpage

I couldn't ask for better parents. They started me on this journey of
discovery years and years ago, and look how far we've come! My mother also
provided helpful feedback on the amount of time that the Ph.D. process was
taking, and I think she'll be pleased that I'm receiving a degree... at last.

My wife Suzanna is my favorite person and the reason I get up joyfully in the
morning. I can't wait to see what life holds for us, together: all the days,
all the ways.

And to Chuchu --- WOOF!
