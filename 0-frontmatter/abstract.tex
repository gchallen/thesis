\hspace{0.25in}

This dissertation presents architectural solutions that improve the
performance of scientific sensor networks. Wireless sensor networks composed
of tens or hundreds of low-power, resource-constrained devices, can aid
scientific exploration by collecting data at scales difficult to achieve with
traditional instrumentation. To do so, sensor networks must provide
high-fidelity data suitable for scientific study, identify and devote system
resources to the data most valuable to the application, and respond to
fluctuations in energy availability. Our thesis is that by managing
distributed resource usage, we can improve the data quality provided by
sensor networks to scientific applications.

We make three core contributions. First, working with volcano seismologists,
we have designed and evaluated the first sensor network enabling the
study of erupting volcanos. Through three field studies we validate that
it produces data suitable for scientific analysis. We found verifying the
timing accuracy of our data both difficult and important and were able
to develop a new \textit{post-hoc} time-rectification technique that can
recover accurate timestamps even in the face of timing protocol failures.

Second, we developed Lance, a general-purpose architecture targeting high
data-rate sensor networks. Lance manages data-collection in order to direct
energy and bandwidth resources toward the most interesting signals. Valuable
data is identified using a two-tiered architecture leveraging the increased
visibility of a network controller while limiting the overhead of centralized
control. Lance has been used to support the volcano-monitoring application as
well as the Mercury project, which enables monitoring of patients with
neuromuscular disorders. We show that Lance's resource allocation heuristic
achieves near-optimal efficiency, within 5\% of an offline optimal solution
in most cases.

Finally, we present IDEA (Integrated Distributed Energy Awareness), a sensor
network service allowing application components to tune their behavior in
response to changing energy availability. This allows the network to adjust
energy load as availability changes, particularly important when
energy-harvesting capabilities are being used. We have used IDEA to build
energy-aware routing and MAC protocols and measured the performance
improvement they provide over non-aware solutions. By better distributing
energy consumption, IDEA extends the network's lifetime by up to 27\%.
