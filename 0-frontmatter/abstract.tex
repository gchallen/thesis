This dissertation discusses several architectural solutions aiding the design
and deployment of scientific sensor networks. Our thesis is that wireless
sensor networks composed of tens or hundreds of low-power,
resource-constrained devices, can aid scientific exploration by providing
high-fidelity data at scales more difficult to achieve with traditional
instrumentation. Doing so requires sensor networks that are able to provide
high-fidelity data for scientific study, identify the data most valuable to
the application and devote system resources to capturing it, and respond to
ongoing fluctuations in distributed energy availability.

We present a case study and two architectural solutions addressing these core
challenges. Working with volcano seismologists we have designed and evaluated
the first sensor network designed to enable the study of erupting volcanos.
We have performed three field studies of this system and validated that it
can produce data suitable for scientific analysis. We found correcting and
validating the timing accuracy of our data extremely difficult and important,
and were able to develop a new \textit{post-hoc} time-rectification technique
that can recover accurate timestamps even in the face of significant timing
protocol failures.

To improve the performance of this application, we developed two
general-purpose architectures targetting high data-rate sensor networks.
Lance directs bulk data-collection in order to direct energy and bandwidth
resources toward the most interesting signals identified by the application.
Valuable data is identified using a two-tiered architecture which manages to
leverage the increased visibility of a whole-network controller while
limiting the overhead of centralized control. While designed to support the
volcano-monitoring application, Lance has also been used to support the
Mercury project, which enables in-home monitoring of patients with
neuromuscular disorders. 

IDEA (Integrated Distributed Energy Awareness) is a sensor network service
allowing application components to tune their behavior in response to
changing energy availability. This allows the network to adjust energy load
as availability changes, of particular importance when energy-harvesting
capabilities are incorporated. We have used IDEA to build energy-aware
routing and MAC-parameter tuning protocols, and measured the performance
improvement they provide over non-aware solutions.
