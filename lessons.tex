\chapter{Discussion and Future Work}
\label{chap-lessons}
\label{sec-cb-future}
\label{sec-cb-futurework}

The preceding chapters demonstrated that medical sensor networks can be
realistically constructed and practically monitored with mote-based 
sensors. Medical sensor networks represent a new class of sensor networks that
is different from conventional systems designed for
single-sink data collection from all statically deployed nodes with
static sensing parameters. Instead, a medical sensor network demands
flexibility in node mobility, node population, and sensing with the goal to
deliver relevant data in real time to responsible receiving devices. 

The CodeBlue architecture ties
together a number of current areas in sensor network research, including node
discovery, routing protocols, and query interfaces, into a single system that
is specifically designed for medical applications. An important contribution
of our design is the realization of a publish/subscribe communication
architecture for sensor networks, and its integration into a complete
application-centric platform.
The dynamic nature of CodeBlue architecture and TinyADMR multicast routing
layer allows medical teams to set up a medical sensor network easily to start
monitoring patients with no
prior infrastructure. The
publish/subscribe communication model allows multicast capability in the
network to improve bandwidth efficiency. The choice of ad-hoc routing protocol
allows nodes to be joining and leaving the network
dynamically. The CBQ query layer provides flexibility to acquire sensor
readings with different sampling rates, for different durations, or only when
certain conditions are met. In addition, LiveNet provides practical
monitoring of a deployed medical sensor network so that the engineering team
can measure and troubleshoot the network without sacrificing the efficiency or
reliability of the network.

Besides the contributions described above, it is worth revisiting the design
goals to justify the design decisions. Moreover, there are several valuable
lessons learned by designing and deploying medical sensor networks. The
remaining part of this chapter discusses these lessons and provides potential
future research directions.

\section{Revisiting design goals}

CodeBlue was designed to fulfill the specific needs of medical sensor
networks. Returning to the goals set forth in Chapter~\ref{chap-codeblue}, the
following highlights the features of CodeBlue design that help achieve each one.

\begin{description}
\item[Scalability and robustness:] Scalability derives directly
from the publish/subscribe communication model, because it allows
efficient multicast delivery of identical data to multiple sinks.
In addition, the use of trigger predicates in CBQ permits queries
to remain quiescent until the trigger conditions are met. 

\item[Multiple concurrent queries:] CBQ supports multiple queries per
node with different query parameters and
destinations. Accomplishing this is simplified by decoupling the query
engine from the routing layer.

\item[Ad-hoc deployment:] CodeBlue supports rapid, ad-hoc deployment 
via its dynamic discovery protocol without need for manual network
configuration.

\item[Support a broad class of medical sensors:] The sensor hardware
abstraction layer makes it easy to add new sensor types, as only a
low-level driver and an entry to the query processor's table of
supported sensors is required.

\item[External programmatic interface:] The CodeBlue network is made
available to many potential Internet-based applications through a
proxy server using standard Web Services protocols.
\end{description}

\section{Improving reliability}

Even though most of the design goals are met, our evaluation of the CodeBlue
prototype points to a number of areas for future work. The first priority is
to add reliability mechanisms to the communication layer, although our results
show that packet loss can be mitigated somewhat through redundant
transmissions. The fact that CodeBlue system suffers from low packet delivery
performance under high load shows that our adoption of a best effort protocol
was not suitable for some of our target scenarios. However, we do not believe
that reliable routing is required for all medical data; rather, the system
should allow each query to specify its reliability needs in terms of
acceptable loss, data rate, or jitter.

Another area to explore is the design of effective techniques to mitigate the
impact of bandwidth limitation. For example, each CodeBlue query could specify
a data priority that would allow certain messages (say, an alert from a
critical patient) to have higher priority than others in the presence of radio
congestion. This approach can be combined with rate-limiting congestion
control~\cite{paek_rcrt:_2007, ee-congestion,hull-congestion} to bound the
bandwidth usage of patient sensors to avoid network congestion while 
ensuring the delivery of high priority data packets.

The evaluation of the disaster drill shows that a significant portion
(16-25\%) of packet loss can be attributed to premature query timeouts. 
Under this situation, the data packets were not lost {\em per se} but they
were never transmitted by sensor nodes. This
indicates a significant portion of CBQ query packets, sent as network floods,
did not reach the relevant nodes to renew the desired queries. As a result,
the patient sensors stopped reporting relevant sensor data after previous
queries expired. This is most likely caused by the long-lasting flood that
happened due to a software bug and created heavy interference to normal
query packets. This suggests that using simply network floods to send queries
is not sufficient to obtain acceptable performance when there is heavy traffic
on the wireless channel. Future designs could consider replacing simple
network flooding with more reliable network-wide dissemination approaches 
when sending CBQ queries to prevent this type of data loss.
Trickle~\cite{trickle} and RBP~\cite{rbp-sensys06}
are two examples of such dissemination protocols.

\section{Sensor networks as MANETs}

Chapter~\ref{chap-tinyadmr} presented an investigation of the issues that arise 
when translating MANET-based protocol designs into a sensor 
network context. As sensor networks become more widespread, 
new applications will be developed that present a broad set of 
communication requirements. Given that the MANET community has 
invested a great deal of effort into routing protocols for 
mobile wireless environments, we believe that there is real
value in understanding to what extent this work can be reapplied.

Many of the design concepts in CodeBlue are drawn from both the sensor
networks and mobile ad-hoc networking (MANETs) communities.  As with
much current sensor networks research, we are targeting at very
resource-limited sensor nodes, which are desirable for their small
size and power requirements. Unlike much of the previous work,
CodeBlue is concerned with sensor networks that have variable data
rates, dynamic node populations, node mobility, and multiple
simultaneous users.

These requirements drives us closer to the MANETs space, albeit with an
important distinction: most of the research undertaken by the MANETs community
has not been translated into extremely resource-constrained environments. For
instance, most MANETs routing protocols
(e.g.,~\cite{admr-mobihoc01,aodv,dsr,dsdv}) were designed with PDAs or laptops
in mind. The limitations of our hardware forces us to revisit the design of
MANETs protocols, in particular paying close attention to memory usage and
communication overheads.

Many of the lessons arising from our TinyOS-based implementation of 
ADMR stem from the enormous differences in the assumed hardware
environment. ADMR was developed for relatively high-powered
devices with significantly more radio bandwidth and memory than
is found on typical motes. It is not surprising, then, that 
we faced some challenges implementing ADMR on this platform.

While our goal was to remain faithful to the original ADMR design as
much as possible, the most substantial modification was the introduction
of alternate path-selection metrics. Minimizing hop count performs
poorly, while a simple LQI-based estimation of the path delivery ratio
works well and incurs no additional measurement overhead. 

There is still significant space for improvement 
if TinyADMR is to be effective in large networks. With a large number
of nodes, protocol overhead will readily saturate available bandwidth.
Most MANET protocols use a fixed transmission rate for protocol
packets such as discovery messages. Dynamically tuning these rates
based on background traffic or node density would permit overhead
to scale with available bandwidth. 

Protocol state management under severe memory limitations is another 
area for future work. We have explored various node table eviction
policies, although results demonstrate that deliberately dropping
this state has an adverse effect on performance. More efficient
data structure design and swapping state to external flash may be
viable solutions.


\section{Deployment monitoring}

We have found LiveNet to be a valuable tool in understanding the behavior of
deployed sensor networks, as shown in Chapter~\ref{chap-deploy}. There are
several directions that one could take to further improve LiveNet's
value to studies of deployed sensor networks.

\subsection{On-line trace analysis}

LiveNet, with the current design, performs on-line data collection with offline
post processing and analysis. We believe it would be valuable to 
investigate a somewhat different model, in which the infrastructure 
passively monitors network traffic and alerts an end-user when certain
conditions are met. For example, LiveNet nodes could be seeded with event
detectors to trigger on interesting behavior such as routing
loops, packet floods, or corrupt packets. 

To increase scalability and decrease overhead for merging traces, it may be
possible to perform {\em partial merging} of packet traces 
around windows containing interesting activity. In this way the
LiveNet infrastructure can discard (or perhaps summarize) the
majority of the traffic that is deemed uninteresting. 

Such an approach would
require decomposing event detection and analysis code into components
that run on individual sniffers and those that run on a back-end
server for merging and analysis. For example, it may be possible to
perform merging in a distributed fashion, merging traces pairwise
along a tree; however, this requires that at each level there is
enough correspondence between peer traces to ensure timing correction
can be performed. Having an expressive language for specifying
trigger conditions and high-level analyses strikes us as an
interesting area for future work.

\subsection{Passive monitoring friendly design}

Knowing the value of passive monitoring, future protocols can be designed to
be "friendly" to the passive monitoring systems to improve the
efficiency and accuracy of trace processing. For example, making every packet
unique, even across re-transmissions, will simplify trace merging. Being
able to identify each packet transmission makes it easier to identify bugs
that cause unexpected packet transmissions and allow analyses related
to packet re-transmissions. A protocol could also include internal states such
as number of packets that were forwarded or dropped in infrequent diagnostic
packets to reveal more internal details of the protocol. 


\section{Security}

One shortcoming of CodeBlue system design is security.  Security is
an extremely important concern in medical sensor networks, in order to
maintain patient privacy, avoid external spoofing, and defend against
denial-of-service attacks. However, exploring the rich design space of
security mechanisms in medical sensor networks is beyond the scope of
this dissertation. Our current design assumes that nodes in a CodeBlue
network share encryption keys and use an efficient private-key
encryption scheme such as TinySec~\cite{tinysec}.  A shortcoming of
this design is that it does not support access to the CodeBlue network
from multiple administrative domains with separate keys, and assumes
an out-of-band mechanism for establishing keys.  In a hospital or
disaster response setting, it is reasonable to assume that all nodes
will be programmed in advance with the necessary keys, although this
approach makes it difficult to change keys.

The privacy and security requirements for medical care are
complex and differ depending on the scenario. For example, HIPAA privacy
regulations need not be enforced during life-saving procedures. Designing
the right security into medical sensor networks is worth future research.

