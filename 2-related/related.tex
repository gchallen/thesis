\section{Related Work}
\label{chapter-relatedwork}

\XXXnote{GWA: Things to mention: VoxNet, Vigilance, Cane Toad monitoring,
Turtle monitoring, ExScal (?), Glacier sensing, SMART, Lew's stuff.}

\XXXnote{GWA: Pulled from Sensys'08 introduction. Perhaps some useful
references.}

Many sensor network applications involve the acquisition
of high-resolution signals using low-power wireless sensor nodes. Examples
include monitoring acoustic, seismic, and vibration waveforms in
bridges~\cite{ggb-ipsn07}, industrial equipment~\cite{intel-northseasensys}, 
volcanoes~\cite{volcano-osdi06}, and animal
habitats~\cite{girod-ipsn07,enviromic}.  These systems all attempt to acquire
high data-rate (100~Hz or higher), high-fidelity data across the network,
subject to severe constraints on radio bandwidth and energy usage.

\XXXnote{GWA: Pulled from OSDI'06 related work.}

While the number of sensor network deployments described in the literature
has been increasing, little prior work has focused on evaluating sensor
networks from a scientific perspective.  In addition, the high data rates and
stringent timing accuracy requirements of volcano monitoring represent a
departure from many of the previously-studied applications for sensor
networks.

{\bf Low-data-rate monitoring:} The first generation of sensor network
deployments focused on distributed monitoring of environmental conditions.
Representative projects include the Great Duck
Island~\cite{spm:04habitat,polastre-masters,mainwaring-habitat}, Berkeley
Redwood Forest~\cite{berkeley-redwoods}, and James
Reserve~\cite{cerpa-habitat} deployments. These systems are characterized by
low data rates (sampling intervals on the order of minutes) and very
low-duty-cycle operation to conserve power.  Research in this area has made
valuable contributions in establishing sensor networks as a viable platform
for scientific monitoring and developing essential components used in our
work. 

This previous work has not yet focused on the efficacy of a sensor network as
a scientific instrument.  The best example is the Berkeley Redwood Forest
deployment~\cite{berkeley-redwoods}, which involved 33~nodes monitoring the
microclimate of a redwood tree for 44~days.  Their study focuses on novel
ways of visualizing and presenting the data captured by the sensor network,
as well as on the data yield of the system. The authors show that the
microclimactic measurements are consistent with existing models; however, no
ground truth of the data is established. This paper highlights many of the
challenges involved in using wireless sensors to augment or replace existing
scientific instrumentation.

{\bf High-data-rate monitoring:} A second class of sensor network
applications involves relatively high data rates and precise timing of the
captured signals. The two dominant applications in this area are structural
health monitoring and condition-based maintenance. In each case, arrays of
sensors are used to capture vibration or accelerometer waveforms that must be
appropriately timestamped for later analysis.

NetSHM~\cite{netshm-ewsnsubmission,netshm-emnets05,wisan} is a wireless
sensor network for structural health monitoring, which involves studying the
response of buildings, bridges, and other structures to localize structural
damage, e.g., following an earthquake. This system shares many of the
challenges of geophysical monitoring; indeed, the data rates involved (500~Hz
per channel) are higher than are typically used in volcano studies. 

NetSHM implements reliable data collection using both hop-by-hop caching and
end-to-end retransmissions. Their work explores the use of local computations
on sensors to reduce bandwidth requirements.  Rather than a global
time-synchronization protocol, the base station timestamps each sample upon
reception. The {\em residence time} of each sample as it flows from sensor to
base is calculated based on measurements at each transmission hop and used to
deduce the original sample time.

Several factors distinguish our work. First, NetSHM is designed to collect
signals following controlled excitations of a structure, which simplifies
scheduling.  In our case, volcanic activity is bursty and highly variable,
requiring more sophisticated approaches to event detection and data transfer.
Second, NetSHM has been deployed in relatively dense networks, making data
collection and time synchronization more robust.  Third, to date the NetSHM
evaluations have focused more on network performance and less on the fidelity
of the extracted data.  Other systems for wireless SHM include one developed
by the Stanford Earthquake Engineering
Center~\cite{wimms-lynch06,wimms-wang05} and earlier work by Berkeley on
monitoring the Golden Gate Bridge~\cite{ggb-monitoring}.

Condition-based maintenance is another emerging area for wireless sensor
networks. The typical approach is to collect vibration waveforms from
equipment (e.g., chillers, pumps, etc.) and perform time- and
frequency-domain analysis to determine when the equipment requires servicing.
Intel Research has explored this area through two deployments at a
fabrication plant and an oil tanker in the North
Sea~\cite{intel-northseasensys}. Although this application involves high
sampling rates, it does not necessarily require time synchronization as
signals from multiple sensors need not be correlated.  The initial evaluation
of these deployments only considers the network performance and does not
address data fidelity issues.

% 23 Apr 2006 : GWA : I think that we can cite to GGB in the intro but I
%               can't find a good evaluation.
%
%Another, earlier project using sensor networks for SHM is the Golden Gate
%Bridge project at the University of California, Berkeley~\cite{ggb-report}. This
%work focused largely on developing an architecture for high data-rate sensing
%and reliable data recovery and does not spend any time on data analysis.

% 22 Apr 2006 : GWA : Stuff that didn't get mentioned that I found somewhere:
%
% 1) WiMMS :
% http://eil.stanford.edu/publications/jerry_lynch/SPIE2002Paper.pdf
% http://eil.stanford.edu/publications/jerry_lynch/3WCSCPaper.pdf
% http://eil.stanford.edu/publications/jerry_lynch/USKoreaWSPaper2Sensor.pdf
% http://eil.stanford.edu/publications/yang_wang/IWSHM2005_YangWang_final.pdf
%
% This is Jerry Lynch's group at Stanford and might be worth looking at. They
% do some evaluation. Actually this seems well documented so I think that we
% should look at it.
%
% 2) Group at Notre Dame :
% http://www.nd.edu/~pantsakl/345-ISHMII05.pdf
% http://www.nd.edu/~mhaenggi/pubs/struct06.pdf
%
% This looks like mainly an architecture.

%\subsection{Habitat monitoring}

% 22 Apr 2006 : GWA : I think that maybe this is enough, but maybe mention
%               Zebranet?

% 22 Apr 2006 : GWA : Stuff that didn't get mentioned that I found somewhere:
%
% 1) Group at University of Australia doing "reactive environmental
% monitoring"
% http://www.csse.uwa.edu.au/adhocnets/WSNgroup/soil-water-proj/final.issnip04.25oct.pdf
%
% 2) Estrin's Tier Stuff
% http://lecs.cs.ucla.edu/Publications/papers/tier.pdf
