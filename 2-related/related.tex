\section{Related Work}
\label{sec-relatedwork}

This section discusses two kinds of related systems: those designed to cope
with high data-rates, and others designed to operate in scientific contexts.
We delay a discussion of architectural approaches related to Lance and IDEA
until Chapters~\ref{chapter-lance} and \ref{chapter-idea}, respectively.

\subsection{High Data-Rate Sensing}

Many sensor network applications involve the acquisition of high-resolution
signals using low-power wireless sensor nodes. Examples include monitoring
acoustic, seismic, and vibration waveforms in bridges, industrial equipment,
and animal habitats~\cite{girod-ipsn07,enviromic}. These systems all attempt
to acquire high data-rate (100~Hz or higher), high-fidelity data across the
network, subject to severe constraints on radio bandwidth and energy usage.

A group from UC Berkeley performed the largest deployment of sensor network
nodes for structural health monitoring: 46 nodes placed on the Golden Gate
Bridge in San Francisco~\cite{ggb-ipsn07}. Nodes collect vibration data at
1~kHz, and the network uses many of the same routing and time-synchronization
protocols used by our volcano-monitoring system. A special bulk
data-collection protocol, called Straw, was developed for the deployment. The
highly-linear topology of the deployed network later gave rise to the
topologies used to test the Flush data-collection
protocol~\cite{flush-sensys07}. The deployment at the Golden Gate Bridge also
gave rise to a system called the Structural Health Monitoring Toolkit
(Sentri). This interfaces between the outside world and the sensor network by
trasmitting commands to nodes as necessary.

NetSHM~\cite{netshm-ewsnsubmission,netshm-emnets05,wisan} is a wireless
sensor network for structural health monitoring, which involves studying the
response of buildings, bridges, and other structures to localize structural
damage, e.g., following an earthquake. This system shares many of the
challenges of geophysical monitoring; indeed, the data rates involved (500~Hz
per channel) are higher than are typically used in volcano studies. 

NetSHM implements reliable data collection using both hop-by-hop caching and
end-to-end retransmissions. Their work explores the use of local computations
on sensors to reduce bandwidth requirements. Rather than a global
time-synchronization protocol, the base station timestamps each sample upon
reception. The \textit{residence time} of each sample as it flows from sensor
to base is calculated based on measurements at each transmission hop and used
to deduce the original sample time.

Several factors distinguish our work on volcano science from structural
monitoring applications. First, structural monitoring networks typically
either collecte data following controlled excitations of a structure, or at
periodic intervals, which simplifies transmission scheduling. In our case,
volcanic activity is bursty and highly variable, requiring more sophisticated
approaches to event detection and data transfer. While the Golden Gate Bridge
system is sparsely deployed like our volcano sensor networks, many structural
monitoring applications are deployed in relatively dense networks, making
data collection and time synchronization more robust. 

Condition-based maintenance is another emerging area for wireless sensor
networks. The typical approach is to collect vibration waveforms from
equipment (e.g., chillers, pumps, etc.) and perform time- and
frequency-domain analysis to determine when the equipment requires servicing.
Intel Research has explored this area through two deployments at a
fabrication plant and an oil tanker in the North
Sea~\cite{intel-northseasensys}. Although this application involves high
sampling rates, it does not necessarily require time synchronization as
signals from multiple sensors need not be correlated. The initial evaluation
of these deployments only considers the network performance and does not
address data fidelity issues.

While many early environmental monitoring applications are characterized by
low data-rates, some have focused on applications requiring high-speed data
acquisition. A group at MIT led by Lew Girod has explored several generations
of hardware and software solutions for distributed acoustic monitoring, with
the driving application originally being monitoring colonies of
marmots.~\cite{girod-marmots}. This work led to the Acoustic
ENSBox~\cite{girod-ensbox}, a self-calibrating hardware solution designed to
be easy to deploy in support of acoustic sensing applications. The ENSBox
features an ARM processor, which puts it at a different point on the
power-performance curve from typical sensor network nodes and makes it more
suitable for the high-speed processing necessary to capture acoustic signals.

The software environment for the ENSBox was orignally provided by
EmStar~\cite{emstar}, which targets Linux-based platforms. More recently, the
ENSBox has been used as the basis of the VoxNet platform, an environment
designed for acoustic signal collection and processing.
VoxNet~\cite{voxnet-ipsn08} is comprised of three pieces:
Wavescope~\cite{wavescope}, a programming environment targetting
heterogeneous sensor networks. Wavescope programs are written in
WaveScript~\cite{wavescript-techreport08}, a stream-processing language.
Users compose a set of filters and other stream operators into a ``script''
similar to a data-flow graph.

The VoxNet platform includes a variety of network services, such as
time-synchronization, routing, and node localization that applications
running in this environment can make use of. Once the program is written and
installed on nodes, VoxNet includes a set of control and visualization tools
intended to allow users to interact with the running system and view the data
as it is collected. A system using VoxNet was deployed on 2008 at the Rocky
Mountain Biological Laboratory and used to study the alarm calls of marmots.
Acoustic monitoring has scientific applications to other species as well, as
a variety of animals and birds produce scientifically-interesting
vocalizations.

\XXXnote{TODO: Cane toad monitoring.}

\subsection{Scientific Sensing}

The first generation of sensor network deployments focused on distributed
monitoring of environmental conditions. Representative projects include the
Great Duck Island~\cite{spm:04habitat,polastre-masters,mainwaring-habitat},
Berkeley Redwood Forest~\cite{berkeley-redwoods}, and James
Reserve~\cite{cerpa-habitat} deployments. These systems are characterized by
low data rates (sampling intervals on the order of minutes) and very
low-duty-cycle operation to conserve power. Research in this area has made
valuable contributions in establishing sensor networks as a viable platform
for scientific monitoring and developing essential components used in our
work. 

This previous work has not yet focused on the efficacy of a sensor network as
a scientific instrument. The best example is the Berkeley Redwood Forest
deployment~\cite{berkeley-redwoods}, which involved 33~nodes monitoring the
microclimate of a redwood tree for 44~days. Their study focuses on novel
ways of visualizing and presenting the data captured by the sensor network,
as well as on the data yield of the system. The authors show that the
microclimactic measurements are consistent with existing models; however, no
ground truth of the data is established. This paper highlights many of the
challenges involved in using wireless sensors to augment or replace existing
scientific instrumentation.

\XXXnote{TODO: Vigilance (soil monitoring).}

\XXXnote{TODO: Turtle monitoring.}

\XXXnote{TODO: Glacier sensing.}

\XXXnote{TODO: James reserve.}
