\section{Related Work}
\label{sec-relatedwork}

This section discusses four kinds of related systems, those designed to cope
with high data-rates, designed to operate in scientific contexts, and related
to Lance or IDEA.

\subsection{High Data-Rate Sensing}

Many sensor network applications require collecting high-resolution signals
using low-power nodes, including monitoring acoustic, seismic, and vibration
waveforms in bridges, industrial equipment, and animal habitats. These
systems all attempt to acquire high data-rate ($>$ 100~Hz), high fidelity data
subject to energy and bandwidth constraints.

A group from UC Berkeley performed the largest deployment to date of sensor
network nodes for structural health monitoring: 46 nodes placed on the Golden
Gate Bridge~\cite{ggb-ipsn07}. Nodes collect vibration data at 1~kHz, and the
network uses many of the same routing and time synchronization protocols used
by our volcano monitoring system. A data collection protocol called Straw was
developed for the deployment, and the highly linear deployment topology was
later used to test the Flush data collection protocol~\cite{flush-sensys07}.
This project also produced the Structural Health Monitoring Toolkit (Sentri)
that interfaces between the outside world and the network by relaying
commands to nodes.

NetSHM is a wireless sensor network for structural health monitoring
developed to study the response of structures earthquake
damage~\cite{netshm-ewsnsubmission,netshm-emnets05,wisan}. Sampling data at
500~Hz per channel, this system shares many of the challenges of high
data-rate monitoring. The Wireless Modular Monitoring System (WiMMS) is
another structural monitoring network that has been validated in field
deployments at the Geumdang Bridge in Icheon, South
Korea~\cite{wimms-lynch06}. WiMMS supports a decentralized control algorithm
that responds to structural changes using actuators. Decentralization reduces
the amount of data that must be transmitted to a central location while
eliminating the base station as a point of failure.

NetSHM implements reliable data collection using hop-by-hop caching and
end-to-end retransmissions, using node computation to reduce bandwidth
requirements. Rather than a global time synchronization protocol, the base
station timestamps each sample upon reception. The \textit{residence time} of
each sample as it flows from sensor to base is calculated based on
measurements at each hop and used to compute the actual sample time.

Several factors distinguish our work from structural monitoring applications.
First, structural monitoring networks typically collect data either following
controlled excitations or at periodic intervals, simplifying transmission
scheduling. In contrast, volcanic activity is bursty and variable, requiring
more sophisticated approaches to event detection and data transfer. While the
Golden Gate Bridge system is sparsely deployed like our volcano sensor
networks, many structural monitoring applications are deployed in relatively
dense networks, making data collection and time synchronization more robust. 

Condition-based maintenance is another application of wireless sensor
networks. A typical approach is to collect vibration waveforms from equipment
and perform time- and frequency-domain analyses to determine when it requires
servicing. Intel Research has explored this area through deployments at a
fabrication plant and an oil tanker in the North
Sea~\cite{intel-northseasensys}. Although this application involves high
sampling rates, it does not necessarily require time synchronization as
signals from multiple sensors need not be correlated. 

While many environmental monitoring applications are characterized by low
data-rates, some, such as marmot monitoring, require high-speed data
acquisition. A group at MIT led by Lew Girod has developed several
generations of hardware and software solutions for distributed acoustic
monitoring driven by this application~\cite{girod-marmots}, including the
Acoustic ENSBox~\cite{girod-ensbox}, a self-calibrating hardware solution
designed for ease of deployment. The ENSBox features an ARM processor, which
consumes more power than the MSP430 but is more suited to the high-speed
processing necessary to capture acoustic signals.

The software environment for the ENSBox was originally provided by
EmStar~\cite{emstar}, which targets Linux-based platforms. More recently, the
ENSBox has been used as the basis of the VoxNet platform, an environment
designed for acoustic signal collection and processing.
VoxNet~\cite{voxnet-ipsn08} is comprised of three pieces: the Acoustic
ENSBox, a set of network services, and WaveScope~\cite{wavescope}, a
programming environment targeting heterogeneous sensor networks. WaveScope
programs are written in WaveScript~\cite{wavescript-techreport08}, a
stream-processing language. Users compose filters and other stream operators
into a ``script'' similar to a data-flow graph.

The VoxNet platform includes a set of network services that applications can
use, such as time synchronization, routing, and node localization. Once the
program is written and installed, VoxNet includes a set of control and
visualization tools allowing users to interact with the running system and
view the data as it is collected. A system using VoxNet was deployed in 2008
at the Rocky Mountain Biological Laboratory and used to study the alarm calls
of marmots. Acoustic monitoring has scientific applications to other species
as well, as a variety of animals and birds produce scientifically interesting
vocalizations.

Another high data-rate habitat monitoring application is the cane toad
monitoring project at Portland State University, CSIRO, and the University of
New South Wales. The cane toad is an invasive species in Australia, and their
spread is being monitored due to their impact on the country's native fauna.
The goals of the project are to design a system permitting \textit{in situ}
classification of various frog species based on their vocalizations.

After a pilot study, two iterations advanced the system's design. First, a
hybrid network was developed, mating low-power sensor nodes and middle-tier
devices with more advanced processing and storage capabilities. Data
reduction is performed on the sensor nodes to limit the information sent to
the high-power devices and prolonging the lifetime of the embedded
nodes~\cite{canetoad-tosn}. The second iteration explores compressive
sampling techniques and a classification algorithm that can run directly on
the resource-constrained sensor nodes.

Camera sensor networks also produce high data-rates and have been the subject
of considerable study. A group at UCLA built a system called
Cyclops~\cite{cyclops-sensys05} bringing vision technology to sensor nodes
through low-power imagers better suited to mating with resource-constrained
devices. They show that Cyclops nodes can successfully perform fundamental
vision recognition tasks such as object detection. Another team at UMass
Amherst sees motes fitted with imagers as forming one tier of a multi-tier
multi-modal ($M^2$) sensor network~\cite{m2-nossdav05}. $M^2$ networks
consist of several tiers operating at different power levels, combining the
capabilities of multiple types of devices. The lowest tier might consist of
sensor nodes equipped with vibration sensors and be used to trigger the
operation of more power-hungry devices --- such as Stargates~\cite{stargate}
fitted with webcams --- in the upper tiers.

\subsection{Scientific Sensing}

The first generation of sensor network deployments focused on environmental
monitoring. Representative projects include the Great Duck
Island~\cite{spm:04habitat,polastre-masters,mainwaring-habitat}, Berkeley
Redwood Forest~\cite{berkeley-redwoods}, and James
Reserve~\cite{cerpa-habitat} deployments. These systems are characterized by
low data-rates ($<<$ 1~Hz) and very low-duty-cycle operation to conserve
power. Research in this area has made valuable contributions in establishing
sensor networks as a viable platform for scientific monitoring and developing
essential components used in our work. 

Most previous work has not addressed the efficacy of a sensor network as a
scientific instrument. The best example is the Berkeley Redwood Forest
deployment~\cite{berkeley-redwoods}, which involved 33~nodes monitoring the
microclimate of a redwood tree for 44~days. Their study focuses on novel ways
of visualizing and presenting the captured data, as well as on the data yield
of the system. The authors show that their measurements are consistent with
existing models, but do not establish ground truth. This paper does highlight
the challenges involved in using wireless sensors to augment or replace
existing instrumentation.

A group at UCLA has built a system for soil monitoring and deployed it in the
AMARSS transect in the James Reserve, a biological field station operated by
the University of California. The goal was to augment a set of wired data
loggers with wireless sensor technology, similar to what we have attempted
with our volcano monitoring system. Since 2005 the deployed system has
collected over 26~million measurements, which are retrieved periodically by a
technician visiting the deployment site. The goal is to study the carbon
cycle and estimate the flux of carbon dioxide from the soil.

The largest challenge facing these researchers was coping with missing data.
To address it, they built a system called Suelo~\cite{suelo-sensys09}. Suelo
aids human researchers in monitoring the health of the deployment, as the
soil-monitoring sensors used are quite fragile. Suelo monitors captured
readings captured and tries to distinguish between interesting and faulty
data, initiating human intervention when sensors require replacement or
recalibration.

Similar to our work in its application to environmental hazard mitigation,
previous projects have explored sensor networks for flood and landslide
detection, at MIT and John Hopkins University, respectively. The MIT system
feeds data from a network of sensors into a model designed to predict
rainfall-triggered flooding~\cite{basha-sensys08}. They were able to
demonstrate the accuracy of their approach using data collected over multiple
field deployments on Honduras between 2004 and 2007. JHU's approach to
landslide detection employs sensor columns, vertical underground
installations of several different sensors~\cite{landslide-sensys05}. When a
portion of the network detects a slip surface forming, nodes estimate the
position of the slip surface which is fed into a model that predicts whether
and when a landslide will occur. Both of these projects differ from our work
by involving significant modeling and prediction components, whereas we have
focused primarily on raw data collection.

Ongoing work at Washington State University uses camera-dropped sensor
networks to monitor the activity of Mt. St.
Helens~\cite{wsuvolcano-mobisys09}. This project shares many of our goals,
including ease of deployment and data fidelity, but differs in its focus on
system lifetimes of up to 1~year, and in its aim to replace, rather than
augment, existing volcano monitoring stations. The reliance on helicopter
support while deploying nodes also leads to a different set of design
decisions, including much larger batteries and more capable nodes.

The WSU group uses an iMote2~\cite{imote2} sensor node, which shares the
CC2420 802.15.4 radio with our TMote but features the PXA271 XScale processor
with significantly more computational horsepower than the MSP430. The
expanded form factor and power budget permits the use of GPS at each node,
simplifying time synchronization. While the system aims to provide continuous
data collection, the limited bandwidth of the 802.15.4 radios on each sensor
node and the 900~MHz Freewave~\cite{freewave} modem used to maintain a
connection with the observatory limit the data that can be collected. Similar
to our volcano-monitoring system, an impulse detector implemented as a
short-term average over long-term average (STA/LTA) is used to assign a
priority to the data as it is collected. Data prioritization is used during
data collection and routing to ensure that data from detected events reaches
the observatory first. Currently the system addresses energy limitations by
using heavy AIR-ALKALINE batteries, feasible assuming helicopter support.

\subsection{Data Quality Optimization Frameworks}
\label{lance-sec-related}

Several systems are related to Lance but differ in their goals and
assumptions.

EnviroMic~\cite{enviromic} is designed to support distributed acoustic
recording by leveraging the collective storage resources of multiple sensor
nodes. When multiple nodes detect the same event, it performs cooperative
recording by organizing nodes into groups and rotating recording within the
group as long as the event continues. This is intended to reduce the amount
of data that must be stored by eliminating redundant signal collection.

\clearpage

EnviroMic also distributed storage load within the network to ensure that the
all storage can be used and signals not lost due to full Flash drives. While
nodes may continuously exchange data to rebalance storage, the fundamental
assumption of the architecture is that data will be manually retrieved from
sensor nodes following the deployment. Unlike Lance, EnviroMic is not
intended for applications with real-time data needs.

ICEDB~\cite{zhang2007icedb} supplies a delay-tolerant and priority-driven
query processor for the CarTel~\cite{cartel} system. ICEDB provides SQL
extensions allowing queries to assign both inter- and intra-stream
priorities, which are used by the query processor to manage bandwidth and
storage resources. ICEDB also uses a similar node-level summarization
technique to that used by Lance. While ICEDB considers bandwidth limitations,
it does not consider energy as a constraint, due to readily-available power
in the vehicles where nodes are deployed. The fundamental goal of ICEDB ---
to provide database-like access to mobile nodes that may experience periods
of disconnection or poor connectivity --- differs from that of Lance,
explaining the architectural differences. CarTel nodes are much higher-power
and assumed to be attached to power sources in the vehicles where they are
deployed.

VanGo~\cite{vango} provides an architecture for collecting and processing
high-resolution sensor data on resource-constrained nodes. VanGo provides a
programming model based on a linear filter chain and focuses on implementing
efficient signal-processing operations with limited computational power.
WaveScope~\cite{wavescope} and Flask~\cite{flask-tr} are languages for stream
processing applications. These systems are largely complementary to Lance,
and while they could be used to process signal data prior to collection, our
focus is on collecting \textit{raw} sensor data from large networks. These
systems do not attempt to optimize data collection under varying energy and
bandwidth constraints. 

\subsection{Energy Load-Balancing Services}
\label{idea-sec-related}

Previous work has addressed the problem of energy load balancing in contexts
such as sensor coverage, role assignment, and energy-aware routing. Other
efforts in sensor networks have focused on reducing the power consumption at
individual nodes without considering energy distribution. Many of these
efforts are specific to a particular application or component and do not
provide a service that can be used by a variety of applications. 

A number of existing systems such as Odyssey~\cite{odyssey-osr99},
PowerScope~\cite{powerscope-wmcsa99} and more recently
Cinder~\cite{cinder-mobiheld09}, have addressed measuring or adapting to
energy variations on battery-powered devices, primarily to support mobile
applications. This naturally produces a difference in approach from IDEA,
since IDEA targets networks of multiple nodes treated as a single entity.
Since nodes are collaborating we expect nodes to sacrifice for each other,
whereas mobile device users would likely be upset if they discovered that
their phone was running low on power trying to improve the lifetime of a
nearby stranger's phone.

Quanto~\cite{quanto-osdi08} provides a framework for tracking and
understanding energy consumption in embedded sensor systems. The existence of
systems such as Quanto was a primary motivation for IDEA, since the
visibility distributed resource tracking provides creates an opportunity to
adapt to changes in availability across the network. Currently IDEA requires
that components model their own energy consumption, which may be difficult
for components with complex behavior. We are exploring integrating Quanto and
IDEA to provide more precise tracking of energy at runtime, which could
eliminate the need for component-specific modeling and ease application
integration.

Computer scientists at UMass Amherst have deployed GPS-enabled sensors to
study the movement of a threatened species of turtle, which led to Eon, a
language and runtime system designed to enable energy-aware programming. Eon
claims to be the first energy-aware programming language and tries to make
resource usage explicit to programmers by allowing them to annotate their
code with energy states. At runtime, Eon will adapt the behavior of the node
based on resource availability. Eon~\cite{eon-sensys07} performs similar
energy tracking and forward projection but focuses on single-node, not
network-wide adaptations.

SORA~\cite{sora-nsdi05} performs decentralized resource allocation based on
an economic model where nodes respond to incentives to produce data or
perform specific tasks and each node attempts to maximize its profit for
taking a series of actions. While SORA, using correctly set prices, could
produce network-wide behavior similar to IDEA, the connection between prices
and the behavior of the network is not clear. IDEA simplifies the problem of
global network control through the energy objective function encapsulating
the application's goal.

Some work on energy-aware routing~\cite{ShahRabaey2002,381685} has addressed
equitable energy distribution within the network by probabilistically
choosing between multiple good paths between each source and sink pair.
LEACH~\cite{leach} and other similar approaches attempt to distribute energy
in an entirely decentralized way, using local heuristics to do so.
Lexicographically maximum rate allocation~\cite{fairrate-sensys08} uses a
decentralized algorithm to tune optimal data collection rates in perpetual
networks when static routes are used, all nodes route to a single sink, and
the recharging profiles of the nodes are known ahead of time.

VigilNet~\cite{vigilnet} is a target-tracking system that attempts to save
energy by rotating ``sentry'' duties between a group of nearby nodes. Nodes
that are not assigned as sentries can sleep and conserve energy while the
sentry monitors the area. When an event occurs, the sentry awakes nearby
nodes and initiates the tracking process. VigilNet assigns sentry duties
based on each node's energy availability while trying to ensure that the
entire area being monitoring is covered. Given the desire both to extend
system lifetime and maintain a desired application fidelity, VigilNet could
be implemented using IDEA, allowing both energy availability and harvesting
to be considered when assigning sentries.

\clearpage

The IDEA architecture emerged from our own prior work on energy management
for wireless sensor networks, including Lance (Chapter~\ref{chapter-lance})
Pixie~\cite{pixie-sensys08}, and Peloton~\cite{peloton-hotos09}. Pixie
proposed an operating system and programming framework for sensor network
nodes that promotes resources to a first-class primitive, using tickets to
manage resource consumption and brokers to enable specialized management
policies. Pixie does not consider the energy impact of a node on other nodes.

Peloton proposed an architecture for distributed resource management in
sensor networks combining state sharing, vector tickets to represent
distributed resource consumption, and a decentralized architecture in which
nodes serve as ticket agents managing resource consumption for themselves and
nearby nodes. IDEA shares many features with Peloton and can be viewed as the
beginning of an implementation of the Peloton design, with state sharing to
enable energy decision making and every node serving as a ticket agent for
itself but considering the distributed impact of its own local state.
