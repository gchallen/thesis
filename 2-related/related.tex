\section{Related Work}
\label{sec-relatedwork}

This section discusses four kinds of related systems: ones designed to cope
with high data-rates, ones designed to operate in scientific contexts, and
those related to Lance and IDEA, respectively.

\subsection{High Data-Rate Sensing}

Many sensor network applications involve the acquisition of high-resolution
signals using low-power wireless sensor nodes. Examples include monitoring
acoustic, seismic, and vibration waveforms in bridges, industrial equipment,
and animal habitats~\cite{girod-ipsn07,enviromic}. These systems all attempt
to acquire high data-rate (100~Hz or higher), high-fidelity data across the
network, subject to severe constraints on radio bandwidth and energy usage.

A group from UC Berkeley performed the largest deployment of sensor network
nodes for structural health monitoring: 46 nodes placed on the Golden Gate
Bridge in San Francisco~\cite{ggb-ipsn07}. Nodes collect vibration data at
1~kHz, and the network uses many of the same routing and time-synchronization
protocols used by our volcano-monitoring system. A special bulk
data-collection protocol, called Straw, was developed for the deployment. The
highly linear topology of the deployed network later gave rise to the
topologies used to test the Flush data-collection
protocol~\cite{flush-sensys07}. The deployment at the Golden Gate Bridge also
gave rise to a system called the Structural Health Monitoring Toolkit
(Sentri). This interfaces between the outside world and the sensor network by
transmitting commands to nodes as necessary.

NetSHM~\cite{netshm-ewsnsubmission,netshm-emnets05,wisan} is a wireless
sensor network for structural health monitoring, which involves studying the
response of buildings, bridges, and other structures to localize structural
damage, e.g., following an earthquake. This system shares many of the
challenges of geophysical monitoring; indeed, the data rates involved (500~Hz
per channel) are higher than are typically used in volcano studies. The
Wireless Modular Monitoring System (WiMMS)~\cite{wimms-lynch06} is another
structural monitoring network with similar goals that has been validated both
in field deployments at the Geumdang Bridge in Icheon, South Korea, and in
laboratory tests. This system is designed around a supporting decentralized
control algorithms in order to respond to structural changes using actuators.
In this context decentralization reduces the amount of data that must be
transmitted to a central location while eliminating the base station as a
single point-of-failure.

NetSHM implements reliable data collection using both hop-by-hop caching and
end-to-end retransmissions. The work explores the use of local computations
on sensors to reduce bandwidth requirements. Rather than a global
time-synchronization protocol, the base station timestamps each sample upon
reception. The \textit{residence time} of each sample as it flows from sensor
to base is calculated based on measurements at each transmission hop and used
to deduce the original sample time.

Several factors distinguish our work on volcano science from structural
monitoring applications. First, structural monitoring networks typically
either collect data following controlled excitations of a structure or at
periodic intervals, which simplifies transmission scheduling. In our case,
volcanic activity is bursty and highly variable, requiring more sophisticated
approaches to event detection and data transfer. While the Golden Gate Bridge
system is sparsely deployed like our volcano sensor networks, many structural
monitoring applications are deployed in relatively dense networks, making
data collection and time synchronization more robust. 

Condition-based maintenance is another emerging area for wireless sensor
networks. The typical approach is to collect vibration waveforms from
equipment (e.g., chillers, pumps, etc.) and perform time- and
frequency-domain analysis to determine when the equipment requires servicing.
Intel Research has explored this area through two deployments at a
fabrication plant and an oil tanker in the North
Sea~\cite{intel-northseasensys}. Although this application involves high
sampling rates, it does not necessarily require time synchronization as
signals from multiple sensors need not be correlated. The initial evaluation
of these deployments only considers the network performance and does not
address data fidelity issues.

While many early environmental monitoring applications are characterized by
low data-rates, some have focused on applications requiring high-speed data
acquisition. An example application is monitoring colonies of marmots. A
group at MIT led by Lew Girod has explored several generations of hardware
and software solutions for distributed acoustic monitoring driven by this
application~\cite{girod-marmots}. This work produced the Acoustic
ENSBox~\cite{girod-ensbox}, a self-calibrating hardware solution designed to
be easy to deploy in support of acoustic sensing applications. The ENSBox
features an ARM processor, which puts it at a different point on the
power-performance curve from typical sensor network nodes and makes it more
suitable for the high-speed processing necessary to capture acoustic signals.

The software environment for the ENSBox was originally provided by
EmStar~\cite{emstar}, which targets Linux-based platforms. More recently, the
ENSBox has been used as the basis of the VoxNet platform, an environment
designed for acoustic signal collection and processing.
VoxNet~\cite{voxnet-ipsn08} is comprised of three pieces:
Wavescope~\cite{wavescope}, a programming environment targeting
heterogeneous sensor networks. Wavescope programs are written in
WaveScript~\cite{wavescript-techreport08}, a stream-processing language.
Users compose a set of filters and other stream operators into a ``script''
similar to a data-flow graph.

The VoxNet platform includes a variety of network services, such as
time-synchronization, routing, and node localization that applications
running in this environment can make use of. Once the program is written and
installed on nodes, VoxNet includes a set of control and visualization tools
intended to allow users to interact with the running system and view the data
as it is collected. A system using VoxNet was deployed in 2008 at the Rocky
Mountain Biological Laboratory and used to study the alarm calls of marmots.
Acoustic monitoring has scientific applications to other species as well, as
a variety of animals and birds produce scientifically-interesting
vocalizations.

Another application of high data-rate sensing to habitat monitoring is the
cane toad monitoring project run by a team at Portland State University,
CSIRO, and the University of New South Wales. The cane toad is an invasive
species in Australia, and their spread is being monitored due to concerns
about their impact on the country's native fauna. The goals of the project
are to design a system permitting \textit{in situ} classification of various
frog species based on their vocalizations.

After completing a pilot study, two additional iterations advanced the design
of the system. In the first, a hybrid network was developed, mating low-power
sensor nodes and middle-tier devices with more advanced processing and
storage capabilities. Data reduction is performed on the sensor nodes in
order to limit the amount of information that must be sent to the high-power
devices, thus prolonging the lifetime of the embedded
nodes~\cite{canetoad-tosn}. The second iteration explores compressive
sampling techniques and also deploys a classification algorithm that can be
run directly on the resource-constrained sensor nodes.

\XXXnote{GWA: TODO: Add Cyclops and other camera-based sensor networks.}

\subsection{Scientific Sensing}

The first generation of sensor network deployments focused on distributed
monitoring of environmental conditions. Representative projects include the
Great Duck Island~\cite{spm:04habitat,polastre-masters,mainwaring-habitat},
Berkeley Redwood Forest~\cite{berkeley-redwoods}, and James
Reserve~\cite{cerpa-habitat} deployments. These systems are characterized by
low data rates (sampling intervals on the order of minutes) and very
low-duty-cycle operation to conserve power. Research in this area has made
valuable contributions in establishing sensor networks as a viable platform
for scientific monitoring and developing essential components used in our
work. 

This previous work has not yet focused on the efficacy of a sensor network as
a scientific instrument. The best example is the Berkeley Redwood Forest
deployment~\cite{berkeley-redwoods}, which involved 33~nodes monitoring the
microclimate of a redwood tree for 44~days. Their study focuses on novel ways
of visualizing and presenting the data captured by the sensor network, as
well as on the data yield of the system. The authors show that the
microclimactic measurements are consistent with existing models, but ground
truth of the data is not established. This paper does highlight many of the
challenges involved in using wireless sensors to augment or replace existing
scientific instrumentation.

A group at UCLA has built a system for soil monitoring and deployed it in the
AMARSS transect in the James Reserve, a biological field station operated by
the University of California. The goal was to augment a set of wired data
loggers with wireless sensor technology, similar in spirit to what we have
attempted with our volcano monitoring system. Since 2005 the deployed system
has collected over 26~million measurements, which are retrieved periodically
by a technician visiting the deployment site. The goal is to study the carbon
cycle and estimate the flux of carbon dioxide from the soil.

The largest challenge facing these researchers was coping with missing data.
To address it, they built a system called Suelo~\cite{suelo-sensys09}.  Suelo
is intended to aid human researchers in monitoring and assisting the health
of the deployment, sense the soil-monitoring sensors used are quite fragile.
Suelo monitors the readings capture by the network and tries to distinguish
between ``interesting'' and ``faulty'' data, and initiating human
intervention when sensors require replacement or recalibration.

Computer scientists at UMass Amherst have been deploying GPS-enabled sensors
to study the movement of a threatened species of turtle. This work led to
Eon, a language and runtime system designed to enable energy-aware
programming. Eon claims to be the first energy-aware programming language and
tries to make resource usage explicit to programmers by allowing them to
annotate their code with energy states. At runtime, Eon will adapt the
behavior of the node based on resource availability.

\section{Data Quality Optimization Frameworks}
\label{lance-sec-related}

Several systems are related to Lance but differ substantially in their goals
and assumptions.

EnviroMic~\cite{enviromic} is a system designed to support distributed
acoustic recording by leveraging the collective storage resources of multiple
sensor nodes. It performs cooperative recording by organizing nodes into
groups when multiple nodes detect the same acoustic event, and using these
groups to ensure that only one node is recording acoustic data for as long as
the event of interest continues. This is intended to reduce the amount of
data that must be stored by eliminating redundant signal collection.

EnviroMic also focuses on distributing the storage load within the network to
ensure that the distributed storage can be utilized and signals of interest
not lost due to full Flash drives. While nodes may exchange data to rebalance
storage during the experiment, the fundamental assumption of the architecture
is that data will be manually retrieved from sensor nodes following the
deployment. Unlike Lance, EnviroMic is not intended for applications with
real-time data needs.

ICEDB~\cite{zhang2007icedb} supplies a delay-tolerant and priority-driven
query processor for the CarTel~\cite{cartel} system. ICEDB provides SQL
extensions allowing queries to assign both inter- and intra-stream
priorities, which are used by the query processor to manage bandwidth and
storage resources. ICEDB also uses a similar node-level summarization
technique to that used by Lance.

While ICEDB considers bandwidth limitations, it does not consider energy as a
constraint. The fundamental goal of ICEDB --- to provide database-like access
to mobile nodes that may experience periods of disconnection or poor
connectivity --- differs from that of Lance, which explains the architectural
differences. CarTel nodes are much higher-power and assumed to be attached to
power sources in the vehicles that they are deployed in.

VanGo~\cite{vango} provides an architecture for collecting and processing
high-resolution sensor data on resource-constrained nodes. VanGo focuses on a
programming model based on a linear filter chain and implementing efficient
signal-processing operations with limited computational power.
WaveScope~\cite{wavescope} and Flask~\cite{flask-tr} are languages for stream
processing applications. These systems are largely complementary to Lance,
and could be used to process signal data prior to collection, although our
focus is on collecting \textit{raw} sensor data from large networks. These
systems do not attempt to optimize data collection under varying energy and
bandwidth constraints. 

\XXXnote{GWA: TODO: Add Vigilnet.}

\section{Energy Load-Balancing Services}
\label{idea-sec-related}

Previous work has addressed the problem of energy load balancing in contexts
such as sensor coverage, role assignment, and energy-aware routing. Other
efforts in sensor networks have focused on reducing the power consumption at
individual nodes without considering energy distribution. Many of these
efforts are specific to a particular application or component and do not
provide a service like IDEA that can be used by a variety of applications. 

A number of existing systems such as Odyssey~\cite{odyssey-osr99},
PowerScope~\cite{powerscope-wmcsa99} and more recently
Cinder~\cite{cinder-mobiheld09}, have addressed measuring or adapting to
energy variations on battery-powered devices, primarily to support mobile
applications. This naturally produces a difference in approach from IDEA,
since IDEA targets networks consisting of multiple nodes but treated as a
single entity. Since nodes are collaborating we can enable more sharing and
ask nodes to sacrifice for each other, whereas mobile device users would
likely be upset if they discovered that their phone was running low on power
because it was trying to improve the lifetime of a stranger's phone located
nearby.

Quanto~\cite{quanto-osdi08} provides a framework for tracking and
understanding energy consumption in embedded sensor systems. The existence of
systems such as Quanto was a primary motivation for IDEA, since the
visibility distributed resource tracking provides creates an opportunity to
adapt to changes in availability across the network. Currently IDEA requires
that components model their own energy consumption, which may be difficult
for components with complex behavior. We are exploring integrating Quanto
into IDEA to provide more precise tracking of energy at runtime, which could
eliminate the need for component-specific modeling and ease the process of
integrating applications with IDEA.

Eon~\cite{eon-sensys07} performs similar energy tracking and forward
projection but focuses on single-node, not network-wide adaptations.
SORA~\cite{sora-nsdi05} focuses on decentralized resource allocation based on
an economic model in which nodes respond to incentives to produce data or
perform specific tasks, with each node trying to maximize its profit for
taking a series of actions. While SORA, using correctly set prices, could
produce similar network-wide behavior to that enabled by IDEA, the connection
between prices and the behavior of the network is not completely clear. IDEA
simplifies the problem of global network control through the energy objective
function which encapsulates the application's goal.

Some work on energy-aware routing~\cite{ShahRabaey2002,381685} has addressed
equitable energy distribution within the network by probabilistically
choosing between multiple good paths between each source and sink pair.
LEACH~\cite{leach} and other similar approaches attempt to distributed energy
in an entirely decentralized way, using local heuristics to do so.
Lexicographically maximum rate allocation~\cite{fairrate-sensys08} uses a
decentralized algorithm to tune optimum data collection rates in perpetual
networks when static routes are used, all nodes route to a single sink, and
the recharging profiles of the nodes are known ahead of time. Rate allocation
could be implemented in IDEA and comparing the two is planned future work.

EnviroMic~\cite{enviromic} is a distributed acoustic storage system for
sensor networks. When EnviroMic nodes hear an acoustic event, a leader is
elected to assign recording tasks to nodes in the group. As storage space is
limited, EnviroMic attempts to push data to quiet sections of the network
with unused storage, balancing storage consumption across the network. Both
of these tasks involve choosing from a set of nodes that can perform the same
storage task, and so EnviroMic could be integrated with IDEA allowing the
energy overheads of data transfers to be considered.

The IDEA architecture emerged from our own prior work on energy management
for wireless sensor networks, including Lance (Chapter~\ref{chapter-lance})
Pixie~\cite{pixie-sensys08}, and Peloton~\cite{peloton-hotos09}. Pixie
proposed an operating system and programming framework for sensor network
nodes that promotes resources to a first-class primitive, using tickets to
manage resource consumption and brokers to enable specialized management
policies. Pixie does not consider the energy impact of a node on other nodes.

Peloton proposed an architecture for distributed resource management in
sensor networks combining state sharing, vector tickets to represent
distributed resource consumption and a decentralized architecture in which
nodes serve as ticket agents managing the resource consumption of themselves
and on behalf of nearby nodes. IDEA shares many features with Peloton and can
be viewed as the beginnings of an implementation of the Peloton design, with
state sharing to enable energy decision making and every node serving as a
ticket agent for itself but considering the distributed impact of its own
local state.
