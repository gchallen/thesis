\section{Motivation}
\label{lance-sec-motivation}

Wireless sensor networks are becoming more common for applications that focus
on reliable collection of raw signals at relatively high sampling rates, as
opposed to in-network aggregation of low-data-rate samples. These
applications generally make use of extensive offline analysis to study the
collected data, and it is often infeasible or impossible to perform this
computation within the network itself. Even in cases where it is possible to
shift computation to the network, a system designer may wish to extract raw
data occasionally for calibration or testing.

These systems typically record data to flash at each sensor node and make use
of a reliable bulk-transfer protocol to collect data at a base station. Given
that the network is capable of sampling data at a higher rate than it can be
downloaded, it is not possible to collect the complete signal from all nodes.
These systems may employ lossless compression to reduce bandwidth
requirements, but once the network grows large enough or sampling rates grow
high enough the aggregate sampling rate will outstrip the bulk-transfer
protocol's bandwidth. The system is therefore forced to make decisions about
what data to collect and what to throw away. In most cases these decisions
are application-specific: for example, a volcanologist may be chiefly
interested in a specific type of seismic tremor, while a doctor may be
looking for a particular type of muscular tremor associated with a specific
condition. The implication is that the system must be able to determine the
intrinsic \textit{value} of a given signal to determine whether resources
should be devoted to storing and downloading that signal.

The typical approach to download management is a FIFO model where downloads
occur in a round-robin fashion across the network once a trigger occurs. In
general, new data may be sampled and stored to flash while a download is
taking place. Therefore, the trigger frequency, download cycle duration and
the number of nodes in the network all effect the amount of data captured by
such an approach. For example, the Flush~\cite{flush-sensys07} protocol
achieves only 8~Kbps for a reliable transfer over multiple hops; the Fetch
(described in Section~\ref{evaluation-subsec-fetch} and
Straw~\cite{ggb-ipsn07} protocols fare somewhat worse. The RCRT
protocol~\cite{rcrt-sensys07} is designed for a case where all nodes are
transmitting simultaneously to the sink, although this approach severely
limits the obtained per-node throughput. As a result, when incoming data
rates exceed download capacity, FIFO download management can produce
excessive delays between data acquisition and retrieval.

Lance assumes that sensor nodes contain adequate flash storage to buffer
signals prior to download. While the popular TMote~Sky platform supports a
relatively small 1~MB flash, more recent sensor designs~\cite{shimmer} have
several GB~of flash, and we expect this trend to continue. Rather than
focusing on per-node storage, our primary concern is with limitations on
bandwidth and energy.

Our goal is to develop a general-purpose approach to bandwidth and energy
management that complements a reliable data-collection protocol. Such a
system should have several key properties. First, it should be
\textit{customizable}, allowing different applications to specify their own
policies for storage management and bandwidth prioritization. Second, the
system should target a range of optimization goals. Examples include
maximizing overall data priority, bounding energy consumption, maximizing
temporal or spatial coverage of the collected data, or achieving fairness
across sensor nodes. Third, the system should be decoupled from a specific
routing protocol, reliable collection protocol, or sensor node platform,
making it possible to leverage the system in different settings.
