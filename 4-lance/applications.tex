\section{Adaptating Applications to Lance}
\label{lance-sec-adaptation}

As mentioned, the Lance approach grew out of challenges emerging from our
2005 field deployment at Reventador. Although this system was successfully
deployed, it exhibited several deficiencies which led to a significant loss
of data~\cite{volcano-osdi06}.

The first problem is that the decision used to download a given signal was
based on a simplistic binary approach, based on the event-detection algorithm
running on each node. As a result, the system could not prioritize certain
events over others. The event-detector logic used a simple threshold scheme,
and as reported in~\cite{volcano-osdi06}, the threshold was set too low,
causing the network to trigger on less than 5\% of the actual seismic events.

The second problem was that following each trigger, the network initiated a
\textit{nonpreemptive} download from every node in the network in a
round-robin fashion. This policy caused the system to devote resources to
downloading small precursor earthquakes that immediately preceded larger
eruptions~\cite{volcano-osdi06}. As a result, many such larger events were
not captured.

Finally, our 2005 system made no attempt to manage energy. As a result, the
expected lifetime of the network is only about a week (using D-cell
batteries), necessitating frequent battery changes over a long deployment.
Clearly, this system could benefit from a prioritized approach to download
management that also considers energy costs to increase lifetime.

\subsection{Volcano Monitoring}
\label{lance-subsec-volcano}

To address these problems, we reimplemented our previous volcano monitoring
system using Lance. Many of the components of the original system, such as
multihop routing, time synchronization, reliable download protocol, and flash
storage interface, remained unchanged. The node-level event detector was
replaced by an ADU summarization function, as described below. The base
station code for responding to correlated events was replaced with Lance's
optimizer and policy modules. Our deployment of the completed system at
Tungurahua volcano in August 2007 is discussed in
Section~\ref{lance-sec-deployment}.

The original system was intended to detect correlated seismic events from
across the network and download data from all nodes, regardless of whether
every node detected the event. This was based on an event detector that
computes two exponentially-weighted moving averages (EWMA) of the seismic
signal with different gain settings, described previously in
Section~\ref{evaluation-sec-eventdetection}.

This policy is straightforward to implement in Lance by using the ``ratio of
two averages'' as the node-level summarization function. Rather than
performing thresholding at the node level, we report the maximum ratio over
the ADU as its value, allowing Lance to prioritize different events. The base
station's policy modules are configured as shown in
Section~\ref{lance-sec-example-policies}, using a chain of \texttt{filter},
\texttt{correlated}, and \texttt{spacespread} to implement the equivalent of
the event triggering policy used in the original system. Note that the Lance
version of the system differs from the original in that download management
is value-driven rather than FIFO. Also, Lance can download ADUs from
different events out of order, avoiding the nonpreemptive download problems
of the earlier system.

While our original system was designed to capture short earthquakes, were
also interested in determining whether Lance could be used to capture
different types of volcanic activity. For this, we make use of the Real-Time
Seismic Amplitude Measurement (RSAM)~\cite{rsam}, which computes the average
seismic amplitude over a given time window. Intuitively, RSAM measures the
total amount of ground shaking caused by earthquakes and tremor, and is often
used by volcano observatories to characterize the overall level of seismic
activity.

Different summarization functions and policy modules can be used to implement
a wide range of geophysical monitoring systems with Lance. For example, a
hazard monitoring system could be configured to periodically report RSAM
values for all sensor nodes and download only the strongest events for
further analysis. By limiting downloads to those ADUs with RSAM above some
threshold, energy can be saved. In contrast, a scientific study that wishes
to perform earthquake localization~\cite{aki-richards-80} or tomographic
inversion~\cite{lees-lindley-94} would prefer to download only small
earthquakes with clearly delineated onsets, which can be used to determine
the velocities of seismic waves. Likewise, a researcher studying explosive
events would prefer to download only seismic events with a corresponding
infrasonic component, since non-explosive earthquakes should not generate any
infrasound.

\subsection{Other Application Domains}

We believe that Lance can be used to benefit many applications that make use
of high-resolution signals delivered over a bandwidth-limited wireless
network. These applications require high data rates, precluding continuous
data collection, and rely on classification techniques to determine which
signals to download. Two examples follow.

\begin{enumerate}

\item \textbf{Structural monitoring:} Structural monitoring systems collect
vibration waveforms from a building, bridge, or other structure in order to
study structural properties and seismic response. In previous
systems~\cite{netshm-emnets05,ggb-ipsn07}, data collection has been triggered
manually or on a simple periodic schedule. Instead, Lance can be used to
prioritize signals following an earthquake or forced excitation of the
structure, similar to the EWMA and RSAM functions described earlier. To save
energy, the system could choose a subset of nodes from which to download data
to achieve a good spatial distribution across the structure. The size of the
subset could be chosen depending on the strength of the excitation. In
addition, policy modules can be used to perform periodic downloading of ADUs
from each sensor for calibration, as well as to determine whether each sensor
is still functioning properly.

\item \textbf{Animal habitat monitoring:} Habitat monitoring applications
that deploy high-bandwidth sensors, such as microphones or cameras, are good
candidates for prioritized data extraction. An example application may
attempt to download interesting audio signals facilitating offline species
classification or localization~\cite{girod-ipsn07}. The summarization
function could involve either a triggered event detector, an audio waveform
classifier, or motion detector from a series of camera images~\cite{cyclops}.

At the base station, policy modules can use offline knowledge of node
positions to modify the initial ADU value. One approach might enhance spatial
coverage by prioritizing data collection from nodes nearby the source of the
signal. Another could reject noise by deprioritizing signals detected by only
one node. For example, if fewer than three nodes report an audio event, it is
impossible to perform acoustic localization and Lance need not waste
bandwidth on the signal. Policy modules can take other metrics into account
as well, such as the SNR of the recorded signal or the time of day (e.g.,
reducing confidence in camera images taken at night).

\end{enumerate}
