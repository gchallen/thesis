\section{Related Work}
\label{lance-sec-related}

\XXXnote{GWA: TODO: Need to flesh this out more.}

Several systems are related to Lance but differ substantially in their goals
and assumptions. EnviroMic~\cite{enviromic} is a system designed to support
distributed acoustic recording by leveraging the collective storage resources
of multiple sensor nodes. EnviroMic focuses on storage management and load
balancing, and assumes that data will be manually retrieved from sensor nodes
following the deployment. Unlike Lance, EnviroMic is not intended for
applications with real-time data needs.

ICEDB~\cite{zhang2007icedb} supplies a delay-tolerant and priority-driven
query processor for the CarTel~\cite{cartel} system. While ICEDB considers
bandwidth limitations, it does not consider energy as a constraint. ICEDB
provides SQL extensions allowing queries to assign both inter- and
intra-stream priorities, which are used by the query processor to manage
bandwidth and storage resources. ICEDB also uses a similar node-level
summarization technique to that used by Lance. 

VanGo~\cite{vango} provides an architecture for collecting and processing
high-resolution sensor data on resource-constrained nodes. VanGo focuses on a
programming model based on a linear filter chain and implementing efficient
signal-processing operations with limited computational power.
WaveScope~\cite{wavescope} and Flask~\cite{flask-tr} are languages for stream
processing applications. These systems are largely complementary to Lance,
and could be used to process signal data prior to collection, although our
focus is on collecting \textit{raw} sensor data from large networks. These
systems do not attempt to optimize data collection under varying energy and
bandwidth constraints. 
