\section{Related Work}
\label{lance-sec-related}

Several systems are related to Lance but differ substantially in their goals
and assumptions.

EnviroMic~\cite{enviromic} is a system designed to support distributed
acoustic recording by leveraging the collective storage resources of multiple
sensor nodes. It performs cooperative recording by organizing nodes into
groups when multiple nodes detect the same acoustic event, and using these
groups to ensure that only one node is recording acoustic data for as long as
the event of interest continues. This is intended to reduce the amount of
data that must be stored by eliminating redundant signal collection.

EnviroMic also focuses on distributing the storage load within the network to
ensure that the distributed storage can be utilized and signals of interest
not lost due to full Flash drives. While nodes may exchange data to rebalance
storage during the experiment, the fundamental assumption of the architecture
is that data will be manually retrieved from sensor nodes following the
deployment. Unlike Lance, EnviroMic is not intended for applications with
real-time data needs.

ICEDB~\cite{zhang2007icedb} supplies a delay-tolerant and priority-driven
query processor for the CarTel~\cite{cartel} system. ICEDB provides SQL
extensions allowing queries to assign both inter- and intra-stream
priorities, which are used by the query processor to manage bandwidth and
storage resources. ICEDB also uses a similar node-level summarization
technique to that used by Lance.

While ICEDB considers bandwidth limitations, it does not consider energy as a
constraint. The fundamental goal of ICEDB --- to provide database-like access
to mobile nodes that may experience periods of disconnection or poor
connectivity --- differs from that of Lance, which explains the architectural
differences. CarTel nodes are much higher-power and assumed to be attached to
power sources in the vehicles that they are deployed in.

VanGo~\cite{vango} provides an architecture for collecting and processing
high-resolution sensor data on resource-constrained nodes. VanGo focuses on a
programming model based on a linear filter chain and implementing efficient
signal-processing operations with limited computational power.
WaveScope~\cite{wavescope} and Flask~\cite{flask-tr} are languages for stream
processing applications. These systems are largely complementary to Lance,
and could be used to process signal data prior to collection, although our
focus is on collecting \textit{raw} sensor data from large networks. These
systems do not attempt to optimize data collection under varying energy and
bandwidth constraints. 
