\section{Policy Modules}
\label{lance-sec-policies}

Policy modules provide an interface through which applications can tune the
operation of the Lance optimizer. Since policy modules are loaded into the
system at the base station, they can be modified at any time without
necessitating reprogramming of the sensor nodes themselves. In this section
we provide a general discussion of policy modules and examples of policies
that can be implemented using this feature.

\subsection{Definitions}

A policy module is an application-supplied function that takes as input an
ADU $a_i = \{ i, n_i, t_i, d_i, \bar{c}_i, s_i, v_i \}$ and produces a new
ADU $a'_i$ with a possibly modified value $v'_i$. Policy modules run at the
base station, can maintain internal state, and operate with global knowledge
of the ADUs stored in the network.

A series of policy modules $\{m_1, m_2, ... m_n\}$ are composed into a linear
chain, which is passed the ADU information extracted from the ADU summary
messages received at the base station. The first policy module in the chain
is responsible for assigning the initial value $v_i$ to the ADU based on the
summary information $s_i$ calculated by the sensor node. The final ADU value
produced by the chain is used as input to Lance's optimizer for the purpose
of scheduling ADUs for download.

Lance requires that policy modules be efficient in that they can process the
stream of ADU summaries received from the network in real time. In practice
this is not difficult to accomplish, as the rate of ADU summary reception is
modest, and the base station (typically a PC or laptop) is assumed to have
adequate resources. For example, a 100-node network with an ADU size of
60~s would receive an ADU summary every 600~ms. Typical policy modules
take a small fraction of this time to run.

One of the main benefits of policy modules is that they permit significant
changes to the network's behavior \textit{without requiring changes to the
node-level summarization function}. Changing the latter would typically
involve reprogramming sensor nodes. In the field, it is often undesirable to
reprogram the network except when absolutely necessary, and in many cases it
is difficult to reach sensor nodes physically once deployed. Although systems
such as Deluge~\cite{deluge} permit over-the-air reprogramming, any changes
to the sensor node software could result in unexpected failures that can be
very difficult to debug without manual intervention. On the other hand,
introducing new policy modules at the base station is relatively
straightforward and can be quickly reversed without risking sensor node
failures.

\subsection{Example policy modules}

\begin{table}[t]
\label{lance-fig-policymodules}
\begin{center}
\begin{tabular}{|l|l|} \hline
\textit{Policy module} & \textit{Description} \\ \hline
\texttt{filter} & Set ADUs below threshold to zero value \\
\texttt{boost} & Set ADUs above threshold to max value \\
\texttt{timespread} & Dilate ADU values across time \\
\texttt{spacespread} & Dilate ADU values across space \\
\texttt{adjust} & Add or subtract offset to ADU value \\
\texttt{smooth} & Apply low-pass filter to remove noise \\
\texttt{debias} & Median filter DC debiasing \\
\texttt{correlated} & Boost values for correlated events \\
\texttt{costfilter} & Filter ADUs above cost threshold \\
\texttt{valuefilter} & Filter ADUs below value threshold \\ \hline
\end{tabular}
\end{center}

\caption{\textbf{Standard policy modules provided by Lance.}}

\label{lance-sec-example-policies}
\end{table}

Policy modules can be used to encapsulate a wide range of data collection
goals, and make it easy to customize Lance's behavior for specific
applications. We provide a standard toolkit of general-purpose policy
modules, summarized in Figure~\ref{lance-fig-policymodules}. Application
developers are free to implement their own modules as well. By composing
modules in a linear chain, it is easy to implement various behaviors without
requiring a general-purpose ``policy language.''

\begin{itemize}

\item \textbf{Value thresholding:} \texttt{filter} is perhaps the simplest
example of a policy module that filters out ADUs with a value below a given
threshold $T$ by setting their values to zero. Setting $v'_i = 0$ prevents
an ADU from being considered for download by the optimizer. This type of
filtering can be used to force a drop of low-valued data. Conversely, the
\texttt{boost} policy module sets the value for an ADU above a given
threshold to the maximum value, ensuring it will be downloaded as soon as it
is feasible.

\item \textbf{Value adjustment and noise removal:} Policy modules can be used
to remove the effects of noise or correct for node-level value bias, for
example, based on poor sensor calibration or differences in site response.
Moreover, since each node computes the ADU summary based only on local sensor
data, it may be necessary to normalize the ADU values in order to compare
values across nodes.

\hspace{0.25in} \texttt{adjust} adds or subtracts a node-specific offset to
each ADU value to correct for differences in sensor
calibration.\texttt{smooth} applies a simple low-pass filter on the raw ADU
values to remove spikes caused by spurious sensor noise. Likewise,
\texttt{debias} is intended to remove sensor-specific DC~bias from the ADU
values. \texttt{debias} computes the median ADU value for a given node over a
time window. It then subtracts the median from each ADU value before passing
it along to the next module in the chain.

\hspace{0.25in} Likewise, when a sensor network contains multiple sensors
with varying sensitivity, it is natural to prioritize data from more
sensitive instruments. In cases where networks are deployed to monitor fixed
physical phenomena, it may be desirable to prioritize data from nodes located
close to the phenomena being observed. The \texttt{adjust} module can be used
to scale raw ADU values based on a sensor's location, SNR, or other
attributes.

\item \textbf{Value dilation:} Another useful policy is to dilate a high (or
low) ADU value observed in one ADU across different ADUs sampled at different
times or different nodes. This can be used to achieve greater spatial or
temporal coverage of an interesting signal observed at one or more nodes. The
\texttt{timespread} detects ADUs with a value above some threshold $T$ and
assigns the same value to those ADUs sampled just before and just after.

\hspace{0.25in}
Likewise, the \texttt{spacespread} module groups ADUs from across multiple
nodes into time windows and assigns the maximum ADU value to all ADUs in that
window. Define a window $W(t,\delta)$ as the set of ADUs such that $t-\delta
\leq t_i \leq t+\delta$ where $t$ represents the center of the window and
$\delta$ the window size. \texttt{spacespread} determines the maximum ADU in
the window $v^* = \arg_{i \in W} \max v_i$ and sets $v'_i = v^{*}$ for each
ADU in $W$.

\item \textbf{Correlated event detection:} The \texttt{correlated} module is
used to select for ADUs that appear to represent a correlated event observed
across the entire sensor network. \texttt{correlated} counts the number of
ADUs within a time window $W(t,\delta)$ with a nonzero ADU value. If at least
$k$ ADUs meet this criterion, we assume that there is a correlated stimulus,
and the values for all ADUs in the set are passed through. Otherwise, we
filter out the ADUs in the window by setting $v'_i = 0$ for each ADU in $W$.

\hspace{0.25in} As an example of composing policy modules to implement an
interesting behavior, consider the chain \[
\mathit{filter}(T)\rightarrow\mathit{correlated}(k)\rightarrow\mathit{spacespread}
\] This policy filters incoming values, rejects time-correlated sets with
fewer than $k$ ADUs above the threshold, and assigns the max value across the
set to all ADUs. This can be useful in systems that wish to perform
collection of time-correlated data, but avoid spurious high-value data from
just a few nodes. This policy is similar to the volcanic earthquake detector
discussed in Section~\ref{evaluation-sec-eventdetection}, expressed as a
simple policy module chain.

\item \textbf{Cost-based filtering:} Lance's optimizer considers both the
cost of ADUs as well as their application-assigned values when making
download scheduling decisions. The cost vector $\bar{c}_i$ is also available
to the policy module chain, allowing policy modules to perform their own
adjustments to the ADU value according to cost, permitting applications to
augment Lance's own policies for energy scheduling. For example, the
\textit{costfilter} policy module filters out ADUs with a total energy cost
$\sum_j c_i^j$ greater than some threshold; this ensures that the network
avoids expending an arbitrary amount of energy to download a given ADU
(regardless of its data value). Policy modules give applications a great deal
of control over energy usage to complement Lance's own energy scheduling
policy.

\item \textbf{Value-based filtering:} Lance's policy modules can store and
utilize history when modifying ADU values. This can be useful when \textit{a
priori} information about the expected ADU value distribution is known. For
the volcano monitoring example, traces of activity at the volcano of interest
may be available before deployment. These may be used to produce a
distribution of ADU values based on the activity levels seen during that time
period. Lance can use this distribution to decide how interesting a
particular ADU is in the context of a longer period of activity.

\hspace{0.25in} This can be particularly helpful as a way of bootstrapping
the system. Without seeding it in this way, if Lance begins running while the
volcano is quiet it will greedily begin downloading the highest-value ADUs
available despite the fact that these do not in fact contain interesting
data, wasting energy that could be saved and used later. Instead, it could
use the \textit{valuefilter} policy module with a filter threshold chosen
based on the expected ADU distribution, perhaps chosen to ADUs with values in
the top percentile. The filter threshold can also be adjusted online based on
the vaue distribution as ADUs are sampled.

\end{itemize}

\subsection{Interaction with ADU Summary Delivery}

The policy module chain is invoked each time a new summarization message
arrives at the base station. Once the application has assigned an initial
value, the ADU is passed to the policy module chain for processing. Note that
ADU inputs may stall along the policy module chain, or be grouped by policy
modules, so that inserting a new ADU into the chain may produce that ADU as
output, may not produce an output, or may produce a different ADU or set of
ADUs as output.

Note that policies modules like \texttt{correlated} which rely on information
from multiple nodes have to cope with delayed or missing information
resulting from dropped summarization messages. These policy modules will
usually hold sets of ADUs while awaiting members that have not yet arrived,
and then release a whole group at once after they have heard from a complete
set of nodes or a timeout occurs. When data is missing or delayed, we expect
different policy modules to respond differently. Some may choose to compute
their functions over incomplete sets of ADUs, others may not and either set
the values to zero (inhibiting download) or leave them unchanged when they
cannot gather a complete set to operate over.

Depending on the sensitivity of the policy modules in use to missing data,
different summarization message formats can be used. For our experiments we
send summarization messages each time an ADU is sampled, but which contain
the last $k$ ADU summaries (up to the maximum packet size). This ensures
that, if one is dropped, the next will contain the missing information. Other
applications may want to reduce the summarization overhead further by not
sending summarization messages each time an ADU is produced, with resulting
higher delays in policy modules like \texttt{correlated}.
