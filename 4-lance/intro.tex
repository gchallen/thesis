\chapter{Addressing Data Quality with Lance}
\label{chapter-lance}

One of the core challenges emerging from our 2005 deployment at Reventador
was the inability of the network to reliably collect all sampled data. Given
the relatively modest size of our deployed network (16~nodes), we anticipated
this problem impeding future efforts to deploy large scientific macroscopes
at scale.

Given that bandwidth and energy constraints prevented us from acquiring
continuous signals from all nodes, our 2005 volcano monitoring system strove
to identify and acquire the most ``interesting'' signals --- such as
earthquakes and explosions --- and thereby avoid wasting resources on
``uninteresting'' signals. In many cases, including our own application,
these kind of resource-management decisions are made on an \textit{ad hoc}
basis or embedded in application-specific logic, often resulting in
suboptimal solutions that consume excessive bandwidth or lose data. We
believed that an architectural solution to coordinated resource management
that optimizes the application-specific value of the data acquired could both
improve the performance of our volcano-monitoring macroscope while also
benefiting other high data-rate applications.

Optimizing reliable data acquisition requires a coordinated approach to
managing both limited energy capacity and severely constrained radio
bandwidth. Depending on the sampling rate and resolution, downloading signals
may take longer than real time: while low-power sensor node radios obtain
single-hop throughput of about 100~kBps, the the best reliable protocols
achieve less than 8~kBps for a single transfer over multiple
hops~\cite{flush-sensys07}. Likewise, to achieve long lifetimes in the field,
the energy cost of downloading a signal from the network must be carefully
considered. The fundamental challenge is how to best direct these limited
network resources to acquire the most valuable data to the application.

This chapter describes \textit{Lance}, a general approach to bandwidth and
energy management for reliable signal collection in wireless sensor networks.
In Lance, each node acquires data at potentially high rates. For each
application data unit (ADU), each node generates a concise ADU
\textit{summary}, which is periodically sent to the base station and used to
compute a ADU \textit{value}. Lance computes an ADU \textit{download
schedule} based on these values, and uses a reliable transfer protocol to
download ADUs according to this schedule.

Energy usage and battery lifetime are major concerns for long-term sensor
network deployments. Lance incorporates a \textit{cost estimator} that
predicts the energy cost for reliably downloading each ADU from the network.
We describe a novel energy cost estimation algorithm that uses information on
the network topology to determine the energy cost at the sensor node hosting
the ADU as well as intermediate nodes impacted by the multihop transfer
protocol. Information from the cost estimator is used to adjust the download
schedule for ADUs, allowing Lance to target a battery lifetime for the
network by load-balancing download operations in a manner that adheres to an
energy schedule.

Lance incorporates a general framework for managing bandwidth and energy that
decouples the mechanism for prioritizing resource allocation from the
application-specific policies used to assign ADU values. This is accomplished
through user-supplied \textit{policy modules} that permit a range of
sophisticated prioritization policies to be tailored for specific
applications. Policy modules allow Lance to target a broad range of
optimization metrics, including node-local and network-wide value
maximization, lifetime targeting, and acquiring temporally- or
spatially-correlated data from across the network. Policy modules allow the
network's behavior to be significantly altered at the base station, without
reprogramming the sensor nodes themselves.

This chapter is structured as follows:

\begin{enumerate}

\item Section~\ref{lance-sec-architecture} presents the Lance architecture in
detail, which is the first system to provide a value-driven bandwidth and
energy management framework for high-data-rate sensor networks.

\item In Section~\ref{lance-sec-policies} we describe Lance's policy modules,
which offer a clean separation of policy and mechanism that allows the system
to be tailored to a broad range of applications.

\item Section~\ref{lance-sec-adaptation} discusses the process of
adapting our volcano-monitoring system to the Lance architecture and also
point out other applications that are a good fit for this approach.
Throughout the rest of the chapter we focus on the volcano-monitoring example
in detail.

\item Section~\ref{lance-sec-evaluation} shows through detailed simulation
measurements that Lance achieves \textit{near optimal} efficiency (greater
than 96\% in most cases) under a range of data distributions and resource
limitations.

\item In Section~\ref{lance-sec-deployment} we present results from an
eight-node field deployment of Lance at Tungurahua volcano, Ecuador,
demonstrating the system's performance in a real field setting and the
flexibility of policy modules for altering the network's operation following
deployment.

\item Finally, Section~\ref{lance-sec-related} discusses relevant related
work.

\end{enumerate}
