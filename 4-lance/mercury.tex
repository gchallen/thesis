\section{Application to Wearable Sensors for Activity Monitoring}
\ref{lance-sec-mercury}

Another application domain that we are exploring is motion analysis of
patients with movement disorders, such as Parkinson's
Disease~\cite{parkinsons-embs07}. In this context, up to ten sensor nodes
equipped with triaxial accelerometers and gyroscopes are placed on the
patient's limb segments (two each on the arms and legs plus one each on the
torso and waist), collecting high-resolution data at rates up to 100~Hz or
more. The goal is to capture data from the body sensor network during periods
of dyskinesia (abnormal movements) or bradykinesia (slowness of movement)
associated with the disease. The base station will typically be a laptop
located in the home, and as such will experience a wide variation in
bandwidth to the body sensor network (including disconnections), depending on
the patient's location.

Use of low-power wireless sensors keeps the size and weight of each device
down: for example, the wearable sensor node described
in~\cite{parkinsons-embs07} measures $44 \times 20 \times 13$~mm and weighs
just 10~g. While the sensor network is not spatially distributed, and all
nodes are within a single radio hop of each other, the data rates greatly
exceed the radio channel bandwidth: a single node will consume more than a
quarter of the best-case radio capacity, assuming no protocol overhead or
retransmissions.

Adapting the Lance system to support this medical monitoring application
resulted in a system called Mercury~\cite{mercury-sensys09}. Mercury makes
use of Lance to drive the energy and bandwidth management. Each sensor node
computes a series of high-level \textit{features} from the raw sensor data,
such as peak amplitude, maximum entropy, and RMS. The node prioritization
function assigns higher priority to features appearing to represent abnormal
movement. The raw signal is also stored as separate ADUs with lower priority
than the features, allowing Lance to restrict downloads of the raw data to
periods with a strong radio link to the sensors. During periods of
disconnection, nodes will buffer ADUs for later transmission; the wearable
sensors we are using support a large (up to 2~GByte) flash memory for this
purpose. Policy modules at the base station estimate the available bandwidth
to the body sensors, based on radio link quality, and prioritize downloads
accordingly.

The process of using Lance to drive the Mercury application resulted in
several changes to the architecture. For the volcano application, the node
summaries which are sent to the base station are not considered valuable in
and of themselves. They are only used to see the application-specific value
assignment process. In Mercury, the nodes are computing sets of features
which both have clinical value in and of themselves and are used to assign a
value to the underlying raw data. In fact, the first goal of the Mercury
application is to download sets of features from each node spanning all time
intervals. Any spare energy left over after this primary goal is accomplished
can be used to download the raw samples themselves.

Another change was to the policy module architecture itself. When developing
the Mercury application we found that the linear structure of the Lance
policy module chain made certain tasks difficult. In addition, the only
modification that the original policy modules were able to make was to the
assigned value, which made coordinating between policy modules at different
levels of the stack difficult.

In response to these challenges we loosened several of the restrictions on
Lance policy modules. First, rather than being assembled in a linear chain,
we provide several different hooks where policy module processing can be
triggered. This helps separate policy modules that can be composed with ones
that may need to run either first or last to operate properly. Second, we now
allow policy modules to annotate the ADU data structures with additional
metadata as they are passed along. This allows communication between policy
modules, and permits certain complex policy modules to be split into several
simpler pieces.
