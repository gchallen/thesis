\section{Application to Wearable Sensors for Activity Monitoring}
\label{lance-sec-mercury}

Another application domain that we are exploring is motion analysis of
patients with movement disorders, such as Parkinson's
Disease~\cite{parkinsons-embs07}. In this context, up to ten sensor nodes
equipped with triaxial accelerometers and gyroscopes are placed on the
patient's limb segments (two each on the arms and legs plus one each on the
torso and waist), collecting high-resolution data at rates up to 100~Hz or
more. The goal is to capture data from the body sensor network during periods
of dyskinesia (abnormal movements) or bradykinesia (slowness of movement)
associated with the disease. The base station will typically be a laptop
located in the home, and as such will experience a wide variation in
bandwidth to the body sensor network (including disconnections), depending on
the patient's location.

Use of low-power wireless sensors keeps the size and weight of each device
down: for example, the SHIMMER wearable sensor node~\cite{parkinsons-embs07}
measures $44 \times 20 \times 13$~mm and weighs just 10~g. While the sensor
network is not spatially distributed, and all nodes are within a single radio
hop of each other, the data rates greatly exceed the radio channel bandwidth:
a single node will consume more than a quarter of the best-case radio
capacity, assuming no protocol overhead or retransmissions.

Adapting the Lance system to support this medical monitoring application
resulted in a system called Mercury~\cite{mercury-sensys09}. Mercury makes
use of Lance to drive the energy and bandwidth management. Each sensor node
computes a series of high-level \textit{features} from the raw sensor data,
such as peak amplitude, maximum entropy, and RMS. The node prioritization
function assigns higher priority to features appearing to represent abnormal
movement. The raw signal is also stored as separate ADUs with lower priority
than the features, allowing Lance to restrict downloads of the raw data to
periods with a strong radio link to the sensors. During periods of
disconnection, nodes will buffer ADUs for later transmission; the wearable
sensors we are using support a large (up to 2~GB) flash memory for this
purpose. Policy modules at the base station estimate the available bandwidth
to the body sensors, based on radio link quality, and prioritize downloads
accordingly.

The process of using Lance to drive the Mercury application resulted in
several changes to the architecture. For the volcano application, the node
summaries which are sent to the base station are not considered valuable in
and of themselves. They are only used to see the application-specific value
assignment process. In Mercury, the nodes are computing sets of features
which both have clinical value in and of themselves and are used to assign a
value to the underlying raw data. In fact, the first goal of the Mercury
application is to download sets of features from each node spanning all time
intervals. Any spare energy left over after this primary goal is accomplished
can be used to download raw samples.

Another change was to the policy module architecture itself. When developing
the Mercury application we found that the linear structure of the Lance
policy module chain made certain tasks difficult. In addition, the only
modification that the original policy modules were able to make was to the
assigned value, which made coordinating between policy modules at different
levels of the stack difficult.

In response to these challenges we loosened several of the restrictions on
Lance policy modules. First, rather than being assembled in a linear chain,
we provide several different hooks where policy module processing can be
triggered. This helps separate policy modules that can be composed with ones
that may need to run either first or last to operate properly. Second, we now
allow policy modules to annotate the ADU data structures with additional
metadata as they are passed along. This allows communication between policy
modules, and permits certain complex policy modules to be split into several
simpler pieces.

In order to facilitate more functionality within policy module we also
increased the amount of control these functions can exercise over node-level
behaviors. Rather than being limited to simply affecting the download
distribution, Lance policy modules can now call a number of functions
exported by the node-level software allowing them to, as an example, start
and stop sampling, change sampling rates, enable or disable certain sensors,
and enable or disable Flash logging.

This functionality substantially extends the options that policy modules have
for controlling the energy consumption at each node. The original version of
Lance that focused on bulk data-transfer could only affect the energy
consumption at a node by choosing to use it to download or forward data, or
by avoiding it. When nodes are running low on energy, the updated version can
reduce their sampling rates (which conserves energy), disable logging of
certain kinds of data (further conserving energy), or cease sampling and
logging all-together, completely eliminating the sampling and storage
components of node energy consumption. These additional energy-saving
features are particularly important for the Mercury application because some
of the sensors deployed for medical monitoring --- such as gyroscopes ---
have energy consumptions that rival the radio, and their operation can thus
have a significant impact on the node's ability to meet a lifetime schedule.

Implementing these new control features allows a range of policy modules to
be written. The overall goal of the Mercury system is to enable continuous
data collection for 12 to 18 hours each day, which approximates the interval
between charges when the patient would be wearing the device. Depending on
the lifetime being targeted and the initial battery charge, the Mercury
driver computes an \textit{energy schedule} for each node and tries to ensure
that the nodes discharge rate matches the schedule. If the node is consuming
too much or two little energy, it will adjust the node's behavior to trade
off application fidelity for lifetime. In~\cite{mercury-sensys09}, we
implement and experiment with the following policies:

\begin{enumerate}

\item \textbf{throttle downloads}: since Mercury's first goal is to download
features for each time unit, rather than raw samples, this policy only begins
downloading raw signals if the node has spare energy in its energy budget.
The reduction in radio traffic that accompanies not downloading large amounts
of raw data can extend node lifetimes.

\item \textbf{throttle gyro}: because the gyroscope consumes a great deal of
energy, if a node is behind its energy schedule we can disable the gyroscope
to allow it to catch up. This reduces the fidelity of the data while
substantially increasing the node's lifetime.

\item \textbf{throttle storage}: since another source of energy consumption
is using the Flash memory~\footnote{The SHIMMER~\cite{shimmer} node used for
the Mercury project uses an external Flash card that has a significantly
higher energy cost to read and write than the internal Flash chip used by the
TMote Sky for our volcano deployments. The SHIMMER also carries significantly
more Flash memory: up to 2~GB as opposed to the 1~MB on the TMote Sky.}, we
may want to disable saving raw signals to Flash as another way of saving
power. Again, this choice has an impact on the resulting data fidelity, since
the discarded data cannot be downloaded over the radio at a later time, or
retrieved when the node is cradled to charge.

\item \textbf{throttle sampling}: this is obviously the most drastic option,
since it completely disables all data collection on the node.

\end{enumerate}

The above policies can also be implemented in various permutations with each
other to create, for example, a strategy that may disable both downloads and
the gyroscope depending on how far the node is behind on its energy schedule.

We found that the \textbf{throttle gyro} policy was effective at trading off
feature coverage and lifetime across a range of lifetime targets. If we
define \textit{full feature coverage} as the range of time spanned by
features computed from all sensors (including the gyroscope) and
\textit{degraded feature coverage} as the range of time spanned by features
computed from all sensors but the gyroscope, for a 12~hour lifetime target
\textbf{throttle gyro} enables \~100\% full feature coverage and for a
60~hour lifetime target it enables \~100\% degraded feature coverage. It
accomplishes this while still downloading a portion of the raw data for the
experiment. \textbf{throttle storage} enables full feature coverage across
the range of lifetime targets but does so by severely reducing the number of
raw signals collected.

In general we found the flexibility and power of the Lance architecture
useful for quickly implementing and testing various policies for managing
resource usage. The interface that we built to allow policy module to control
node behavior limited the amount of reprogramming of the nodes themselves
that we had to do in order to enact dramatic changes in node behavior.
Because Mercury targets a network that is usually single-hop, we did not need
to use the heuristic scoring functions described previously. Since we
discovered that these were probably not equally suited to all applications,
we altered the structure of the system to have them implemented as policy
modules, available for applications to use when needed but not required as
part of the Lance architecture itself. Overall we were excited by the success
we experienced using Lance is this very different application context.
