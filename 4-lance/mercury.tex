\section{Using Lance for Mercury}

\XXXnote{GWA: TODO: Rewrite this including more about what we accomplished
with Mercury.}

Another application domain that we are exploring is motion analysis of
patients with movement disorders, such as Parkinson's
Disease~\cite{parkinsons-embs07}. In this context, up to ten sensor nodes
equipped with triaxial accelerometers and gyroscopes are placed on the
patient's limb segments (two each on the arms and legs plus one each on the
torso and waist), collecting high-resolution data at rates up to 100~Hz or
more. The goal is to capture data from the body sensor network during periods
of dyskinesia (abnormal movements) or bradykinesia (slowness of movement)
associated with the disease. The base station will typically be a laptop
located in the home, and as such will experience a wide variation in
bandwidth to the body sensor network (including disconnections), depending on
the patient's location.

Use of low-power wireless sensors keeps the size and weight of each device
down: for example, the wearable sensor node described
in~\cite{parkinsons-embs07} measures $44 \times 20 \times 13$~mm and weighs
just 10~g. While the sensor network is not spatially distributed, and all
nodes are within a single radio hop of each other, the data rates greatly
exceed the radio channel bandwidth: a single node will consume more than a
quarter of the best-case radio capacity, assuming no protocol overhead or
retransmissions.

Following our deployment at Tungurahua Volcano in 2008 we adapted the Lance
system to support this medical monitoring application, resulting in a system
called Mercury~\cite{mercury-sensys09}. Mercury makes use of Lance to drive
the energy and bandwidth management. Each sensor node computes a series of
high-level \textit{features} from the raw sensor data, such as peak
amplitude, maximum entropy, and RMS. The node prioritization function assigns
higher priority to features appearing to represent abnormal movement. The raw
signal is also stored as separate ADUs with lower priority than the features,
allowing Lance to restrict downloads of the raw data to periods with a strong
radio link to the sensors. During periods of disconnection, nodes will buffer
ADUs for later transmission; the wearable sensors we are using support a
large (up to 2~GByte) flash memory for this purpose. Policy modules at the
base station estimate the available bandwidth to the body sensors, based on
radio link quality, and prioritize downloads accordingly.

