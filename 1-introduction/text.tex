\chapter{Introduction: The Macroscope} 
\label{chap-introduction}

\XXXnote{GWA: Text pulled from OSDI'06 introduction.}

Sensor networks are making inroads into a number of scientific explorations,
including environmental monitoring~\cite{rope-emnets05,berkeley-redwoods},
habitat monitoring~\cite{cerpa-habitat,mainwaring-habitat,gdi-sensys04}, and
structural monitoring~\cite{ggb-monitoring,netshm-emnets05,wisan}.  In each
of these domains, the use of low-power wireless sensors offers the potential
to collect data at spatial and temporal scales that are not feasible with
existing instrumentation.  Despite increased interest in this area, little
has been done to evaluate the ability of sensor networks to provide
meaningful data to domain scientists. A number of challenges confound such an
effort, including node failure, message loss, sensor calibration, and
inaccurate time synchronization. To successfully aid scientific studies,
sensor networks must be held to the high standards of current scientific
instrumentation.

\section{Overview of Three Deployments}

In total, we have performed three deployments of iterations of our system at
active volcanoes in Ecuador. All three are summarized below:

\begin{enumerate}

\item \textbf{July, 2004, Tungurahua Volcano:} We deployed three infrasonic
monitoring nodes continuously transmitting at 102~Hz to a central aggregator
node, which relayed the data over a wireless link to the observatory
approximately 9~km away.  Our network was active from July 20--22, 2004, and
collected over 54~hours of infrasonic signals.

\item \textbf{August, 2005, Reventador Volcano:} This deployment featured a larger,
more capable network consisting of sixteen nodes fitted with seismoacoustic
sensors deployed in a 3~km linear array.  Collected data was routed over a
multi-hop network and over a long-distance radio link to a logging laptop
located at the observatory 9~km away from deployment site.  Over three weeks
the network captured 230 volcanic events.

\item \textbf{July, 2007, Tungurahua Volcano:} We returned to Tungurahua Volcano in
2007 and deployed eight sensor nodes in order to test Lance, a framework for
optimizing high-resolution signal collection. The network was operational for
a total of 71~hours, during which time we downloaded 77~MB of raw data.

\end{enumerate}


\section{Background: Wireless Sensor Networks}

\section{Architectural Challenges for Scientific Sensor Networks}

\section{Dissertation Summary and Contributions}

\section{Dissertation Roadmap}

The remainder of this dissertation is organized as follows.

