\chapter{Introduction: The Macroscope} 
\label{chap-introduction}

In his book \textit{The Macroscope}, French futurist Joel de Rosnay
envisioned a new scientific instrument: the \textit{macroscope}. As a
large-scale analog to the microscope, the macroscope would be able to provide
insights into large systems that are difficult to study at scale: entire
cities, ecosystems, or planetary-scale systems.

This dissertation presents the design of a scientific macroscope using
wireless sensor networks as an enabling technology, as well as two
architectural solutions designed to improve the sensor macroscope's
performance. Sensor network nodes have many limitations, and using them to
construct a scientific instrument presents multiple challenges. In this
dissertation, we first validate that data collected by wireless sensor nodes
can meet the needs of a demanding, high data-rate scientific application:
volcano monitoring. We then describe new sensor network architectures that
improve the data quality provided to the application by the sensor
macroscope. Together these techniques begin the process of realizing de
Rosnay's vision using embedded sensing.

\section{Architectural Challenges for Scientific Sensor Networks}

Sensor networks are making inroads into a number of scientific explorations,
including environmental monitoring~\cite{rope-emnets05,berkeley-redwoods},
habitat monitoring~\cite{cerpa-habitat,mainwaring-habitat,gdi-sensys04}, and
structural monitoring~\cite{ggb-monitoring,netshm-emnets05,wisan}. In each of
these domains, the use of low-power wireless sensors offers the potential to
collect data at spatial and temporal scales that are not feasible with
existing instrumentation. A number of challenges confound such an effort,
including node failure, message loss, sensor calibration, and inaccurate time
synchronization. To successfully aid scientific studies, sensor networks must
be held to the high standards of current scientific instrumentation.


\section{Dissertation Summary and Contributions}

This dissertation makes the following contributions:

\begin{itemize}

\item \textbf{Design of a volcano-monitoring sensor network.}
We have built and fielded the first sensor network designed to study active
volcanos. After an initial pilot deployment, a system was designed giving
careful consideration to the scientific requirements.

\item \textbf{Iterative deployment on active Ecuadorean volcanos.}

In total, we have performed three deployments of iterations of our system at
active volcanoes in Ecuador. All three are summarized below:

\begin{enumerate}

\item \textbf{July, 2004, Tungurahua Volcano:} We deployed three infrasonic
monitoring nodes continuously transmitting at 102~Hz to a central aggregator
node, which relayed the data over a wireless link to the observatory
approximately 9~km away.  Our network was active from July 20--22, 2004, and
collected over 54~hours of infrasonic signals.

\item \textbf{August, 2005, Reventador Volcano:} This deployment featured a larger,
more capable network consisting of sixteen nodes fitted with seismoacoustic
sensors deployed in a 3~km linear array.  Collected data was routed over a
multi-hop network and over a long-distance radio link to a logging laptop
located at the observatory 9~km away from deployment site.  Over three weeks
the network captured 230 volcanic events.

\item \textbf{July, 2007, Tungurahua Volcano:} We returned to Tungurahua Volcano in
2007 and deployed eight sensor nodes in order to test Lance, a framework for
optimizing high-resolution signal collection. The network was operational for
a total of 71~hours, during which time we downloaded 77~MB of raw data.

\end{enumerate}

\item \textbf{Detailed evaluation of a scientific macroscope.} Following our
2005 deployment we took a hard look at the performance of our system from the
perspective of our seismology collaborators. Rather than dwelling on metrics
of interest to computer scientists, we attempted to address the core aspects
of the system that would help drive scientific adoption. We identified two
core concerns: data \textit{fidelity}, encompassing the quality and accuracy
of the collected data; and \textit{yield}, measuring the amount of data that
the system could successfully retrieve.

We conducted a rigorous examination of our 2005 deployment along these lines.
A unique challenge arose when attempting to assign timing information to our
data to allow it to be used for scientific analysis, and this led to the
development of a novel time rectification approach. This new technique was
able to correct timing protocol failures during our field deployment and
allow us to accurately assign timestamps to almost all of the data our
network collected.

\item \textbf{Architecture optimizing data quality.} Given limited batteries
and low-bandwidth links, sensor networks must carefully manage resource
usage, particulary with an eye to how these decisions impact the quality of
the data provided to the application. Many sensor networks attempt to balance
the cost and utility of actions taken by the network, but do so in ad-hoc
ways embedded in application-specific logic.

Lance provide an architecural solution to the problem of optimizing reliable
data collection for high data-rate sensor network applications. Lance is
based on two observations that emerged during the evaluation of our 2005
volcano deployment. First, given the low-bandwidth of sensor network radios,
even with a moderately-sized network it was not possible to reliable extract
all the data the nodes were collecting. Given the inevitability of data loss,
Lance attempts to ensure that the subset of data that was downloaded provides
the highest value to the application.

Second, given the high power consumption of sensor network radios, the energy
consumption associated with reliable data collection was the dominant source
of discretionary energy usage in our deployed network. Thus, attempting to
ensure that all nodes met a target lifetime required consideration of the
distributed energy impact of multihop reliable data transfer. Lance attempts
to balance the cost of downloading data against the value of that data to the
application. We were able to develop a simple heuristic for selecting which
data to download that delivers near-optimal performance under a range of
constraints and network properties.

\item \textbf{Service for collaborative distributed energy management.} While
Lance demonstrated the improvements in performance achievable by adjusting
network behavior to meet energy availability, it was limited in several
respects. First, Lance relies on a centralized controller. Since the overhead
of centralized control scales badly as the network size grows, this limits
the applicability of this approach. Second, Lance's only way of effecting
energy usage is in the choice of what data to retrieve from the network. This
is appropriate for high data-rate networks where data download is the primary
source of energy consumption, but not for networks where this is not the
case.

IDEA (Integrated Distributed Energy Awareness) is a sensor network service
designed to provide the benefits of energy awareness to all sensor network
components in a distributed fashion. IDEA distributes information about each
node's energy load and charging rates and provides intelligence that assists
components in adjusting their own state in ways beneficial to the target
application.

\end{itemize}

\section{Dissertation Roadmap}

The remainder of this dissertation is organized as follows. The next chapter
introduces sensor networks for volcano monitoring, a high data-rate
application that we have studied in detail and which has motivated the
architectural contributions outlined in this dissertation; it also presents
related efforts aimed at building scientific sensor macroscopes using sensor
network technology.

Chapter~\ref{chapter-evaluation} presents the design and a detailed
evaluation of a sensor network system built to study active volcanos.
Chapters~\ref{chapter-lance} and \ref{chapter-idea} presents Lance and IDEA,
two architectural approaches tackling problems emerging from our field
deployments. Lance is designed to optimize the output data quality of the
macroscope given constraint on bandwidth and energy, while IDEA aims to
improve the performance of the macroscope by enabling collaborative
management of distributed energy resources within the instrument.

Finally, Chapter~\ref{chapter-lessons} presents lessons learned in the course
of building these systems and presents opportunities for future work, and
Chapter~\ref{chapter-conclusion} concludes.

