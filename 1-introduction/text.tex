\chapter{Introduction}
\label{chap-introduction}

Wireless sensor networks composed of tens or hundreds of low-power,
resource-constrained devices can aid scientific exploration by providing
high-fidelity data at scales difficult to achieve with traditional
instrumentation. In his book \textit{The Macroscope}~\cite{rosnay79}, French
futurist Joel de Rosnay envisioned a new scientific instrument. A large-scale
analog to the microscope, \textit{the macroscope} would provide insights into
systems difficult to study at scale: cities, ecosystems, or the entire
biosphere.

Realizing the scientific macroscope using sensor network technology requires
confronting several core challenges. Collected data must be able to match the
fidelity provided by existing instrumentation. When resource limitations
limit the amount of data that can be collected, the system should focus its
limited energy and bandwidth on the most important signals. To maintain good
performance, the system must gracefully adjust to fluctuations in energy and
bandwidth availability. By managing distributed resource usage, we can
improve the data quality provided by sensor networks to scientific
applications.

\clearpage

This dissertation presents the design of a scientific macroscope using
wireless sensor networks as an enabling technology, along with two
architectural solutions improving the macroscope's performance by managing
distributed resources. After validating that data collected by wireless
sensor nodes can meet the needs of a demanding, high data rate scientific
application --- volcano monitoring --- we then describe new sensor network
architectures improving the data quality provided to the application.
Together these techniques begin the process of realizing de Rosnay's vision
using embedded sensing.

\section{Wireless Sensor Networks}

Wireless sensor networks consist of nodes integrating modest amounts of
computation, storage, and communication capabilities. Low-power
microprocessors, radios, and MEMS sensors enable embedded sensing on power
budgets not feasible a decade ago.

An representative device is the TMote Sky~\cite{moteiv}. A descendant of the
original UC Berkeley Mica ``mote'' sensor node~\cite{hill-thesis}, it
includes a Texas Instruments MSP430 microcontroller, 48~kB of program memory,
10~kB of SRAM, 1~MB of external flash memory, and a 2.4~GHz Chipcon IEEE
802.15.4 radio. The MSP430 is a 16~bit microcontroller running at 4~MHz and a
popular basis for wireless sensor network nodes due to its many
reconfigurable ports and low power consumption. It draws approximately 2~mA
of current while active and can enter sleeps states consuming only microamps.

The CC2420 is a low-power 802.15.4 radio with a raw data rate of 250~kbps. In
practice the raw data rate is reduced considerably by the overheads necessary
to enable medium access control and the limitations of the SPI bus. The
CC2420 consumes roughly 20~mA of current while active but can quickly enter
and leave a low-power sleep state, which enables channel-polling and other
kinds of low-power operation.

Due to their low power budgets, sensor nodes can be lightweight and easy to
deploy. We found it possible to power a TMote Sky for several weeks using a
pair of D-cell batteries. By duty-cycling the radio, the largest consumer of
energy on the device, we could have increased the node's lifetime
considerably. If extended-duration operation is desired, nodes can be
equipped with energy-harvesting capabilities. Due to their low power
consumption, even a small solar panel can allow a TMote to run continuously,
depending on incident sunlight and the efficiency of energy capture and
storage.

\section{Sensor Networks for Science: Architectural Challenges}

Wireless sensor nodes can improve on existing instrumentation in several
ways. In certain cases, low-power embedded nodes can replicate functionality
now performed by larger, heavier devices, reducing the amount of equipment
that must be transported and accelerating the deployment process. In
addition, sensor networks come with telemetry ``built in'', which may also
provide a significant benefit when compared with existing untelemetered
instrumentation. Continuous communication with deployed sensors during the
experiment allows scientists to begin to process data, assess the health of
the network, or make adjustments on the fly. In other cases, embedded sensing
establishes a different set of tradeoffs between the ease of deployment and
the sensing capability of each device. For the same amount of deployment
effort and cost, a researcher can deploy either a small set of extremely
accurate sensors or a much larger array of less accurate sensors. It is
likely that one choice will provide the best performance for a particular
application.

Early sensor network researchers saw the potential for these new devices to
aid in scientific studies. One of the first published deployments of embedded
sensing technology was on Great Duck Island in Maine, where a network of 32
sensor nodes was used to study the nesting behavior of a colony of
Seabirds~\cite{gdi-sensys04}. Another early experiment used similar nodes to
study the microclimate of a single Redwood tree~\cite{berkeley-redwoods}. In
contrast to these early efforts, our group has focused on \textit{high data
rate} scientific applications. As a point of comparison, the Redwoods
deployment collected approximately 0.05 bytes-per-second (bps) from 30~nodes,
for an aggregate network data rate of 1.5 bps. Contrast this with our 2005
volcano monitoring deployment (described in Chapter~\ref{chapter-evaluation})
where each of 16~nodes collected 600~bps of data, for a network data rate of
9600~bps. Compared with low data rate monitoring, high network data rates
produce a distinct set of research challenges:

\begin{itemize}

\item \textbf{Data fidelity.} High data rates stress the ability of the
system to provide high fidelity data. Sampling at high rates introduces
additional load on each node, which can interfere with its participation in
routing, time synchronization, or data transfer protocols. Frequently, timing
accuracy requirements scale with the sampling rate, so that data sampled at
high rates must be assigned precise timestamps to facilitate comparisons
between multiple stations.

\textit{Problem Statement:} Given a target application, build and validate a
system providing data-fidelity sufficient to meet the scientific goals.
Addressed in Chapter~\ref{chapter-evaluation}.

\item \textbf{Optimizing overall data quality.} In addition, high sampling
rates challenge the ability of the system to achieve low-power operation.
Sampling, storing, and transmitting large amounts of data prevents nodes from
entering into low-power states and can consume a great deal of their stored
energy. Once data rates and network sizes pass a certain point, all data
cannot be collected in real time, meaning that the system must choose which
data to discard and which to retrieve. Depending on the phenomena being
monitored, some data is likely to be interesting and some less interesting. A
habitat monitoring application may want to focus on collecting vocalizations
from a certain kind of animal. Differences in sensor quality may also lead to
data value disparities, when poorly-located or poorly-functioning sensors
sample poor-quality of little value to the application.

\textit{Problem Statement:} When all data cannot be collected due to resource
limitations, retrieve the data most valuable to the application. Addressed in
Chapter~\ref{chapter-lance}.

\item \textbf{Network-wide energy disparities.} When energy-harvesting
technologies are deployed, variances in energy availability across the
network challenge the health of the entire system. Many sensor network
protocols concentrate energy usage on a small set of nodes, leading to
degraded operation when those nodes' batteries are exhausted. Enabling good
performance of the network as a whole requires continuously adapting to
variations in resource availability and adjusting network behavior in ways
that allow nodes with large amounts of energy to take on new roles and nodes
with low batteries to reduce their responsibilities.

\textit{Problem Statement:} When disparities in energy availability emerge
due to variations in load and charging rates across the network, adjust
nodes' behavior to tune energy consumption based on the application's goals.
Addressed in Chapter~\ref{chapter-idea}.

\end{itemize}

When tackling these challenges, our approach has been inspired by a vision of
a network of sensor nodes functioning as a \textit{single instrument}. We
have tried to design architectures that abandon node-level views in favor of
network-level perspective. This challenges the way that many sensor network
protocols are designed, since they assume that greedy local decision making
will prove globally beneficial. Instead, the two architectures described in
this dissertation attempt to connect local node behavior with the resulting
value provided by the entire system to the application.

\section{Dissertation Summary and Contributions}

This dissertation makes the following contributions:

\vspace*{0.1in}

\noindent \textbf{Detailed evaluation of a scientific macroscope.} Giving
careful consideration to the scientific requirements, we have built and
fielded the first sensor network designed to study active volcanoes. We have
deployed three system iterations at active volcanoes in Ecuador:

\begin{enumerate}

\item \textbf{July, 2004, Tungurahua volcano:} We deployed three infrasonic
monitoring nodes continuously transmitting at 102~Hz to an aggregator that
relayed the data over a wireless link to an observatory approximately 9~km
away. Our network was active from July 20--22, 2004, and collected over
54~hours of infrasonic signals.

\item \textbf{August, 2005, Reventador volcano:} This deployment featured a
larger, more capable network consisting of sixteen nodes fitted with
seismoacoustic sensors deployed in a 3~km linear array. Collected data
traveled over a multi-hop network and across a long-distance radio link to a
laptop located at the observatory 9~km away. Over three weeks the network
captured 230 volcanic events.

\item \textbf{July, 2007, Tungurahua volcano:} We deployed eight sensor nodes
to test Lance, a framework for optimizing high-resolution signal collection.
The network was operational for a total of 71~hours and downloaded 77~MB of
data.

\end{enumerate}

Following our 2005 deployment we took a hard look at the performance of our
system from the perspective of our seismology collaborators. Rather than
dwelling on metrics interesting to computer scientists, we attempted to
address performance aspects critical to scientific adoption. We identified
two core concerns: data \textit{fidelity}, encompassing the quality and
accuracy of the collected data; and \textit{yield}, measuring the amount of
data that the system could retrieve.

We conducted a rigorous examination of our 2005 deployment along these lines.
A unique challenge arose when attempting to assign timing information to our
data to facilitate scientific analysis. This led to the development of a
novel time rectification approach that was able to correct the timing
protocol failures that occurred during our field deployment and accurately
assign timestamps to almost all of the data our network collected.

\vspace*{0.1in}

\noindent \textbf{Lance: an architecture optimizing data quality.} Given
limited batteries and low-bandwidth links, sensor networks must identify how
resource management decisions affect application data quality. Many sensor
networks attempt to balance the cost and utility of actions taken by the
network, but embed these decisions in application-specific logic.

Lance provides an architectural solution to the problem of optimizing
reliable data collection for high data rate sensor network applications.
Lance is based on two observations that emerged during the evaluation of our
2005 volcano deployment. First, given the low bandwidth of sensor network
radios, even a moderately-sized network makes it impossible to reliably
extract all the data sampled. Given the inevitability of data loss, Lance
attempts to ensure that the data that is downloaded provides the highest
value to the application.

Second, given the high power consumption of sensor node radios, reliable data
collection was the dominant source of discretionary energy usage in our
deployed network. Thus, attempting to ensure that all nodes met a target
lifetime required considering the distributed energy impact of multihop data
transfer. Lance balances the cost of downloading data against its value to
the application using a simple online heuristic that achieves excellent
performance: within 5\% of optimal for most cases.

Lance is a general-purpose architecture targeting high data rate
applications. Along with the volcano-monitoring application, we have also
used Lance to drive a body sensor network enabling continuous monitoring of
patients with neuromuscular disorders.

\clearpage

\noindent \textbf{IDEA: a service enabling collaborative distributed energy
management.} While Lance demonstrated the performance improvements achievable
by adjusting network behavior to meet energy availability, it was limited in
several ways. First, Lance relies on a centralized controller. Since the
overhead of centralized control scales poorly as the network grows, this
limits the applicability of this approach. Second, Lance's only way of
controlling energy usage is choosing what data to download, which is only
appropriate for networks where data download is the dominant source of energy
consumption.

IDEA (Integrated Distributed Energy Awareness) is a sensor network service
that provides the benefits of energy awareness to any sensor network
component in a distributed fashion. IDEA distributes information about each
node's energy load and charging rates in order to assist components in
adjusting their state in ways benefiting the application. By better
distributing energy consumuption, IDEA extends network lifetimes by up to
27\%.

\section{Dissertation Road Map}

The remainder of this dissertation is organized as follows.
Chapter~\ref{chapter-related} introduces sensor networks for volcano
monitoring and presents other projects that have built macroscopes using
sensor network technology. Chapter~\ref{chapter-evaluation} presents the
design and detailed evaluation of a sensor network built to study active
volcanoes. Chapters~\ref{chapter-lance} and \ref{chapter-idea} present Lance
and IDEA, two architectures tackling problems emerging from our field
deployments. Lance optimizes output data quality given constraints on
bandwidth and energy, while IDEA improves performance by enabling
collaborative energy management. Chapter~\ref{chapter-lessons} presents
lessons learned while building these systems and opportunities for future
work.
