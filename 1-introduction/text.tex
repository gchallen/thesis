\chapter{Introduction: The Macroscope} 
\label{chap-introduction}

\XXXnote{GWA: Text pulled from OSDI'06 introduction.}

Sensor networks are making inroads into a number of scientific explorations,
including environmental monitoring~\cite{rope-emnets05,berkeley-redwoods},
habitat monitoring~\cite{cerpa-habitat,mainwaring-habitat,gdi-sensys04}, and
structural monitoring~\cite{ggb-monitoring,netshm-emnets05,wisan}.  In each
of these domains, the use of low-power wireless sensors offers the potential
to collect data at spatial and temporal scales that are not feasible with
existing instrumentation.  Despite increased interest in this area, little
has been done to evaluate the ability of sensor networks to provide
meaningful data to domain scientists. A number of challenges confound such an
effort, including node failure, message loss, sensor calibration, and
inaccurate time synchronization. To successfully aid scientific studies,
sensor networks must be held to the high standards of 
%meeting or exceeding 
current scientific instrumentation.
%capabilities.

\section{Background: Wireless Sensor Networks}

\section{Architectural Challenges for Scientific Sensor Networks}

\section{Dissertation Summary and Contributions}

\section{Dissertation Roadmap}

The remainder of this dissertation is organized as follows.

