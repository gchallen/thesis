\section{Introduction}
\label{sec-introduction}

Energy-harvesting sensor networks experience variations in load and charging
rates that threaten high-fidelity operation. Changing application demands can
produce variations in load rates, while energy-harvesting properties can
produce variations in charging rates. Energy mismanagement
can lead to reduced fidelity, when nodes' batteries empty, or wasted energy,
when nodes harvest energy they cannot store.

Energy harvesting capabilities such as solar charging further complicate the
distributed energy management task. The energy collected at each node may
vary significantly based on node placement, and the energy collected
day-to-day may vary significantly based on weather patterns. Preparing the
network for overnight operation requires capturing as much energy as possible
during the day and minimizing energy wasted by charging full batteries, while
overnight operation itself requires adjusting the network's load profile to
match the distribution of energy stored during daytime.

Fortunately, dense networks provide redundancy that can be used to control
the distribution of energy usage.  Multiple possible routing paths may
connect a node to the sink. Tuning MAC parameters allows nodes to shift
communication load to their neighbors. Sensor inputs from multiple nodes may
be redundant, allowing some to be disabled or operated at reduced fidelity.
The existence of these choices implies that it is possible to tune the energy
load of the network to better match energy availability.  Effective load
tuning can increase the fidelity provided to the application at a fixed
battery size, or allow battery sizes to be reduced while maintaining the
required fidelity level.

Existing sensor network platforms provide little support for collaborative
energy management. Approaches such TinyOS~\cite{tinyos-asplos00},
Pixie~\cite{pixie-sensys08}, Eon~\cite{eon-sensys07} and
Levels~\cite{levels-sensys07} facilitate local control only, failing when
greedy node energy minimization fails to produce the best outcome.
Network-wide solutions such as Lance~\cite{lance-sensys08},
Mercury~\cite{parkinsons-embs07}, and EnviroMic~\cite{enviromic} either
require centralized control or are tailored to the needs of a specific
application domain. In sensor networks the majority of energy consumption is
consumed by multi-node collaboration. We argue that due to the distributed
nature of energy consumption and availability, improving performance
requires consideration of both \textit{where} energy is and \textit{how much}
is being used.

Matching load to availability across the network requires \emph{integrating}
with application components producing energy load, \emph{distributing}
load and availability information to facilitate node decision making, and
\emph{awareness} of the connection between load, availability, and
application-level fidelity. We propose Integrated Distributed Energy
Awareness (IDEA), a sensor network service addressing these goals. IDEA
monitors and models the load and charge rates on each node.  To allow nodes
to reason about their impact on others, each node distributes its model
parameters, updating them as necessary to ensure continued accuracy.  IDEA
clients are responsible for estimating their own distributed energy
impact. When changing state, IDEA helps them evaluate each
proposed option using an energy objective function tailored to meet
specific application goals. By tracking availability and informing the energy
decision-making process, IDEA simplifies the construction of energy-aware
components.

Our paper makes the following contributions. First, we describe IDEA, a new
service uniting energy monitoring, load modeling and distributed state
sharing into a single service facilitating distributed decision making.
Second, we present three case studies illustrating how to use IDEA, including
a component that tunes MAC parameters, an existing routing protocol modified
to choose energy-aware routes, and a application using IDEA to determine how
to localize acoustic events.  Third, using simulation and testbed results we
compare the performance of IDEA with approaches that do not consider energy
distribution, showing that IDEA enables improvements in lifetime of up to
35\%.

The rest of this paper is organized as follows. Section~\ref{sec-motivation}
motivates the need for IDEA using a simple example. In
Section~\ref{sec-architecture} we present the IDEA architecture in detail and
describe our current implementation. We describe our three case studies in
detail in Section~\ref{sec-casestudies}. Section~\ref{sec-evaluation}
presents simulation and testbed results. We review related work in
Section~\ref{sec-related}, and Section~\ref{sec-futurework} outlines future
work and concludes.
