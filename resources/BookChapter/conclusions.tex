\section{Conclusions and Future Work}
\label{sec-conclusions}

Based on our experience with data quality spanning three deployments and
multiple system components we see many directions for future work.  Every
component of datum and dataset quality mentioned here is open for future
research and continued development.

While we have made progress in pinning down the the datum quality
requirements specific to this domain, work continues on hardware-software
co-design, time synchronization and reliable data collection.  We are
currently designing a sensor interface board for the iMote2 intended to
support the volcano application as well as other high data-rate sensing
tasks, and are revisiting many of the original design decisions that produced
the volcano monitoring board described here.  Recent efforts within our group
have continued to work on reliable time synchronization in other settings
such as body sensor networks.  And we have begun a redesign of the data
collection component inspired by approaches such as Hop~\cite{hop-nsdi09}
that seek to reduce power consumption and improve performance over lossy,
low-power links.

As the Lance work brings out, there are tradeoffs possible between a deployed
networks target lifetime and the quantity and quality of the data provided to
the application. Lance takes a position on one end of a sliding spectrum in
that it holds the system lifetime constant, treating it as an input parameter
that must be met, and then tries to turn other knobs to maximize the output
data quality given other constraints. Another approach would be to hold the
output data quality constant and try to allow the system to last as long as
possible while providing data at a minimum fidelity set by the application.
Between these two endpoints there are multiple approaches which would tend to
trade off application data quality for increased system lifetime. Given the
difficulties inherent in performing this tradeoff, we have not yet explored
this interstitial area. However, this area seems quite fertile for future
work.

In addition, maximizing the output of a deployed network under resource
constraints requires continuing to build strong connections between the
notion of data quality operative within the network and the actual needs of
the application. In Lance, we do not dictate that applications use a
simplified scalar score to indicate data quality, but our architecture tends
to push applications in this direction. Many applications, however, may not
be able to reduce a complex set of considerations into a single scalar value.
Enriching this interface may help allow further dataset quality
optimizations. In general, however, the interface between the end user's
goals and the simplified metrics operating within the network must receive
further attention to ensure the broader applicability of solutions that
attempt in-network optimization.

In this chapter we have attempted to zero in on data quality issues inherent
in the development of a sensor network supporting the monitoring of active
volcanoes. Through our experience in this area spanning three deployments we
have grappled with many aspects of data quality, from where the sensor meets
the sensor node up to complicated optimization approaches running at the base
station.  We have attempted to narrate, in a helpful way, our experiences in
this area, and hope that some of the struggles we have elucidated will smooth
the waters for other sensor network deployments.
