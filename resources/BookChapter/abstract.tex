Deploying wireless sensor networks to support geophysics presents an
interesting challenge. High data-rates required by geophysical
instrumentation preclude continuous data collection from even
moderately-sized networks. However, geoscientists are used to working
directly with complete signals, and therefor uncomfortable with in-network
processing that could reduce bandwidth by reporting data products.

Over five years of working with seismologists we have developed a lineage of
solutions driven by their scientific goals. Three field deployments have
provided valuable lessons and helped drive each successive design iteration.
We began by addressing datum quality, encompassing per-sample resolution,
accuracy, and time synchronization.  Later deployments focused on holistic
data quality, which requires considering constraints limiting full data
collection in order to maximize the value of the limited data retrieved. This
chapter uses our three deployments to demonstrate the benefits of iteration.
The first two illustrate our work on datum quality, while the last presents a
new approach to optimizing overall dataset quality.
