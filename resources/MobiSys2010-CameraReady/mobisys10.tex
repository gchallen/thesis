\documentclass{sig-alternate}
\usepackage{graphicx,url,amsmath,amsfonts,amssymb,color,multirow}

\newcommand{\XXXnote}[1]{}

\pagestyle{empty}

\def\sharedaffiliation{
\end{tabular}
\begin{tabular}{c}}

\tolerance 400
\setlength{\emergencystretch}{3em}

\begin{document}

\conferenceinfo{MobiSys'10,} {June 15--18, 2010, San Francisco, California, USA.} 
\CopyrightYear{2010}
\crdata{978-1-60558-985-5/10/06}

\title{IDEA: Integrated Distributed Energy Awareness\\for Wireless Sensor Networks}
\numberofauthors{3}

\author{
\alignauthor Geoffrey Werner Challen\\
\email{challen@eecs.harvard.edu}
\alignauthor Jason Waterman\\
\email{waterman@eecs.harvard.edu}
\alignauthor Matt Welsh\\
\email{mdw@eecs.harvard.edu}
\sharedaffiliation
\affaddr{School of Engineering and Applied Sciences}\\
\affaddr{Harvard University}\\
\affaddr{Cambridge, MA}
}

\maketitle

\begin{abstract}

Energy in sensor networks is a distributed, non-transferable resource. Over
time, differences in energy availability are likely to arise. Protocols like
routing trees may concentrate energy usage at certain nodes. Differences in
energy harvesting arising from environmental variations, such as if one node
is in the sun and another is in the shade, can produce variations in charging
rates and battery levels. Because many sensor network applications require
nodes to collaborate --- to ensure complete sensor coverage or route data to
the network's edge --- a small set of nodes whose continued operation is
threatened by low batteries can have a disproportionate impact on the
fidelity provided by the network as a whole. In the most extreme case, the
loss of a single sink node may render the remainder of the network
unreachable.

While previous research has addressed reducing the energy usage of individual
nodes, the challenge of collaborative energy management has been largely
ignored. We present Integrated Distributed Energy Awareness (IDEA), a sensor
network service enabling effective network-wide energy decision making. IDEA
\textit{integrates} into the sensor network application by providing an API
allowing components to evaluate their impact on other nodes. IDEA
\textit{distributes} information about each node's load rate, charging rate,
and battery level to other nodes whose decisions affect it. Finally, IDEA
enables \textit{awareness} of the connection between the behavior of each
node and the application's energy goals, guiding the network toward states
that improve performance.

This paper describes the IDEA architecture and demonstrates its use through
three case studies. Using both simulation and testbed experiments, we
evaluate each IDEA application by comparing it to simpler approaches that do
not integrate distributed energy awareness. We show that using IDEA can
significantly improve performance compared with solutions operating with
purely local information.

\end{abstract}

\vfill\eject

\category{C.2.4}{Computer-Communication Networks}{Distributed Systems}
\category{C.3}{Special-Purpose and Application-Based Systems}{Real-time and
Embedded Systems}
\terms{Design}
\keywords{Wireless Sensor Networks, Resource Management, Resource
Distribution, Optimization}

\section{Introduction}
\label{sec-introduction}

Energy-harvesting sensor networks experience variations in load and charging
rates that threaten high-fidelity operation. Changing application demands can
produce variations in load rates, while energy-harvesting properties can
produce variations in charging rates. Energy mismanagement
can lead to reduced fidelity, when nodes' batteries empty, or wasted energy,
when nodes harvest energy they cannot store.

Energy harvesting capabilities such as solar charging further complicate the
distributed energy management task. The energy collected at each node may
vary significantly based on node placement, and the energy collected
day-to-day may vary significantly based on weather patterns. Preparing the
network for overnight operation requires capturing as much energy as possible
during the day and minimizing energy wasted by charging full batteries, while
overnight operation itself requires adjusting the network's load profile to
match the distribution of energy stored during daytime.

Fortunately, dense networks provide redundancy that can be used to control
the distribution of energy usage.  Multiple possible routing paths may
connect a node to the sink. Tuning MAC parameters allows nodes to shift
communication load to their neighbors. Sensor inputs from multiple nodes may
be redundant, allowing some to be disabled or operated at reduced fidelity.
The existence of these choices implies that it is possible to tune the energy
load of the network to better match energy availability.  Effective load
tuning can increase the fidelity provided to the application at a fixed
battery size, or allow battery sizes to be reduced while maintaining the
required fidelity level.

Existing sensor network platforms provide little support for collaborative
energy management. Approaches such TinyOS~\cite{tinyos-asplos00},
Pixie~\cite{pixie-sensys08}, Eon~\cite{eon-sensys07} and
Levels~\cite{levels-sensys07} facilitate local control only, failing when
greedy node energy minimization fails to produce the best outcome.
Network-wide solutions such as Lance~\cite{lance-sensys08},
Mercury~\cite{parkinsons-embs07}, and EnviroMic~\cite{enviromic} either
require centralized control or are tailored to the needs of a specific
application domain. In sensor networks the majority of energy consumption is
consumed by multi-node collaboration. We argue that due to the distributed
nature of energy consumption and availability, improving performance
requires consideration of both \textit{where} energy is and \textit{how much}
is being used.

Matching load to availability across the network requires \emph{integrating}
with application components producing energy load, \emph{distributing}
load and availability information to facilitate node decision making, and
\emph{awareness} of the connection between load, availability, and
application-level fidelity. We propose Integrated Distributed Energy
Awareness (IDEA), a sensor network service addressing these goals. IDEA
monitors and models the load and charge rates on each node.  To allow nodes
to reason about their impact on others, each node distributes its model
parameters, updating them as necessary to ensure continued accuracy.  IDEA
clients are responsible for estimating their own distributed energy
impact. When changing state, IDEA helps them evaluate each
proposed option using an energy objective function tailored to meet
specific application goals. By tracking availability and informing the energy
decision-making process, IDEA simplifies the construction of energy-aware
components.

Our paper makes the following contributions. First, we describe IDEA, a new
service uniting energy monitoring, load modeling and distributed state
sharing into a single service facilitating distributed decision making.
Second, we present three case studies illustrating how to use IDEA, including
a component that tunes MAC parameters, an existing routing protocol modified
to choose energy-aware routes, and a application using IDEA to determine how
to localize acoustic events.  Third, using simulation and testbed results we
compare the performance of IDEA with approaches that do not consider energy
distribution, showing that IDEA enables improvements in lifetime of up to
35\%.

The rest of this paper is organized as follows. Section~\ref{sec-motivation}
motivates the need for IDEA using a simple example. In
Section~\ref{sec-architecture} we present the IDEA architecture in detail and
describe our current implementation. We describe our three case studies in
detail in Section~\ref{sec-casestudies}. Section~\ref{sec-evaluation}
presents simulation and testbed results. We review related work in
Section~\ref{sec-related}, and Section~\ref{sec-futurework} outlines future
work and concludes.

\section{Motivation}
\label{lance-sec-motivation}

Wireless sensor networks are becoming more common for applications that focus
on reliable collection of raw signals at relatively high sampling rates, as
opposed to in-network aggregation of low-data-rate samples.  These
applications generally make use of extensive offline analysis to study the
collected data, and it is often infeasible or impossible to perform this
computation within the network itself.  Even in cases where it is possible to
shift computation to the network, a system designer may wish to extract raw
data occasionally for calibration or testing.  Examples of such applications
include structural health
monitoring~\cite{netshm-spots06,ggb-ipsn07,wimms-lynch06}, acoustic
sensing~\cite{vango,vigilnet,girod-ipsn07,enviromic}, distributed camera
networks~\cite{cyclops}, and geophysical monitoring~\cite{volcano-osdi06}.

These systems typically record data to flash at each sensor node and make use
of a reliable bulk-transfer protocol to collect data at a base station. Given
that the network is capable of sampling data at a higher rate than it can be
downloaded, it is not possible to collect the complete signal from all nodes.
The system is therefore forced to make a decision about what data to collect
and what to throw away. In most cases this decision is application-specific:
for example, a volcanologist may be chiefly interested in a specific type of
seismic tremor, and a biologist may be looking for acoustic signatures of a
specific species of marmot~\cite{girod-ipsn07}. The implication is that the
system must be able to determine the intrinsic {\em value} of a given signal
to determine whether resources should be devoted to storing and downloading
that signal.

Previous approaches have involved simple mechanisms tailored for specific
applications. For example, in the
NetSHM~\cite{netshm-spots06,netshm-emnets05} structural monitoring system,
data collection is triggered manually following an excitation of the
structure.  Sentri~\cite{ggb-ipsn07} has been used for vibration monitoring
at the Golden Gate Bridge; it is unclear from the paper how sampling and
communication are triggered, but reported experimental results suggest manual
operation. The Reventador volcano monitoring system~\cite{volcano-osdi06}
used a simple triggering algorithm to detect seismic events and initiate
reliable transfer to the base station.  The Intel Predictive Maintenance
system~\cite{intel-northseasensys} performs high-data-rate sampling staggered
over infrequent, periodic collection periods to extend system lifetime.  All
of these systems involve a tight binding of the {\em mechanisms} used to
manage storage and bandwidth with their respective application-specific {\em
policies}.

The typical approach to download management is a FIFO model where downloads
occur in a round-robin fashion across the network once a trigger occurs. In
general, new data may be sampled and stored to flash while a download is
taking place.  Therefore, the trigger frequency, download cycle duration and
the number of nodes in the network all effect the amount of data captured by
such an approach.  For example, the Flush~\cite{flush-sensys07} protocol
achieves only 8~Kbps for a reliable transfer over multiple hops; the
Fetch~\cite{volcano-osdi06} and Straw~\cite{ggb-ipsn07} protocols fare
somewhat worse. The RCRT protocol~\cite{rcrt-sensys07} is designed for a case
where all nodes are transmitting simultaneously to the sink, although this
approach severely limits the obtained per-node throughput. As a result, when
incoming data rates exceed download capacity, FIFO download management can
produce excessive delays between data acquisition and retrieval.

Lance assumes that sensor nodes contain adequate flash storage to buffer
signals prior to download.  While the popular TMote~Sky platform supports a
relatively small 1~MB flash, more recent sensor designs~\cite{shimmer} have
several GB~of flash, and we expect this trend to continue. Rather than
focusing on per-node storage, our primary concern is with limitations on
bandwidth and energy.

Our goal is to develop a general-purpose approach to bandwidth and energy
management that complements a reliable data-collection protocol. Such a
system should have several key properties.  First, it should be {\em
customizable}, allowing different applications to specify their own policies
for storage management and bandwidth prioritization. Second, the system
should target a range of optimization goals. Examples include maximizing
overall data priority, bounding energy consumption, maximizing temporal or
spatial coverage of the collected data, or achieving fairness across sensor
nodes. Third, the system should be decoupled from a specific routing
protocol, reliable collection protocol, or sensor node platform, making it
possible to leverage the system in different settings. 

\section{Network Hardware and Architecture}
\label{evaluation-sec-architecture}

\begin{figure}[t]
\begin{center}
\includegraphics[width=1.0\hsize]{./3-evaluation/figs/node.pdf}
\end{center}
\caption{\textbf{Our wireless volcano monitoring sensor node.}}
\label{evaluation-fig-node}
\end{figure}

Our volcano monitoring sensor station consists of a Moteiv TMote
Sky~\cite{moteiv} wireless sensor network node, an 8~dBi~2.4GHz external
omnidirectional antenna, one or more seismometers, a microphone, and a custom
hardware interface board. An example node with components labeled is shown in
Figure~\ref{evaluation-fig-node}. The TMote Sky was designed to run
TinyOS~\cite{tinyos-asplos00}, and all of our software development made use
of this environment. We chose the TMote Sky for several reasons. The MSP430
microprocessor provides a large number of configurable ports, easily
supporting external devices, and the large amount of flash memory was useful
for buffering collected data.

We built a custom hardware board to integrate the TMote Sky with the
seismoacoustic sensors. The board features up to four Texas Instruments
AD7710 analog to digital converters (ADCs) providing up to 24~bits per
channel of resolution. Although the MSP430 microcontroller provides on-board
ADCs, they are unsuitable for our application. First, they provide only
16~bits of resolution while we required at least 20~bits. Second,
seismoacoustic signals require an aggressive filter centered around 50~Hz.
Due to the infeasibility of implementing such a filter using analog
components, it is usually approximated digitally, requiring several factors
of oversampling. To perform this filtering, the AD7710 is sampling at over
30~kHz while presenting an output word rate of 100~Hz. The high sample rate
and computation required by digital filtering are best delegated to a
specialized device.

Each sensor node was powered by a pair of alkaline D~cell batteries.
Anticipating a remote network location, D~cells provided the best combination
of low cost, high capacity, and ready availability, and are able to power a
node for over a week. Approximately 75\% of the power drawn by each node is
consumed by the sensor interface board, primarily due to the high power
consumption of the ADCs.

For sensors, nodes are fitted with either a Geospace Industrial GS-11
geophone --- a single-axis seismometer with a corner frequency of 4.5~Hz,
oriented in the vertical plane of motion --- or triaxial Geospace Industries
GS-1 seismometers with corner frequencies of 1~Hz, yielding separate signals
in each of the three axes. Both sensors are passive instruments: ground
motion generates a voltage which is amplified and digitized by the sampling
board. In addition, each node was attached to an omnidirectional microphone,
the Panasonic WM-034BY, which has been used in other infrasonic monitoring
studies~\cite{johnson-etal-04b}.

\subsection{Network Operation}

Given the current capabilities of wireless sensor network nodes, we set out
to design a data collection network meeting the application's scientific
requirements Before motivating and explaining our design we provide a
high-level overview of typical network operation.
Figure~\ref{evaluation-fig-architecture} outlines the main system components.

\begin{figure}[t!]
\begin{center}
\includegraphics[width=0.7\hsize]{./3-evaluation/figs/architecture.pdf}
\end{center}
\caption{\textbf{Schematic representation of our sensor network
architecture.}}
\label{evaluation-fig-architecture}
\end{figure}

Each node samples two or four channels of seismoacoustic data at 100~Hz,
storing the data in local flash memory. Nodes also transmit periodic status
messages and participate in routing and time-synchronization protocols. When
a node detects an interesting event, it routes a message to the base station
laptop. If enough nodes report an event within a short time interval, the
laptop initiates data collection, which proceeds in a round-robin fashion.
Between 30~and~60~s of data is downloaded from each node with a reliable data
collection protocol, ensuring that all buffered data from the event is
retrieved. When data collection completes, nodes return to sampling and
storing sensor data.

\subsection{Overcoming High Data Rates: Event Detection and Buffering}

When designing high data-rate sensing applications an important limitation of
current sensor network nodes is low radio bandwidth. IEEE 802.15.4 radios,
such as the Chipcon CC2420, have raw data-rates of around 30~kBps. However
overheads caused by packet framing, medium access control (MAC), and multihop
routing reduce the achievable data-rate to less than 10~kBps even in a
single-hop network. Because of the high data-rates involved (600-1200~Bps
from each node) it is infeasible to continuously transmit all data.

Rather, nodes are programmed to locally detect interesting seismic events and
transmit event reports to the base station. If enough nodes trigger in a
short time interval, the base station attempts to download the last 60~s of
data from each node. This forgoes continuous data collection for increased
resolution following significant seismic events, which include earthquakes,
eruptions, or long-period (LP) events, such as tremors. The download window
of 60~s was chosen to capture the bulk of the eruptive and earthquake events,
although many LP events can exceed this window (sometimes lasting minutes or
hours). To validate our network against existing scientific instrumentation,
our network was designed for high-resolution signal collection rather than
extensive in-network processing.

Sampled data is stored in the local flash memory of the node, which is
treated as a circular buffer. Each block of data is timestamped using the
local node time, which can later be mapped onto a global time, as explained
in Section~\ref{evaluation-subsec-timerectification}. Each node runs an
\textit{event detector} on locally-sampled data. Good event detection
algorithms produce high event-detection rates while maintaining small
false-positive rates. The sensitivity of the detection algorithm links these
two metrics: a more sensitive detector correctly identifies more events at
the expense of producing more false positives. We evaluate the performance of
our event detection algorithm further in
Section~\ref{evaluation-sec-eventdetection}.

\vfill\eject

The data set produced by our previous deployment at Tungurahua
volcano~\cite{volcano-ewsn05} aided in the design of the event detector. We
implemented a short-term average/long-term average threshold detector, which
computes two exponentially-weighted moving averages (EWMAs) with different
gain constants. When the ratio between short-term average and the long-term
average exceeds a fixed threshold, the detector fires. The detector threshold
allows nodes to distinguish between low-amplitude signals, perhaps caused by
distant earthquakes, and high-amplitude signals caused by nearby volcanic
activity.

When the event detector on a node fires it routes a small message to the base
station laptop. If the base station receives triggers from 30\% of the active
nodes within a given time window, the laptop initiates data collection from
the entire network, including nodes that did not report the event. This
global filtering prevents spurious event detections from triggering a data
collection cycle. Because each node can buffer only 20 minutes of eruption
data locally, and data collection from the entire network may exceed this
envelope (we found that fetching 60~s of data from 16 nodes takes over 1
hour) each node pauses sampling and reporting events until its data has been
uploaded.

\subsection{Reliable Data Transmission and Time Synchronization}
\label{evaluation-subsec-fetch}

\begin{figure}[t!]
\begin{center}
\includegraphics[width=0.7\hsize]{./3-evaluation/figs/fetchprotocol.pdf}
\end{center}

\caption{\textbf{Operation of the \texttt{Fetch} data transfer protocol.}
Fetch requests include the block and node ID and are propagated using a
reliable broadcast protocol. 256~byte blocks are fragmented into 8~chunks,
each sent in a single radio message. The sample transfer shows packets being
lost during transmission and a repair operation which requests any missing
chunks. Once all chunks for a block are received, the next block is
requested.}

\label{evaluation-fig-fetchprotocol}
\end{figure}

Extracting high-fidelity data from a wireless sensor network is challenging
for two primary reasons. First, radio links are lossy and frequently
asymmetrical. Second, low-cost crystal oscillators on sensor nodes have low
tolerances and so clock rates vary across the network. Much prior research
has focused on addressing these challenges.

We developed a reliable data-collection protocol, called \texttt{Fetch}, that
retrieves buffered data from each node over a multihop network.
Figure~\ref{evaluation-fig-fetchprotocol} provides an overview of the
protocol's operation. Samples are buffered locally in \textit{blocks} of
256~B, tagged with a sequence number and timestamp. During transmission each
requested block is fragmented into a number of \textit{chunks}, each sent in
a single radio message. The base station laptop retrieves a block by flooding
a request to the network using \texttt{Drip}, a variant of the TinyOS
\texttt{Trickle}~\cite{trickle} data-dissemination protocol. The request
contains the target node ID, the block sequence number, and a bitmap
identifying missing chunks in the block.

The target node replies by sending the requested chunks over a multihop path
to the base station. The routing tree is constructed using
\texttt{MultiHopLQI}, a variant of the TinyOS
\texttt{MintRoute}~\cite{awoo-multihop} routing protocol modified to select
routes based on the CC2420 Link Quality Indicator (LQI) metric. Link-layer
acknowledgments and retransmissions are used at each hop to improve
reliability. Retrieving one minute of stored data from a two-channel sensor
node requires fetching 206~blocks and can takes several minutes to complete,
depending on the quality of the multihop path and the node's depth in the
routing tree. We evaluate the performance of Fetch in more detail in
Section~\ref{evaluation-sec-performance}.

Scientific volcano studies require that sampled data be accurately
timestamped. In our case, a global clock accuracy of 1~ms was sufficient. We
chose to use the Flooding Time Synchronization Protocol (FTSP)~\cite{ftsp} to
establish a global clock across our network. The published accuracy of FTSP
is very high and the TinyOS code was straightforward to integrate into our
application. One of the nodes used a Garmin GPS receiver to map the FTSP
global time to GMT. Unfortunately, FTSP failures during our field deployment
made assigning accurate timestamps challenging, as documentented in
Section~\ref{evaluation-sec-timing}.

\subsection{Command and Control}

A feature missing from most traditional volcanic data acquisition equipment
is real-time network control and monitoring. The long-distance radio link
between the observatory and the sensor network allowed the laptop to monitor
and control the network's activity. Every 10~s, each node transmits a
\textit{status message} to the base station that includes its position in the
routing tree, buffer status, local and global timestamps, battery voltage,
and other information. In addition, the base station can issue a
\textit{command} to each node, instructing it to respond with an immediate
status message, start or stop data sampling, and set various software
parameters.

We developed a Java-based GUI for monitoring the network's behavior and
manually setting parameters, such as sampling rates and event detection
thresholds. In addition, the GUI was responsible for controlling data
collection following a triggered event, moving significant complexity out of
the sensor network. All packets received by the laptop from the sensor
network were logged, facilitating later analysis of the network's operation.

The GUI also displayed a table summarizing network state, based on the
periodic status messages transmitted by each node. Each table entry included
the node ID; local and global timestamps; various status flags; the amount of
locally-stored data; depth, parent, and radio link quality in the routing
tree; and the node's temperature and battery voltage. This functionality
greatly aided sensor deployment, as a team member could rapidly determine
whether a node had joined the network and the quality of its radio
connectivity.

\section{Case Studies}
\label{idea-sec-casestudies}

Throughout the rest of the paper we demonstrate IDEA using two examples.
Section~\ref{idea-sec-evaluation} presents results demonstrating the
performance improvements that IDEA delivers for each application.

\subsection{Energy-Aware Routing}

The first example shows how to integrate IDEA with an existing routing
protocol, namely the Collection Tree Protocol (CTP)~\cite{ctp-sensys09}. CTP
is a spanning-tree routing protocol that is a standard component in
TinyOS~\cite{tinyos-asplos00}. In CTP, each node selects its parent in the
spanning tree based on the \textit{expected number of transmissions} (ETX) to
reach the sink. This is an additive metric intended to limit queue occupancy
at nodes along each routing path and maximize packet delivery rates. Although
ETX can be directly converted to an energy measure (assuming the energy costs
to transmit along a link are known), CTP does not explicitly consider energy
availability in its routing decisions.

We integrate IDEA with CTP to create \textit{ICTP}, an energy-aware
load-balancing routing protocol that combines the use of ETX with IDEA's
energy objective function. As described in
Section~\ref{idea-subsec-energyobjectivefunctions}, we parameterize the
tradeoff between pure ETX and pure energy objective using the weighting
factor $\alpha$. When $\alpha = 1$ the minimum ETX path is always used and
ICTP behaves identically to unmodified CTP. When $0 < \alpha < 1$, potential
parents with path ETX $<$ minimum ETX $\cdot \frac{1}{\alpha}$ will be
considered, with the one producing the best energy objective score chosen.
When $\alpha = 0$, ETX is not considered at all and parent selection is
performed entirely on the basis of energy. Hence, $\alpha$ indirectly
controls the degree of path stretch that is induced by energy awareness. 

Making the link power a function of its usage requires that the radio be
disabled when it is not in use. Low-power listening (LPL)~\cite{tinyos-lpl}
enables radio duty-cycling without requiring nodes arrange fixed transmission
schedules. It is well-suited for environments where network topologies and
traffic patterns are highly variable, since these variations challenge
duty-cycling techniques that assume \textit{a priori} knowledge of traffic
patterns.

The choice of LPL polling rate at a given node affects the continuous energy
drain required to periodically poll the channel as well as the cost to other
nodes to communicate with the given node. Assuming we model the radio as
drawing $I_{listen}$ and $I_{transmit}$ mA of current in listen and transmit
modes, respectively, then, given an interval between radio checks of $\gamma$
sec, the current draw required to poll the channel is $\frac{1}{\gamma} \cdot
t_{check} \cdot I_{listen}$, where $t_{check}$ is the time the radio must
remain on to detect channel activity. The cost to transmit a packet to a node
using an LPL interval of $\gamma$ is, on average, $\frac{\gamma}{2} \cdot
I_{transmit}$. We can observe then that increasing $\gamma$ or polling the
channel less frequently \textit{reduces} the current draw on the receiving
node while \textit{increasing} the communication cost on sending nodes.

When using LPL, nodes poll the radio channel at a fixed rate, listening for
packets addressed to them. The radio is shut off when not polling or sending
packets. To send a packet to another node the sender must know that node's
polling interval, and repeatedly send the packet with reduced MAC backoffs
until either the packet is acknowledged, ending the packet train and
indicating a successful transmission, or the length of the packet train
exceeds the receiver's polling interval, at which point the transmission
fails.

In order to build routes, CTP must periodically broadcast the current parent
and ETX to neighboring nodes. ICTP adds additional information to these
broadcasts, specifically the \textit{expected power}, or \textit{EPX}, for
transmissions to the node's parent. This information increases the size of
the broadcast packet sent by ICTP slightly, but does not appreciably affect
the energy consumption of the protocol's own data sharing, since the cost to
transmit a packet using LPL is a function of the receiver's polling interval,
not the packet size.

The local state space $s_n = \left\{s_n^{p_1}, s_n^{p_2}, \ldots, s_n^{p_k}
\right\}$ is defined by the node $n$'s neighbors $p_n = \left\{p^1, p^2,
\ldots, p^k \right\}$, each of them a prospective parent. CTP uses four-bit
wireless link estimation~\cite{Fonseca07} to estimate the ETX to each
neighbor, which ICTP multiplies by the power-per transmission to produce the
EPX to each neighbor, $EPX(n, p_n^i)$. Through ICTP data dissemination node
$n$ also learns the EPX from each neighbor to their current parent,
$EPX(p_n^i, \textrm{parent}(p_n^i))$. We have modified CTP to measure the
traffic rate $\delta(n)$, which is the number of packets per given interval
that node $n$ is forwarding to the sink. This is a function of both its own
packet generation rate and of the traffic induced by nodes upstream that it
is routing for. Given these parameters the projected energy load vector
$\bar{L}(s_n^{p_i})$ has two components: $L_n = \textrm{EPX}(n, p_n^i) \cdot
\delta(n)$ and $L_{p_i} = \textrm{EPX}(p_n^i, \textrm{parent}(p_n^i)) \cdot
\delta(n)$. Based on this information, IDEA chooses the best neighbor as the
node's parent.

Depending on the energy objective function chosen ICTP responds to variations
in load and charging rates in different ways. For the following discussion we
assume that the application uses the \textit{maximize first-node lifetime}
objective function described in
Section~\ref{idea-subsec-energyobjectivefunctions}, and so is willing to
trade off reduced charging rates or lifetimes at nodes that are not the
network's lifetime bottleneck in order to increase the lifetime of the node
projected to die first. Routing trees by their very nature concentrate load
near the base station, which we assume is powered. Without considering
variances in non-routing load or charging rates ICTP will attempt to balance
load across nodes that can communicate directly with the base station,
arranging the routing tree considering both the number of nodes upstream from
each of the base station's neighbors and the quality of their link to the
base station.

ICTP also responds to spatial variations in charging rates by building a tree
that is sensitive to where in the network energy is available. ICTP will
route around shadows in the network, or build routing backbones using
quickly-charging nodes or nodes whose batteries are full while attempting to
push nodes low on batteries into leaf roles, reducing or eliminating their
routing responsibilities.

Because ICTP reacts to changes in energy available by potentially choosing
routes with larger ETX, small values of $\alpha$ can begin to effect the
achieved packet delivery rate. We were able to find values of $\alpha$ that
produced significant performance improvements while leaving the delivery rate
unaltered. CTP has a persistent retransmission policy which assists us in
achieving good performance.

\subsection{Distributed Localization}

The second case study illustrates how to use IDEA to control discrete, rather
than continuous, network behavior. We consider a system designed to perform
acoustic source localization. Several previous systems have explored this
application in different contexts, including urban sniper
localization~\cite{shooter-localization} and localizing animals based on
mating calls~\cite{girod-marmots}. Using IDEA, it is possible to carefully
manage the energy load at each sensor node to prolong battery lifetime while
maintaining high localization accuracy.

Acoustic source localization involves calculating the location of an acoustic
source by collecting arrival times at several stations and performing a
back-azimuth computation. We assume a dense sensor network deployment, so
that an acoustic event is detected by many sensors. We also assume that for
each event, any set of four nodes that heard the event can correctly perform
the localization to within the application's error tolerance, since this is
the minimum number to overconstrain the localization calculation.

A centralized approach to localization requires nodes to transmit data to a
base station where the computation is performed. Because we assume that nodes
cannot accurately compute an arrival time by considering only their own
sampled data, they must transmit a sizeable amount of data to the base
station to implement the centralized strategy, with the bulk data transfer
required producing a significant load on the nodes that heard the event as
well as nodes required to route data. This approach also does not scale well
as the size of the network increases.

To avoid the overheads of centralization we want to perform the localization
inside the network. However, the cost to transmit signals and perform the
computation are still high, so it is important that localization be done in a
way sensitive to the availability of energy within the network.

When an event occurs, the goal is to select a single \textit{aggregator} node
and three \textit{signal provider} nodes from the set of nodes that detected
the event. The signal providers will transmit a portion of the acoustic
signal to the aggregator, which performs the localization computation using a
time-of-arrival and angle-of-arrival computation~\cite{Niculescu03adhoc}.
For each event we expect multiple valid aggregator and signal provider sets
to exist, each with its own energy consumption signature. We refer to a
selection of four such nodes as a \textit{localization plan}. 

Nodes that heard the signal participate in a leader election process, seeded
by the value of the IDEA energy objective function for each proposed
localization plan. Each candidate aggregator computes the energy objective
function for the localization plan or plans that they are the aggregator for.
If more than three nodes within a single hop of an aggregator heard the
event, then the aggregator will have multiple plans to consider. The
aggregator chooses the local plan with the best score and broadcasts a
message advertising that score, which is propagated to all nodes that heard
the event. If the aggregator does not hear a broadcast with a better score,
it assumes that it won the leader election and proceeds to perform the
localization as planned. 

\section{Evaluation}
\label{lance-sec-evaluation}

\begin{figure}[t]
\begin{center}
\includegraphics[width=1.0\hsize]{./4-lance/figs/linear.pdf}
\end{center}

\caption{\textbf{Per-node distribution of ADU value and energy usage for the
linear simulation experiment.} The top graph shows the amount of data value
downloaded from each node, while the bottom graph breaks down the amount of
energy used by each node into the downloading, routing and overhearing
components. Node~1 is closest to the base station.}

\label{lance-fig-linear}
\end{figure}

\begin{figure}[t]
\begin{center}
\includegraphics[width=1.0\hsize]{./4-lance/figs/error.pdf}
\end{center}

\caption{\textbf{Effect of cost vector error on optimality.} The optimization
process is guided by the cost vectors, but predicting the energy cost of
operations before they are performed can be difficult. Here we show the
impact of introducing a degree of error into the cost vectors used by the
optimizer. As can be seen, we can tolerate a relatively high degree of error,
as long as the shape of the cost vector does not change.}

\label{lance-fig-error}
\end{figure}

This section presents a careful evaluation of Lance conducted along several
lines. Using a high-level system simulator and synthetic data sets, we
evaluate the three scoring functions described in
Section~\ref{lance-subsec-optimizer}. We motivate the use of the
\textit{cost-bottleneck} scoring function and demonstrate that it performs
better than simpler alternatives. Next, we look at the effect of varying
parameters such as download bandwidth and network lifetime, as well as the
impact of errors in the cost vectors. We also present results from
experiments run on a 50-node sensor network testbed using realistic data
sets.

\subsection{Metrics and Methodology}

\begin{figure}[t]
\begin{center}
\includegraphics[width=1.0\hsize]{./4-lance/figs/crossover.pdf}
\end{center}

\caption{\textbf{Crossover between bandwidth and energy constraint dominance
as lifetime is varied.} This graph shows the transition between bandwidth and
energy constrained regions for an optimal system. The right axis shows the
percent of energy consumed by the most highly-drained node, and the left axis
shows the amount of time spent downloading.}

\label{lance-fig-crossover}
\end{figure}

As stated in Section~\ref{lance-sec-problem-definition}, the high-level goal
of Lance is to download a set of ADUs maximizing the total value subject to
energy and bandwidth constraints. The \textit{optimal} solution is defined as
the solution to the multidimensional knapsack problem, which yields a set of
downloaded ADUs $\mathcal{O} = \{a_1, a_2, ... a_n\}$ that maximize data
value subject to bandwidth and energy constraints. The total data value of
the optimal solution $\hat{v}(\mathcal{O}) = \sum_{a_i \in \mathcal{O}} v_i$.
Recall that computing the optimal solution requires \textit{a priori}
knowledge of all of the ADU values sampled by the network over time. We
define \textit{optimality} as the fraction of the data value downloaded by
Lance compared to the optimal solution. That is, given a set of downloaded
ADUs $\mathcal{L}$ with total data value $v(\mathcal{L})$, we define
optimality as $v(\mathcal{L}) / \hat{v}(\mathcal{O})$.

We begin by presenting results based on a realistic system simulator that
allows us to quickly vary parameters such as ADU data value distribution,
network topologies, download speeds, energy costs, and target lifetimes. We
also present results from Lance running on MoteLab~\cite{motelab}, a sensor
network testbed deployed over 3~floors of the Harvard EECS building. Our
simulation experiments use a 10-node linear topology as well as a 25-node
realistic tree topology shown in Figure~\ref{lance-fig-topology}(a).
Both topologies use per-node ADU download speeds based on empirical
measurements taken using the testbed. In our experiments, the ADU size is
36~KB and each node samples one ADU every 60~s (or 600~Bps of data).

We draw ADU values from several distributions in an attempt to understand
Lance's behavior as the properties of the sampled data change. Three value
distributions are used: uniform random, exponentially distributed, and Zipf
with exponent $\alpha = 1$. We also make use of an ADU value distribution
based on a 6~hour seismic signal collected at Reventador Volcano, Ecuador in
2005. Except where stated, no policy modules were used. In order to compute
optimal solutions, we frame the problem as a multi-dimensional knapsack
problem as previously described and solve the resulting instance using
\texttt{lpsolve}.

The energy costs for various operations are modeled as follows. The
background current drain of each node is set to 2~mA, based on empirical
measurements of a TMote~Sky sensor node performing high-data-rate sampling
and storing to flash. We also measured the current consumption to download an
ADU from a sensor node, and derived the energy costs for downloading ($E_d =
17.6$~mA/s), routing ($E_r = 16.9$~mA/s), and overhearing ($E_o = 2$~mA/s).
Our experiments assume that each node can only overhear its parent in the
routing tree; developing more detailed overhearing models is the subject of
future work. Computing the components of the cost vector for a particular ADU
is done by multiplying the current consumption by the ADU download time for
each node either downloading, routing, or overhearing the transmission.

\subsection{Effectiveness of Scoring Functions}
\label{lance-sec-eval-heuristics}

\begin{figure}[t]
\begin{center}
\includegraphics[width=1.0\hsize]{./4-lance/figs/motivationexample.pdf}
\end{center}

\caption{\textbf{Example simple download problem.}}

\label{lance-fig-simple}
\end{figure}

\begin{table}[t]
\begin{center}
\begin{tabular}{|l|l|ccc|}
\hline
& & \multicolumn{3}{|c|}{Scoring Functions} \\ \hline
& & Value & Cost & Cost \\
Distribution & Lifetime & Only & Total & Bottleneck \\ \hline
\multirow{3}{*}{Uniform} & 4 months & 62.4\% & 90.5\% & \textbf{93.2\%} \\
& 11 months & 43.4\% & 68.0\% & \textbf{96.9\%} \\
& 18 months & 44.6\% & 49.0\% & \textbf{90.0\%} \\ \hline
\multirow{3}{*}{Exponential} & 4 months & 83.9\% & 85.1\% & \textbf{88.0\%}
\\
& 11 months & 70.4\% & 82.0\% & \textbf{93.0\%} \\
& 18 months & 67.2\% & 72.8\% & \textbf{91.2\%} \\ \hline
\multirow{3}{*}{Zipfian} & 4 months & 84.7\% & \textbf{91.4\%} & 87.1\% \\
& 11 months & 63.8\% & 91.1\% & \textbf{96.2\%} \\
& 18 months & 53.1\% & 86.9\% & \textbf{93.8\%} \\ \hline
\end{tabular}
\end{center}

\caption{\textbf{Optimality of different scoring functions.} This table
summarizes simulation results evaluating the three different scoring
functions. Results are shown for several different lifetime targets and value
distributions. \textit{cost-bottleneck} out-performs the others in almost all
cases.}

\label{lance-fig-table}
\end{table}

We begin by evaluating the three scoring functions described
Section~\ref{lance-subsec-optimizer}. We want to see which is the most able
to approximate the optimal solution across a range of target lifetimes and
ADU distributions. As discussed earlier we expected the \textit{value-only}
scoring function, without considering the energy or bandwidth overhead of
downloading each ADU, to consume more energy downloading high-valued ADUs
when it could have increased the total data value by downloading several
slightly less-highly valued ADUs with lower costs. The \textit{cost-total}
scoring function incorporates a notion of cost, but will tend to favor nodes
closer to the base station at the expense of high-valued ADUs that are more
routing hops away. The \textit{cost-bottleneck} scoring function should
strike a balance between the two, since it considers only the most
significant cost vector component when ranking ADUs.

To develop intuition, consider the simple download problem shown in
Figure~\ref{lance-fig-simple}. The figure shows the network topology with the
energy required to reliably use each link labeled. Note that since we assume
that node radios are left on throughout the transfer the cost to route data
towards the sink is driven by the worst link along the path. links. So
Nodes~2 and 1 in the figure shown will have a cost of 100~units to route each
packet for Node~3 due to the poor link between Nodes~2 and 3. Three ADUs are
available in the network, one at each of nodes 1, 2, and 3, and the values of
each are labeled. Also assume that Nodes 1, 2 and 3 have 300, 600 and 2,000
units of available energy, respectively.

The question is: given the battery states, download costs, and values of each
ADU, which ADU should Lance download? To summarize the three ADUs available:

\begin{itemize}

\item \textbf{$ADU_1$}: located at Node~1 with value $v_1 = 10$ and cost vector
$\bar{c}_1 = \left[10, 0, 0\right]$. Note we are not considering the cost at the base
station, which we assume to be powered.
\vspace*{-0.1in}
\item \textbf{$ADU_2$}: located at Node~2 with value $v_2 = 20$ and cost vector
$\bar{c}_2 = \left[15, 15, 0\right]$.
\vspace*{-0.1in}
\item \textbf{$ADU_3$}: located at Node~3 with value $v_3 = 40$ and cost vector
$\bar{c}_3 = \left[100, 100, 100\right]$.

\end{itemize}


For this example, the three scoring functions make three different choices:

\begin{itemize}

\item \textbf{\textit{value-only}} will choose the ADU with the highest
value, $ADU_3$, regardless of the energy impact on the network.

\item \textbf{\textit{cost-total}} will compute the total cost of each ADU
and use that to weight the ADU's value. For this set of ADUs, $ADU_1$ has the
best ratio of value to total cost of the three choices: $0.1$, $0.066$ and
$0.013$ respectively.

\item \textbf{\textit{cost-bottleneck}} will first compute the bottleneck
node for each ADU by determining which node will be the most impacted by
downloading it. In this case, Node~1 is the bottleneck node for all three
ADUs, since it must devote 3.3\% of its remaining energy to $ADU_1$ (versus
0\% for Nodes~2 and 3), 5.0\% to $ADU_2$ (versus 2.5\% for Node~2 and 0\% for
Node~3), and 33\% to $ADU_3$ (versus 16\% for Node~2 and 5\% for Node~3).

\hspace{0.25in} Identifying the bottleneck cost causes
\textit{cost-bottleneck} to choose $ADU_2$, which has the best ratio of value
to  bottleneck cost (or cost to Node~1 in this case): $0.13$ versus $0.1$ for
$ADU_1$ and $0.04$ for $ADU_3$.

\end{itemize}

To convince ourselves that the \textit{cost-bottleneck} scoring function has
made the correct choice for this simple example, imagine that we can
repeatedly download ADUs with these values from each node in the network
until nodes run out of energy. We can download 30 copies of $ADU_1$ for a
total value of 300, 20 copies of $ADU_2$ for a total value of 400, or 3
copies of $ADU_3$ for a total value of 120. $ADU_2$ is the correct choice.

In many cases the node connecting the network to the powered sink quickly
becomes the bottleneck node, since all network traffic must be routed through
it. Figure~\ref{lance-fig-linear} shows simulation results using the 10-node
linear topology with exponentially-distributed ADU values, and a target
lifetime of 3~months. Nodes are numbered in increasing distance from the base
station. The graph confirms the intuition behind the scoring function
behavior. \textit{value-only} downloads roughly equal value from each node,
but fails to match the optimal performance. \textit{cost-total} downloads
more data from nodes near the sink. \textit{cost-bottleneck} comes close to
matching the optimal solution, retrieving over 99\% of the value retrieved by
the optimal solution. The smooth fall-off in the amount of value downloaded
from each node is a result of the decaying transfer speeds as the protocol
must communicate with nodes deeper in the tree. Node~1 is always the
bottleneck node during this experiment.

\begin{figure}[t]
\begin{center}
\includegraphics[width=1.0\hsize]{./4-lance/figs/zipfian.pdf}
\end{center}

\caption{\textbf{Scoring function performance on Zipfian distribution.} The
\textit{cost-bottleneck} scoring function outperforms the other two across a
range of lifetime targets.}

\label{lance-fig-zipfian}
\end{figure}

\begin{figure}[t]
\begin{center}
\includegraphics[width=1.0\hsize]{./4-lance/figs/speeds.pdf}
\end{center}

\caption{\textbf{Effect of varying download bandwidth.} Lance maintains a
high degree of optimality as the per-node download bandwidth is varied. Here
results are shown for the three synthetic distributions across a 25~node tree
topology and the \textit{cost-bottleneck} scoring function. Note that the
$y$-axis starts at 90\%.}

\label{lance-fig-speeds}
\end{figure}


Table~\ref{lance-fig-table} summarizes simulation results from a variety of
different lifetimes and value distributions, run on the 25-node tree
topology. The table shows that the \textit{cost-bottleneck} scoring function
outperforms the other two in most cases, with optimality values between
87.1\% and 96.9\%. The one exception is the 4-month Zipfian data set, where
\textit{cost-total} slightly outperforms \textit{cost-bottleneck}.
Figure~\ref{lance-fig-zipfian} shows the effect of varying the network's
target lifetime, using the 25-node tree topology and a Zipfian data value
distribution. As the figure shows, the \textit{cost-bottleneck} scoring
function maintains a high degree of optimality as the network bandwidth
changes.

\vfill\eject

To illustrate the effect of varying lifetime targets in more detail,
Figure~\ref{lance-fig-crossover} shows how the optimal solution transitions
between bandwidth-dominant and energy-dominant constraints as the target
lifetime increases. At low lifetime targets, the system is bandwidth
constrained and cannot download data fast enough to exhaust the nodes'
batteries. At high lifetime targets, the system is energy constrained and
cannot download continuously without exhausting the nodes' batteries.

\subsection{Bandwidth Adaptation}
\label{lance-sec-eval-params}

Next, we evaluate the impact of varying the download bandwidth in
Figure~\ref{lance-fig-speeds}, using the 25-node tree topology and the
\textit{cost-bottleneck} scoring function. We vary the per-node download
bandwidth from 128~to~2048~Bps and peg the target lifetime at 8~months.
As the figure shows, Lance performs well across the range of bandwidths, with
optimality greater than 97\% in all cases.

\subsection{Effect of Cost Vector Error}

Our last simulation study evaluates the effect of introducing errors into
estimated download cost. This experiment uses the 25-node topology,
\textit{cost-bottleneck} scoring function, and an exponential data
distribution. As described in Section~\ref{lance-subsec-costestimation}
estimating the cost of performing different operations \textit{a priori} can
be difficult. As shown in Figure~\ref{lance-fig-error}, even a 40\% error in
each component of the cost vector $\bar{c}_i$ for a given ADU only degrades
optimality by approximately 15\%. We conclude that accurate estimations of
download costs are not strictly necessary to achieve good performance.

Note that \textit{cost-bottleneck} depends on the ability to correctly
identify the bottleneck node. This is critical as that is the node whose cost
will be used to weight the value. In general we cannot compute the cost
perfectly \textit{a priori}, as described above. If two nodes have similar
values of $B(a_i,n)$ during the calculation of the bottleneck score, it is
possible that, assuming some error in the cost vector $\bar{c}_i$, we will
identify the wrong bottleneck and weight the value incorrectly. If the errors
are evenly distributed then over time we expect these mistakes to even out.
However, if there is a persistent miscalculation affecting one or both of the
nodes then we may repeatedly identify the wrong node as the bottleneck,
reducing optimality.

\subsection{Testbed Experiments}
\label{lance-sec-eval-policies}

\begin{figure}[t]
\begin{center}
\begin{tabular}{cc}
\includegraphics[width=0.45\hsize]{./4-lance/figs/topology25.pdf} &
\includegraphics[width=0.45\hsize]{./4-lance/figs/topology50.pdf} \\
\textbf{(a)} &
\textbf{(b)} \\
\end{tabular}
\end{center}

\caption{\textbf{Topologies for testbed experiments.} This graph shows the 25
(a) and 50 (b) node topologies used for our testbed experiments.}

\label{lance-fig-topology}
\end{figure}

\begin{figure}[t]
\begin{center}
\includegraphics[width=1.0\hsize]{./4-lance/figs/big.pdf}
\end{center}

\caption{\textbf{Optimality and energy use in the 50-node testbed
experiment.} Lance achieved near-optimal performance during this 8-hour
testbed experiment, retrieving 98\% of the value obtained by the offline
optimal algorithm.}

\label{lance-fig-big}
\end{figure}

In this section, we present results from the Lance system running on the
MoteLab testbed, in 25-node and 50-node configurations shown in
Figure~\ref{lance-fig-topology}. These experiments stress the system in a
realistic setting subject to radio interference and congestion, exercising
the multihop routing protocol, Fetch reliable data-collection protocol, and
ADU summary traffic generated by the nodes. For these experiments, we
injected artificial ADU values directly into each node rather than relying on
the nodes sampling real sensor data; this approach allows us to perform
repeatable experiments that explore a wider range of ADU value distributions.
We use the \textit{cost-bottleneck} scoring function.

Figure~\ref{lance-fig-big} shows the results of a 50-node testbed experiment
using a Zipfian data distribution and a target lifetime of 6~months. The
upper portion of the figure shows the amount of data value obtained by Lance
from each node, compared to the optimal solution (which was computed
offline). Nodes are sorted by decreasing optimal value. As the figure shows,
Lance achieves close to the optimal solution, with an optimality of 98\%
overall. In some cases, Lance incorrectly downloads more data from some nodes
and less data from others; this is due to the inherent limitations of an
online solution that cannot foresee future ADU values. The lower portion of
the figure shows the energy breakdown for each node with downloading,
forwarding, and overhearing costs shown. Some nodes consume more than others
because of their location in the routing tree. For example, node~103 uses a
great deal of energy for routing packets as it is one hop from the base
station, although no ADUs are ever downloaded from that node.

\begin{figure}[t!]
\begin{center}
\includegraphics[width=0.7\hsize]{./4-lance/figs/fill.pdf}
\end{center}

\caption{\textbf{Usage of policy modules to affect download distribution.}
Here we illustrate the use of policy modules in the context of the
volcano-monitoring application. The graph compares the download behavior of
the system with and without the policy module chain described in
Section~\ref{lance-sec-ewma-deployment}, which assign greater values to ADUs
corresponding to correlated seismic activity. The graph is colored at a
particular timestamp and node ID if we downloaded that signal from that node.
The top graph shows the ADU values over time, with the threshold for the
\texttt{filter} component of the policy module chain indicated.}

\label{lance-fig-fill}
\end{figure}

Finally, we demonstrate the use of Lance's policy modules. For this
experiment, we use a distribution of ADU data values based on a 6-hour
seismic trace collected at Reventador Volcano, Ecuador in
2005. The raw seismic data is divided into ADUs of 36
KB and ADU values $v_i$ are assigned by computing the ratio of two
EWMA~filters on the data; this assigns greater value to ADUs that contain
earthquakes, as described in Section~\ref{lance-sec-ewma-deployment}. For
each node in the 25-node topology, the ADU values from the seismic trace are
attenuated based on a hypothetical signal source and assigned to each of the
25-nodes based on their location with respect to the signal source. We then
enable a policy module chain, as described in
Section~\ref{lance-sec-policies}, that assigns higher priority to ADUs that
correspond to correlated seismic activity across the network.

Figure~\ref{lance-fig-fill} shows the result of this experiment running on
the MoteLab testbed. The upper portion of the figure shows the ADU values
over time; the middle portion, the set of ADUs downloaded by the system with
no policy modules in use; and the lower portion, the ADUs downloaded with the
policy module chain in use. As the figure shows, the policy modules cause the
network to prefer correlated seismic events and download an ADU from all
nodes in the network when such an event is detected. Gaps in the set of ADUs
downloaded are due to download timeouts. In one case, a single ADU is
downloaded spuriously due to an incorrect value being reported by that node
to the base station. This use of policy modules shows the drastic change in
the system behavior that is affected without programming the sensor nodes
themselves.

\section{Related Work}
\label{idea-sec-related}

Previous work has addressed the problem of energy load balancing in contexts
such as sensor coverage, role assignment, and energy-aware routing. Other
efforts in sensor networks have focused on reducing the power consumption at
individual nodes without considering energy distribution. Many of these
efforts are specific to a particular application or component and do not
provide a service like IDEA that can be used by a variety of applications. 

A number of existing systems such as Odyssey~\cite{odyssey-osr99},
PowerScope~\cite{powerscope-wmcsa99} and more recently
Cinder~\cite{cinder-mobiheld09}, have addressed measuring or adapting to
energy variations on battery-powered devices, primarily to support mobile
applications. This naturally produces a difference in approach from IDEA,
since IDEA targets networks consisting of multiple nodes but treated as a
single entity. Since nodes are collaborating we can enable more sharing and
ask nodes to sacrifice for each other, whereas mobile device users would
likely be upset if they discovered that their phone was running low on power
because it was trying to improve the lifetime of a stranger's phone located
nearby.

Quanto~\cite{quanto-osdi08} provides a framework for tracking and
understanding energy consumption in embedded sensor systems. The existence of
systems like Quanto was a primary motivation for IDEA, since the visibility
distributed resource tracking provides creates an opportunity to adapt to
changes in availability across the network. Currently IDEA requires that
components model their own energy consumption, which may be difficult for
components with complex behavior. We are exploring integrating Quanto into
IDEA to provide more precise tracking of energy at runtime, which could
eliminate the need for component-specific modeling and ease the process of
integrating applications with IDEA.

Eon~\cite{eon-sensys07} performs similar energy tracking and forward
projection but focuses on single-node, not network-wide adaptations.
SORA~\cite{sora-nsdi05} focuses on decentralized resource allocation based on
an economic model in which nodes respond to incentives to produce data or
perform specific tasks, with each node trying to maximize its profit for
taking a series of actions. While SORA, using correctly set prices, could
produce similar network-wide behavior to that enabled by IDEA, the connection
between prices and the behavior of the network is not completely clear. IDEA
simplifies the problem of global network control through the energy objective
function which directly expresses the application's goal.

Some work on energy-aware routing~\cite{ShahRabaey2002,381685} has addressed
equitable energy distribution within the network by probabilistically
choosing between multiple good paths between each source and sink pair.
LEACH~\cite{leach} and other similar approaches attempt to distributed energy
in an entirely decentralized way, using local heuristics to do so.

EnviroMic~\cite{enviromic} is a distributed acoustic storage system for
sensor networks. When EnviroMic nodes hear an acoustic event, a leader is
elected to assign recording tasks to nodes in the group. As storage space is
limited, EnviroMic attempts to push data to quiet sections of the network
with unused storage, balancing storage consumption across the network. Both
of these tasks involve choosing from a set of nodes that can perform the same
storage task, and so EnviroMic could be integrated with IDEA allowing the
energy overheads of data transfers to be considered.

The IDEA architecture emerged from our own prior work on energy management
for wireless sensor networks, including Lance~\cite{lance-sensys08},
Pixie~\cite{pixie-sensys08}, and Peloton~\cite{peloton-hotos09}. Lance
focused specifically on the problem of bulk data-transfer using resource
vectors and centralized control. By balancing the value and distributed cost
of retrieving sampled signals we enable near-optimal performance.  Pixie
proposed an operating system and programming framework for sensor network
nodes that promotes resources to a first-class primitive, using tickets to
manage resource consumption and brokers to enable specialized management
policies. Pixie does not consider the energy impact of a node on other nodes.

Peloton proposed an architecture for distributed resource management in
sensor networks combining state sharing, vector tickets to represent
distributed resource consumption and a decentralized architecture in which
nodes serve as ticket agents managing the resource consumption of themselves
and on behalf of nearby nodes. IDEA shares many features with Peloton and can
be viewed as the beginnings of an implementation of the Peloton design, with
state sharing to enable energy decision making and every node serving as a
ticket agent for itself but considering the distributed impact of its own
local state.

\section{Future Work and Conclusions}
\label{sec-futurework}

As future work we are interested in addressing the problem of cross-component
interaction in order to optimize several IDEA components running
simultaneously. This is complicated by the fact that there are likely to be
dependencies between components that cause decisions made by one to affect
others. As an example, the LPL intervals used by a node would effect the
power cost to use the link seen by the routing protocol. In addition we are
investigating ways to model the impact of node failure on other nodes. Many
sensor network protocols will try to work around nodes leaving the network or
going offline, but this repair process is costly and causes load within the
network to shift.

To conclude, we have described the IDEA architecture in detail, motivated its
use through three examples, and demonstrated that for each example IDEA can
improve performance by better managing distributed energy resources. We have
also discussed the process of developing an application-specific energy
objective function and shown how this can improve the performance of a
localization application while maintaining application fidelity. 

\section{Acknowledgements}

This project was supported by the National Science Foundation under grant
numbers CNS-0519675 and IIS-0926148, and by the Microsoft Corporation.


\begin{footnotesize}
\bibliographystyle{abbrv}
\bibliography{mobisys10}
\end{footnotesize}

\end{document}
