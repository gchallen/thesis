\section{Future Work and Conclusions}
\label{sec-futurework}

As future work we are interested in addressing the problem of cross-component
interaction in order to optimize several IDEA components running
simultaneously. This is complicated by the fact that there are likely to be
dependencies between components that cause decisions made by one to affect
others. As an example, the LPL intervals used by a node would effect the
power cost to use the link seen by the routing protocol. In addition we are
investigating ways to model the impact of node failure on other nodes. Many
sensor network protocols will try to work around nodes leaving the network or
going offline, but this repair process is costly and causes load within the
network to shift.

To conclude, we have described the IDEA architecture in detail, motivated its
use through three examples, and demonstrated that for each example IDEA can
improve performance by better managing distributed energy resources. We have
also discussed the process of developing an application-specific energy
objective function and shown how this can improve the performance of a
localization application while maintaining application fidelity. 

\section{Acknowledgements}

This project was supported by the National Science Foundation under grant
numbers CNS-0519675 and IIS-0926148, and by the Microsoft Corporation.
