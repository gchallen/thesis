\section{Sensor Networks for Volcanic Monitoring}

Wireless sensor networks capable of assisting the study of volcanoes are exciting the
geophysics community.  The increased scale promised by lighter,
faster-to-deploy equipment will allow them to address scientific questions
beyond the practical reach of current equipment.  Today's
typical volcanic data collection station consists of a group of bulky,
heavy, power-hungry components, difficult to move and requiring car batteries
for power.  Remote deployments often require vehicle or helicopter assist to
install equipment and perform maintenance.  Local storage is also a
limiting factor.  Stations typically log data to a local Compact Flash card
which must be periodically retrieved, requiring researchers to regularly
return to each station.

These limitations make it difficult to deploy large networks of existing
equipment. Yet these large-scale experiments hold out the possibility of
achieving important insights into the inner workings of volcanoes.  For
example, volcanic tomography~\cite{lees-tomography} is the study of a
volcano's interior structure.  Collecting and analyzing signals from 
multiple stations can produce precise mappings of the internal structure of
the volcanic edifice.  In general the precision and accuracy of these
mappings increases as stations are added to the data collection network.
Studies such as these may help resolve debates over the physical processes at
work within the volcano's interior.

\subsection{Scientific Requirements}

The geophysics community has well-established tools and techniques used to
process the signals extracted by volcanic data collection networks.  These
analytical methods require that our wireless sensor network provide data of
extremely high fidelity.  A single missed or corrupted sample can invalidate
an entire record. Small differences in sampling rates between two nodes can
frustrate analysis.  Samples must be accurately time-stamped to allow
comparisons between nodes and between networks.

An important feature of volcanic signals is that much of the data analysis
focuses on discrete events, such as eruptions, earthquakes, or tremor
activity.  Although volcanoes differ significantly in the nature of their
activity, during our deployment many of the interesting signals at
Reventador spanned less than 60~sec and occurred at the rate of several dozen
per day. This allowed us to design the network to capture
time-limited events, rather than continuous signals. 

This is not to say that recording individual events is adequate to answer all
of the scientific questions volcanologists pose.  Indeed, understanding
long-term trends requires complete waveforms spanning long time intervals.
However, the low radio bandwidth of typical wireless sensor nodes makes them
inappropriate for these types of studies and for this reason we have focused
on triggered event collection.


Volcanic studies also require large inter-node separations to obtain
widely-separated views of the seismic and infrasonic signals as they
propagate.  Array configurations used often consist of one or more, possibly
intersecting, lines of sensors. The resulting topologies raise new challenges
for sensor network design, as much previous work has focused on dense
networks where each node has a large number of neighbors.  Linear
configurations also affect achievable network bandwidth, which degrades as
data must be transmitted over multiple hops.  Node failure poses a serious
problem in sparse networks because a single node failure can obscure a
large portion of the network.
