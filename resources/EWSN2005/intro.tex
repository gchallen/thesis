\section{Introduction}

Wireless sensor networks have the potential to greatly benefit studies
of volcanic activity. Volcanologists currently use wired arrays of
sensors, such as seismometers and acoustic microphones, to monitor
volcanic eruptions. These sensor arrays are used to determine the source
mechanism and location of an earthquake or explosion, study the interior
structure of the volcano, and differentiate true eruptions from noise
or other signals (e.g., mining activity) not of volcanological
interest. A typical campaign-type study will involve placement of one or
more stations on various sites around a volcano.  Each station typically
consists of a few (less than five) wired sensors distributed over
a relatively small area (less than 100~m$^2$), and records data locally
to a hard drive or flash card.  The data must be manually retrieved
from the station, which may be inconveniently located. Power
consumption of these systems is very high, requiring large batteries
and solar panels for long deployments.

Embedded wireless sensor networks, consisting of small, low-power devices
integrating a modest amount of CPU, memory, and wireless communication,
could play an important role in volcanic monitoring.
Wireless sensor nodes have lower power requirements, are easier to 
deploy, and can support a larger number of sensors distributed over 
a wider area than current wired arrays. Using long-distance wireless
links, data can be monitored in real time, avoiding the need for
manual data collection from remote stations. Such an approach is not without
its challenges, however. Volcanic time-series data are often sampled
continuously at rates of 40~Hz or more, far greater than the 
low frequencies used in environmental monitoring
studies~\cite{gdi-ewsn04}. Due to limited radio bandwidth, however,
complete signals cannot be captured and transmitted from a large sensor
array. For such a network to run for extended periods of time, careful power
management techniques, such as triggering and in-network event detection, must be developed. 
In addition, signals from multiple sensor 
nodes must be accurately synchronized against a global time base.

To demonstrate the use of wireless sensors for volcanic monitoring,
we developed a wireless sensor network and deployed it on Volc\'{a}n
Tungurahua, an active volcano in central Ecuador. This network was 
based on the Mica2 sensor mote platform and consisted of three 
infrasonic (low-frequency acoustic) 
microphone nodes transmitting data to an aggregation
node, which relayed the data over a 9~km wireless link
to a laptop at the volcano observatory. A separate GPS
receiver was used to establish a common time base for the infrasonic
sensors. During this deployment, we recorded over 54~hours of
infrasonic signals at a rate of 102~Hz per node, resulting in
over 1.7~GB of uncompressed log data. Throughout the deployment the volcano
produced several small or moderate explosions an hour, though the rate and
energy of eruptions varied considerably.  

This small-scale deployment provided a proof-of-concept as well as a
wealth of real acoustic signals that we have used to develop a
larger-scale prototype. In order to scale to a larger number of nodes,
we have developed a distributed signal correlation scheme, in which
individual infrasonic motes capture signals locally and communicate only
to determine whether an ``interesting'' event has occurred. By only
transmitting well-correlated signals to the base station, radio 
bandwidth usage is greatly reduced.
%For example, the distributed event detector consumes only XXX~mA on
%average, while continuous sampling and transmission requires XXX~mA.

This paper describes the design, implementation, and deployment of a
wireless sensor network for volcanic monitoring. This paper makes the
following contributions. First, this is the first application to our
knowledge of mote-based sensor arrays to volcanic studies. Second, we 
demonstrate that it is possible to capture infrasonic signals from 
an erupting volcano using a wireless sensor network, and that the 
captured data correlates well with a colocated, wired seismic 
and acoustic array. Third, we develop a distributed, in-network event 
detection and correlation algorithm the greatly reduces communication
requirements for larger-scale sensor arrays. 

The rest of this paper is organized as follows. In
Section~\ref{sec-background} we present the scientific background
for the use of infrasonic arrays to monitor volcanic activity.
Section~\ref{sec-design} presents the design of our wireless sensor
network for capturing continuous infrasonic signals, and 
Section~\ref{sec-deploy} describes our experience with a real sensor
network deployment at Volc\'{a}n Tungurahua. In Section~\ref{sec-distrib}
we describe the distributed event correlation scheme, and
we evaluate its 
performance with respect to scalability, bandwidth, and power
consumption in Section~\ref{sec-evaluation}. Finally, Section~\ref{sec-conclusions} presents future work 
and concludes.



