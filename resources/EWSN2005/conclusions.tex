\section{Future Work and Conclusions}
\label{sec-conclusions}

Seismology presents many exciting opportunities for wireless sensor 
networks. Low-power, wireless sensors can greatly improve spatial
resolution, signal-to-noise, and the ability to discern interesting
volcanic events from other sources. In this paper, we have demonstrated 
the feasibility of using wireless sensors for volcanic studies. Our 
deployment at Volc\'{a}n Tungurahua provided a wealth of experience
and real data from which we can develop more sophisticated tools for
volcanic instrumentation.

Our primary direction for future work is to expand the number of
sensors in the array and distribute them over a wider aperture.
This approach will make it possible to instrument volcanoes at a
resolution that has not generally been possible with existing wired
systems. In addition, we plan to integrate seismic sensors into the array,
providing a multimodal view of volcanic activity.  Seismic sensors may also
be able to act in a triggering capacity, exploiting the precursory nature of
the seismic signals as shown earlier.

In order to meet these goals, it is critical to manage energy and
bandwidth usage carefully. By pushing computation to the sensor nodes
themselves, we can shift away from continuous data collection to allowing
the network to report only well-correlated signals. In addition, we
plan to develop distributed algorithms for source back-projection and
various filtering schemes that will further distill the seismic and
acoustic signals.  We intend to return to Tungurahua in early 2005 to
test the seismo-acoustic array and distributed event detection scheme.

Our long-term plans are to provide a permanent, reprogrammable sensor
array on Tungurahua. This resource will benefit numerous research
groups that are performing studies on the volcano, and allow
scientists to retask the network for specific experiments. Clearly,
this raises challenges in the areas of programming models and resource
management and sharing. We hope to provide a high-level language
framework for reprogramming the sensor array~\cite{regiment-dmsn04}
that will give scientists an abstract view of the network, as well as
Web-based tools for remote management~\cite{motelab}.

\section*{Acknowledgments}

The authors wish to thank Pratheev Sreetharan for his assistance with
the infrasonic sensor board design. Hassan Sultan developed tools to
analyze the sensor data. Thaddeus Fulford-Jones, Bill Walker, and Jim
MacArthur provided invaluable technical support in preparation for our
deployment. Finally, we wish to thank the Instituto Geof\'{i}sico,
EPN, Quito for their gracious hospitality and assistance with
logistics in Ecuador.

% Pratheev, Hassan, Thaddeus, Bill Walker, Jim MacArthur
% IGEPN
