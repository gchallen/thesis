% Text from Jeff Johnson

Figure 1 (2005-08-16 09 45.14) - Eruption (or explosion) earthquake
recorded with 12 elements of the wireless seismic (GS-11 sensors)
array.  Seismic traces are normalized and plotted with the y-axis
corresponding to slant distance from the summit (1750 to 3750 m).
First arrivals (i.e., compressional P-wave arrivals) have been
identified manually and correspond to the emergence of seismic energy
above background.  A second order polynomial (solid red line) is fit
to those arrivals which are robustly identified.  The move-out for
this explosion earthquake, which shows increasing arrival times with
distances from the vent region, is expected for a shallow near-vent
source.  This explosion event is corroborated with corresponding
acoustic radiation (see Figures xxx?) attributed to explosive
expansion of gas.

Figure 2 (2005-08-15 16.04.37) - Volcano-tectonic earthquake recorded
with 13 elements of the wireless seismic (GS-11 sensors) array.
Seismic traces are normalized and plotted with the y-axis
corresponding to slant distance from the summit (1750 to 4350 m).
First arrivals (i.e., compressional P-wave arrivals) have been
identified manually and correspond to the emergence of seismic energy
above background.  A second order polynomial (solid red line) is fit
to those arrivals which are robustly identified.  The earliest
arrivals for this volcano-tectonic event are recorded in the middle of
the array and arrivals at the array endpoints are relatively delayed.
This distribution implies a deeper source with increased distance to
the near vent sensor nodes.  At the same time, seismic velocity in the
uppermost cone, which is comprised of unconsolidated volcanic
deposits, is presumed to be slower.  Volcano-tectonic events such as
this are likely generated due to the fracturing of solid media
typically induced by pressurization within the edifice.

General text to include:
The 2005 Reventador wireless sensor array was established in a radial
geometry for two primary reasons: 1) to understand the attenuation
structure of volcanic explosion earthquakes, and 2) to calculate the
seismic wavefield incidence for events occurring near to the volcanic
conduit.  Both seismic attenuation, which is the study of how seismic
ground shaking decreases with propagation distance and frequency
content, and wavefield analyses are the focus of ongoing study with
the collected dataset.  

Seismic wavefield incidence is an especially important parameter to
recover because it provides us with information about the relative
depth of volcanic earthquakes (see Figure 3).  Typically depth is hard
to determine through traditional means because seismic stations at
volcanoes are most easily deployed around the periphery of the volcano
(providing good epicentral locations but poor depth constraints).  At
Reventador we deployed many stations along a radial trajectory away
from the vent in order to calculate the move-out of first arrivals
(i.e., time distribution of compressional P-Wave first arrivals).
This allows us to discriminate sources originating in the shallow
portion of the cone near the vent from events occurring more deeply.
Furthermore variations in apparent velocity, calculated from the
move-out, enable us to estimate origin depth of various events.  

For a simplified 2-D geometry, the apparent velocity (Va) of arrivals
across the array is determined by the intrinsic P-wave velocity Vp of
the medium and angle of incidence (i) of seismic energy (Va = Vp /
cos(i)).  For a near-surface source in the vicinity of the vent, the
incidence approaches zero and apparent velocity approaches intrinsic
compressional velocity of the medium.  Conversely, for a deeper source
(i.e., volcano-tectonic event), seismic angles can steeply impinge
upon the free surface of the volcanic edifice leading to infinite
apparent velocity (dotted line, inset, Figure 3).  For both types of
events presented in Figure 3, the lowest apparent velocities (1.5 to
2.0 km/s) is observed towards the far end of the array (Figure 3).
This is the value that most closely approximates the seismic P-wave
velocity of the volcano in this locale.  Knowing the velocity
structure of the volcanic edifice provides further important
geophysical constraints that are used to assess earthquake source
parameters.

