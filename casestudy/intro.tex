\section{Introduction}

Wireless sensor networks, comprised of many resource-limited nodes linked by
low-bandwidth wireless radios, have been the focus of intense research during
the last few years.  Ever since their conception, wireless sensor networks
have excited other scientific communities because of their potential to
facilitate data acquisition and scientific studies.  Collaborations between
computer scientists and other domain scientists have produced networks
capable of recording data at a scale and resolution not previously possible.
Wireless sensor networks have the potential to greatly advance the pursuit of
scientific knowledge across a wide variety of disciplines.

Two years ago, our team of computer scientists from Harvard University began
collaborating with volcanologists at the University of North Carolina, the
University of New Hampshire, and the Instituto Geof\'{i}sico in Ecuador.
Studying active volcanoes involves a wide range of instrumentation, including
arrays built to collect seismic and infrasonic (low-frequency acoustic)
signals. In the summer of 2004 we deployed a small wireless sensor network on
Volc\'{a}n Tungurahua in central Ecuador as a proof of
concept~\cite{volcano-ewsn05}.  For three days four nodes equipped with
microphones collected continuous data from the erupting volcano.

In August, 2005, we deployed a larger, more capable network on Volc\'{a}n
Reventador in northern Ecuador.  The array consisted of sixteen nodes
equipped with seismoacoustic sensors deployed over 3~km.  Collected data was
routed through a multihop network and over a long-distance radio link to an
observatory where it was logged by a laptop.  Over three weeks our network
captured 230 volcanic events, producing useful data and providing the
opportunity for us to evaluate the performance of large-scale sensor networks
for collecting high-resolution volcanic data.

In contrast with existing volcanic data acquisition equipment, our nodes are
smaller, lighter, and consume less power.  The resulting improved spatial
distribution greatly facilitates scientific studies of wave propagation
phenomena and volcanic source mechanisms.  Additionally, and extremely
important for a successful collaboration, the volcanic data collection
application includes challenging computer science problems.  Studying active
volcanoes requires {\em high data rates}, demands {\em high data fidelity} and
necessitates sparse arrays with {\em high spatial separation} between nodes.
The intersection of these scientific requirements with the current
capabilities of wireless sensor network nodes creates difficult computer
science problems requiring research and novel engineering.  
