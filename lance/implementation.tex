\section{Prototype Implementation}
\label{sec-implementation}
\label{sec-implementation-nsm}

We have implemented Lance in TinyOS 2.x~\cite{tinyos-asplos00} for 
TMote~Sky and iMote2 sensor nodes. The TMote~Sky features a
1~MB flash memory (ST~M25P80) divided into 16~sectors of 64~KB each.
Our current prototype matches the ADU size to the sector size to
simplify storage management, but this is not a fundamental limitation
in the design. Sensor nodes participate in a multihop spanning-tree 
protocol rooted
at the base station; we use the Collection Tree Protocol provided with
TinyOS~2.0 for this purpose. Nodes send a periodic storage summary to the
base station. To improve reliability, we use a
sliding window approach in which each summary includes information on
the last 5~ADUs recorded by the node. The node prioritization function
is implemented as an application-supplied NesC component conforming to
a simple API. Our prototype uses our own Fetch~\cite{volcano-osdi06}
reliable transfer protocol, although it would be straightforward to
replace this with another protocol such as Flush~\cite{flush-sensys07}.

The Lance download manager runs at the base station and receives
data from the network via a ``gateway node'' connected to the base
station by a serial cable or radio modem.  The download manager is 
implemented in Perl and makes use of several external utilities for
reading and parsing storage summary packets and sending download
requests to the network. Policy modules are implemented as separate
UNIX processes (which can be in any language; we typically use Perl) 
that read storage summaries on {\tt stdin} and produce modified
storage summaries on {\tt stdout}. A simple configuration script is
used to compose multiple policy modules into a pipe. We find that
using standard scripting languages and UNIX utilities makes it very
easy to implement a range of policy modules.
A suite of Python utilities for logging, data visualization, and 
managing the network through a GUI are also provided.
