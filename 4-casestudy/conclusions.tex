\section{Observations, Future Work and Conclusions}

Our sensor network at Volc\'{a}n Reventador was deployed for three weeks,
during which time we collected seismoacoustic signals from 230~events. We have
only just begun the process of rigorously analyzing our system's performance,
but here we share some early observations as well as knowledge acquired in
the field.

% MDW: Sounds like we are bashing Deluge; we need to be really careful how we
% position ourselves here.
In general, we were pleased with the performance of our system.  During the
19-day deployment, we were able to retrieve data from the network 61\% of the
time. Many short outages occurred because, due to the volcano's remote
location, it was often impossible to power the logging laptop around the
clock.  By far the longest continuous network outage was caused by a software
component failure which took the system offline for 3 days, until researchers
returned to the deployment site to address the problem.  

Our event-triggered model worked well.  During the deployment, 230 eruptions
were detected, and we logged nearly 107 MBytes of data.  By examining the
data downloaded from the network we were able to verify that the local and
global event detectors were functioning properly. As described above, we
disabled sampling during data collection, implying that two back-to-back
events will not be recorded by the system. In some instances, this meant that
a small seismic event would trigger data collection and a large explosion
shortly thereafter would be missed. We plan to revisit our approach to event
detection and data collection to take this into account.

% MDW: We have *way* more than two directions...
This deployment raises many exciting directions for future work.  We plan to
continue improving our sensor network design and pursuing additional
deployments at active volcanoes.  This work will focus on improving event
detection and prioritization, as well as optimization of the data collection
path. We hope to deploy a much larger (100 node) array for several months,
with continuous Internet connectivity via a satellite uplink. We are
collaborating with the SensorWebs project at NASA and JPL to allow our
ground-based sensor network to trigger satellite imaging of the volcano after
a large eruption. We assembled equipment required to test this idea at
Reventador but were unable to establish a reliable Internet connection at the
deployment site.

A more ambitious research goal involves sophisticated distributed
data processing within the sensor network itself. For example, sensor
nodes can collaborate to perform calculations of energy release,
signal correlation, source localization, and perhaps tomographic
imaging of the volcanic edifice. By pushing this computation into the
network, we can greatly reduce the radio bandwidth requirements and
scale up to much larger arrays. We are excited by the opportunities
that sensor networks have opened up for geophysical studies.

