\chapter{Conclusion}
\label{sec-cb-conclusions}
\label{sec-cb-conclusion}

Driven by the observation that low power wireless sensor technologies have
matured enough to be used to help medical care, this dissertation has shown
that building medical sensor networks with resource-limited nodes is
practical after addressing several key challenges.

We argued that a new network architecture is needed and proposed CodeBlue, a
system architecture for medical sensor networks. CodeBlue is based on the
publish/subscribe communication model in order to allow dynamic and efficient
data transmission between patient sensors and the receiving devices. CodeBlue
provides a simple query layer that supports continuous and trigger-based 
physiologic sensor queries. It also defines a flexible sensor abstraction
layer that allows new sensor hardware to be easily added to the system.

We presented lessons learned by applying an ad-hoc multicast routing protocol,
ADMR, on resource limited wireless sensors. We showed that using a novel path
delivery ratio routing metric (PATH-DR) results in significant routing
performance improvement. We also identified the scalability limitations of the
multicast layer caused by the bandwidth and memory constraints. These
scalability issues are caused by the routing tree maintenance overhead and
the limitation on routing table size. These lessons resulted in TinyADMR, a
critical component to support multicast routing in CodeBlue. 

In addition, we addressed the challenges for monitoring and evaluating 
deployed medical sensor networks. Under the resource constraints from the mote
platform, we argued that studying a deployed network must rely on a passive
approach. We designed and implemented LiveNet to facilitate practical
monitoring of a deployed sensor network without affecting the monitored
network. We showed such approach is effective and practical by applying it to
evaluate a CodeBlue deployment.

Finally, the research presented in this dissertation is merely a
first step in designing and building medical sensor networks. There are
remaining challenges to be addressed by future work. They include
avoiding network congestion, providing quality of service, and network
security. We hope our effort serves as the basis of future designs that will
address remaining issues and further advance the field of study. 
