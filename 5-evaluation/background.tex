\section{Background}
\label{evaluation-sec-background}


%\subsection{Previous work}

% Tungurahua and Reventador deployment
%Our group is one of the first to explore the use of low-power 
%wireless sensor networks for geophysical monitoring, particularly 
%on active volcanoes. Although wireless telemetry is used on 
%seismic arrays at dozens of volcanoes worldwide, such installations involve 
%large, power-hungry equipment that is not well-suited for rapid, 
%dense deployments. As a result, even many of the most most threatening 
%and active volcanoes (e.g., Mount St. Helens~\cite{malone-81}) maintain 
%networks of less than ten stations.
%One of the primary limitations of volcano seismological studies is simply
%the lack of spatial coverage on a particular volcano.  
%This is true of
%heavily-monitored volcanoes like Mt. St. Helens and Mt. Fuji as well as
%on remote locations in Ecuador.

%Our team has been collaborating for the last two years and has deployed
%two wireless arrays on volcanoes in Ecuador. Our first deployment in
%July 2004 at Volc\'{a}n Tungurahua demonstrated the proof-of-concept, 
%involving three wireless sensor nodes sampling continuous infrasonic 
%time series data~\cite{volcano-agu04,volcano-ewsn05}.  
%The sensor array transmitted continuous, real-time infrasonic signals to
%a base-station laptop via a radio modem 9~km from the deployment site.
%The network operated for 54~hours and collected data on a dozen
%eruptive events during this time. This deployment helped us to optimize
%node placement, packaging, and radio communication in a challenging
%environment. The second installation at Volc\'{a}n Reventador is the
%subject of this paper and is described in the next section.

%In this paper we focus on our second deployment in August 2005, 
%in which we deployed a larger, more sophisticated network on the 
%upper flanks of Volc\'{a}n Reventador. A previous magazine 
%article~\cite{volcano-ieeeic06} detailed the network architecture
%and an overview of the deployment itself. Briefly, the network 
%consisted of 16~nodes, each with seismometers and microphones, 
%distributed over a 3~km extent running radially from the vent. 

