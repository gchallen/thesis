\section{Background}
\label{evaluation-sec-background}

Scientists monitor volcanoes for two non-exclusive reasons: (1) to monitor
hazards by assessing the level of volcanic unrest; and (2) to understand
physical processes occurring within the volcano, such as magma migration and
eruption mechanisms~\cite{scarpa-96,mcnutt-96}.  The most common instrument
used is the seismometer, which measures ground-propagating elastic radiation
from both sources internal to the volcano (e.g., fracture induced by
pressurization) and on the surface (e.g., expansion of gases during an
eruption)~\cite{mcnutt-96}.  In addition, microphones are sometimes employed
to record {\em infrasound}, low-frequency ($<$ 20~Hz) acoustic waves
generated during explosive events.  Infrasound is useful for differentiating
shallow and surface seismicity and for quantifying eruptive styles and
intensity~\cite{johnson-etal-04b}.

\subsection{Existing volcano instrumentation}

The type of instrumentation used to study volcanoes depends on the the
science goals of the deployment. We are focused on the use of wireless
sensors for temporary field deployments involving dozens of sensor stations
deployed around an expected earthquake source region, with inter-node spacing
of hundreds of meters. A typical campaign-style deployment will last weeks to
months depending on the activity level of the volcano, weather conditions,
and science requirements.

Geophysicists often use standalone dataloggers (e.g., Reftek
130~\cite{reftek}) that record signals from seismometers and microphones to a
flash drive.  These data loggers are large and power-hungry, typically
powered by car batteries charged by solar panels.  The sheer size and weight
precludes deployments of more than a small number of stations in remote or
hazardous areas. Additionally, data must be retrieved manually from each
station every few weeks, involving significant effort. Analog and digital
radio telemetry enables real-time transmission of data back to an
observatory.  However, existing telemetry equipment is very bulky and its
limited radio bandwidth is a problem for collecting continuous data from
multiple channels.

\subsection{Sensor network challenges}

Wireless sensor networks have the potential to greatly enhance understanding
of volcanic processes by permitting large deployments of sensors in remote
areas.  Our group is one of the first to explore the use of wireless sensor
networks for volcano monitoring. We have deployed two wireless arrays on
volcanoes in Ecuador: at Volc\'{a}n Tungurahua in July
2004~\cite{volcano-ewsn05}, and at Reventador in August
2005~\cite{volcano-ieeeic06}.  The science requirements give rise to a number
of unique challenges for sensor networks, which we outline below.

{\bf High-resolution signal collection:}
Data from seismometers and microphones must be recorded at relatively high
data rates with adequate per-sample resolution.  A sampling rate of 100~Hz
and resolution of 24~bits is typical.  This is in contrast to sensor networks
targeting low-rate data collection, such as environmental
monitoring~\cite{gdi-sensys04,berkeley-redwoods}.

{\bf Triggered data acquisition:}
Due to limited radio bandwidth (less than 100~Kbps when accounting for MAC
overhead), it is infeasible to continuously transmit the full-resolution
signal.  Instead, we rely on triggered data collection that downloads data
from each sensor following a significant earthquake or eruption. This
requires sensor nodes to continuously sample data and detect events of
interest.  Event reports from multiple nodes must be collated to accurately
detect {\em global} triggers across the network.

{\bf Timing accuracy:} 
To facilitate comparisons of signals across nodes, signals must be
timestamped with an accuracy of one sample time (i.e., 10~ms at 100~Hz).
Data loggers generally incorporate a GPS receiver and use low-drift
oscillators to maintain accurate timing. However, equipping each sensor node
with a GPS receiver would greatly increase power consumption and cost.
Instead, we rely on a network time synchronization protocol~\cite{rbs,ftsp}
and a {\em single} GPS receiver. However, correcting for errors in the time
synchronization protocol requires extensive post-processing of the raw
timestamps.

%\subsection{Previous work}

% Tungurahua and Reventador deployment
%Our group is one of the first to explore the use of low-power 
%wireless sensor networks for geophysical monitoring, particularly 
%on active volcanoes. Although wireless telemetry is used on 
%seismic arrays at dozens of volcanoes worldwide, such installations involve 
%large, power-hungry equipment that is not well-suited for rapid, 
%dense deployments. As a result, even many of the most most threatening 
%and active volcanoes (e.g., Mount St. Helens~\cite{malone-81}) maintain 
%networks of less than ten stations.
%One of the primary limitations of volcano seismological studies is simply
%the lack of spatial coverage on a particular volcano.  
%This is true of
%heavily-monitored volcanoes like Mt. St. Helens and Mt. Fuji as well as
%on remote locations in Ecuador.

%Our team has been collaborating for the last two years and has deployed
%two wireless arrays on volcanoes in Ecuador. Our first deployment in
%July 2004 at Volc\'{a}n Tungurahua demonstrated the proof-of-concept, 
%involving three wireless sensor nodes sampling continuous infrasonic 
%time series data~\cite{volcano-agu04,volcano-ewsn05}.  
%The sensor array transmitted continuous, real-time infrasonic signals to
%a base-station laptop via a radio modem 9~km from the deployment site.
%The network operated for 54~hours and collected data on a dozen
%eruptive events during this time. This deployment helped us to optimize
%node placement, packaging, and radio communication in a challenging
%environment. The second installation at Volc\'{a}n Reventador is the
%subject of this paper and is described in the next section.

%In this paper we focus on our second deployment in August 2005, 
%in which we deployed a larger, more sophisticated network on the 
%upper flanks of Volc\'{a}n Reventador. A previous magazine 
%article~\cite{volcano-ieeeic06} detailed the network architecture
%and an overview of the deployment itself. Briefly, the network 
%consisted of 16~nodes, each with seismometers and microphones, 
%distributed over a 3~km extent running radially from the vent. 

