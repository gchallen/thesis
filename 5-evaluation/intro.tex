\chapter{Evaluation of 2005 Deployment}
\label{chapter-evaluation}

In this chapter, we take a hard look at the performance of a wireless sensor
network deployed on an active volcano. We evaluate its effectiveness as a
scientific instrument using two metrics: data \textit{fidelity} and
\textit{yield}. Fidelity encompasses the quality and consistency of retrieved
seismoacoustic signals, while yield reflects the quantity of data delivered
by the network. 

The core contribution of this chapter is an analysis of the efficacy and
accuracy of a volcano-monitoring sensor network as a scientific instrument.
In this chapter, we evaluate the data collected from a 19-day field
deployment of 16~wireless sensors on Reventador volcano, Ecuador, along the
following axes:

\begin{itemize}

\item \textbf{Robustness:} We find that the sensor nodes themselves were
extremely reliable but that overall robustness was limited by power outages
at the base station and a single three-day software failure. Discounting the
power outages and this single failure, mean node uptime exceeded 96\%.

\item \textbf{Event detection accuracy:} Our network was designed to trigger
data collection following volcanic events such as earthquakes and eruptions.
We measure the accuracy of our distributed event-detection algorithm, finding
that the algorithm has a zero false positive rate. However, the network
failed to detect many seismic events due to a poor choice of event-detection
parameters and limitations of our data collection protocol.

\item \textbf{Data transfer performance:} We evaluate the ability of our data
collection protocol to transfer complete signals following an event. We find
a 90th percentile \textit{event yield} (fraction of nodes for which all data
for an event was collected) of 94\% and a latency of 63~sec per radio hop for
downloading 60~sec worth of data.

\item \textbf{Timing accuracy:} Data collected by each node must be
timestamped to within a single sample time (10~ms) to enable seismological
analysis. We evaluate the stability of the underlying time synchronization
protocol (FTSP~\cite{ftsp}), and develop a novel approach to \textit{time
rectification} that accurately timestamps each sample despite failures of the
FTSP protocol. We show that this approach recovers timing with a
90th-percentile error of 6.8~msec in a 6-hop network.

\item \textbf{Data fidelity:} Finally, we take a seismological view of the
captured data and present a head-to-head comparison of data recorded by our
sensor network against a colocated data logger. We also evaluate the
consistency of the recorded signals in terms of seismic and acoustic wave
arrival times across the network, showing that the data is consistent with
expected physical models of the volcano's activity.

\end{itemize}

\XXXnote{GWA: TODO: Rewrite this.}

The rest of this chapter is organized as follows.
Sections~\ref{evaluation-sec-robustness} through
\ref{evaluation-sec-fidelity} present a detailed analysis of the network's
performance along each of the evaluation metrics described above.
Section~\ref{evaluation-sec-lessons} presents several lessons learned from
the deployment.
