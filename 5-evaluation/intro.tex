\section{Introduction}

% Sensor networks used in many scientific explorations:
%     - Habitat monitoring (GDI, original Cerpa paper)
%     - Environment (redwoods) soil monitoring
%     - Civil structures (NetSHM, Wisden, GGB)
% Despite the great interest, however, not much has been done to
% validate the sensor network's ability to provide scientifically
% meaningful data to end-users. A number of challenges arise in
% sensor nets that make this difficult: sensor calibration, node
% failure, message loss, and imprecise time synchronization.
% However, if sensor nets are ultimately going to be successful
% in scientific field work they must be held to a very high standard;
% meeting or exceeding the capabilities of existing scientific 
% instrumentation.

Sensor networks are making inroads into a number of scientific explorations,
including environmental
monitoring~\cite{rope-emnets05,berkeley-redwoods}, habitat
monitoring~\cite{cerpa-habitat,mainwaring-habitat,gdi-sensys04}, and
structural monitoring~\cite{ggb-monitoring,netshm-emnets05,wisan}.  
In each of these
domains, the use of low-power wireless sensors offers the potential to 
collect data at spatial and temporal scales that are 
not feasible with existing instrumentation.
Despite increased interest in this area, little has been done to
evaluate the ability of sensor networks to provide meaningful
data to domain scientists. A number of challenges confound such an effort,
including node failure, message loss, sensor calibration, and inaccurate time
synchronization. To successfully aid scientific studies, sensor networks
must be held to the high standards of 
%meeting or exceeding 
current scientific instrumentation.
%capabilities.

% In this paper, we take a hard
% look at the performance of a sensor network deployed on an active
% volcano with a focus on its effectiveness as a scientific instrument.
% This requires evaluation along two core metrics: data fidelity
% and data yield. Fidelity is defined in terms of the quality of
% the captured signals both in terms of individual samples as well
% as consistency of the signals across multiple nodes. Data yield is
% a measure of the quantity of data returned by the network during
% the field deployment.

In this paper, we take a hard look at the performance of a wireless sensor
network deployed on an active volcano. We evaluate its effectiveness as a
scientific instrument using two metrics: {\em
data fidelity} and {\em yield}. Data fidelity encompasses the quality and
consistency of retrieved seismoacoustic signals, while data yield reflects
the quantity of data delivered by the network. 
%and latencies associated with data collection.
%the network delivered.

% Monitoring volcanic eruptions raises new challenges for sensor
% networking as the data fidelity requirements are very high.
% First, data rates are relatively high with each node sampling
% multiple channels of data at 100 Hz. This is in contrast to
% many previous sensor network applications that focus on
% very low-rate data collection (one sample every few minutes).
% Second, signal collection must be reliable with no dropped samples
% in order to support detailed analysis of the signal. Third,
% data from multiple nodes must be accurately timestamped against
% a global clock (typically tied to GPS) to support comparisons of
% signals across multiple nodes; e.g., determining the seismic
% wave arrival.

Typical volcano monitoring studies employ GPS-synchronized data
loggers recording both seismic and acoustic signals. These provide 
high data fidelity and yield but are bulky, power
hungry, and difficult to deploy. Existing analog and digital
telemetry is similarly cumbersome. 
% MDW: The spin here is CRITICAL: we are not just replacing
% dataloggers with motes.
The use of wireless sensors could enable studies involving many 
more sensors distributed over a larger area. However, the 
science requirements pose a number of difficult challenges for sensor
networks.  First, seismoacoustic monitoring requires high data rates, 
with each node sampling multiple channels at 100~Hz. Second,
signal analysis requires complete data, necessitating reliable 
data collection. Third, volcano studies compare signals
across multiple sensors, requiring that collected data be accurately 
timestamped against a GPS-based global clock.
% MDW: "meaningful global clock" is too vague.

% The core contribution of this paper is a thorough analysis of the
% efficacy of a volcano-monitoring sensor network as a scientific
% instrument, and is the first paper (to our knowledge) to take a
% holistic view of sensor network performance and data quality against
% such a stringent measure. XXX Check this claim against GDI stuff!!!

% MDW: "core" makes it stronger. 
The core contribution of this paper is an analysis of the efficacy 
and accuracy of a volcano-monitoring sensor network as a scientific 
instrument. This is the
first paper to our knowledge to take a science-centric view of 
a sensor network with such demanding data-quality requirements. 
In this paper, we evaluate the data collected from 
a 19-day field deployment of 16~wireless sensors on 
Reventador volcano, Ecuador, along the following axes:

\begin{list}{$\bullet$}{\setlength{\topsep}{0.05in}
                        \setlength{\leftmargin}{0.15in}
                        \setlength{\itemsep}{0.05in}}
%\begin{itemize}{}{\setlength{\topsep}{1.0in}}
%   - Robustness: How effectively did the network continuously 
%     report data during its lifetime. We find that the sensor
%     nodes themselves were very reliable (with uptimes exceeding
%     99%) but the base station infrastructure suffered numerous
%     outages due to power failures.

\item 
{\bf Robustness:} We find that the sensor nodes
themselves were extremely reliable but that overall robustness 
was limited by power outages at the base station and a single 
three-day software failure. Discounting the power outages
and this single failure, mean node uptime exceeded 96\%.
%
%   - Event detection accuracy: Our network was designed to
%     trigger data collection following earthquakes and eruptions
%     and we evaluate the effectiveness of our distributed event
%     detector in capturing signals of interest.
%
% MDW: "interesting seismoacoustic events" is jargon and too vague.
\item 
{\bf Event detection accuracy:} Our network was designed to 
trigger data collection following volcanic events such as 
earthquakes and eruptions. We measure the accuracy of our
distributed event-detection algorithm, finding that the 
algorithm has a zero false positive rate. However, the network
failed to detect many seismic events due to a poor choice of 
event-detection parameters and limitations of our data collection
protocol.
%Establishing detection 
%accuracy is hindered by the lack of complete ground truth.
%
%
%   - Performance: We evaluate the ability of the network to deliver
%     complete signals to the base station following an event, 
%     both in terms of the yield as well as throughput and latency.
%
\item 
{\bf Data transfer performance:} We evaluate the ability of our
data collection protocol to transfer complete signals following an 
event. We find a 90th percentile {\em event yield} (fraction of nodes for
which all data for an event was collected) of 94\% and a latency 
of 63~sec per radio hop for downloading 60~sec worth of data.

%   - Timing accuracy: As mentioned above, time synchronization is
%     essential. We evaluate both the raw performance of the
%     underlying time synchronization protocol (FTSP) and describe
%     a new technique for time rectification of the captured signals
%     allowing us to map each sample onto a global, GPS-based
%     timestamp despite only having a single GPS receiver in the field
%     and failures of the FTSP protocol.
%
\item 
{\bf Timing accuracy:} Data collected by each node
must be timestamped to within a single sample time (10~ms) to
enable seismological analysis. We evaluate the
stability of the underlying time synchronization protocol 
(FTSP~\cite{ftsp}), and develop a novel approach to {\em time 
rectification} that accurately timestamps each sample despite
failures of the FTSP protocol. We show that this approach
recovers timing with a 90th-percentile error of 6.8~msec in a 6-hop network.
%
%   - Data fidelity: Finally, we take a seismological view of the
%     captured data and present a head-to-head comparison of signals
%     recorded by our sensor network against a data logger colocated
%     with one of our nodes; and also look at the consistency of the
%     data across the network, with an interesting seismological
%     finding that suggests many earthquakes originated near the
%     {\em middle} of our sensor array rather than at or near the
%     vent.
%
\item 
{\bf Data fidelity:} Finally, we take a seismological view of
the captured data and present a head-to-head comparison of data 
recorded by our sensor network against a colocated data logger.  
We also evaluate the consistency of the recorded signals in terms 
of seismic and acoustic wave arrival times across the network, 
showing that the data is consistent with expected physical models 
of the volcano's activity.
% MDW: I think the discrimination is irrelevent here.
%which are used to
%discriminate different types of volcanic activity.
%
%\end{itemize}
\end{list}

The rest of this paper is organized as follows. The next section
provides background on the use of wireless sensors for volcano
monitoring and outlines the underlying science goals. In 
Section~\ref{sec-design} we describe the architecture of our 
system and the field deployment at Reventador.
Sections~\ref{sec-robustness}~through~\ref{sec-fidelity} present
a detailed analysis of the network's performance along each of the
evaluation metrics described above. Section~\ref{sec-related} 
discusses related work and Section~\ref{sec-lessons} presents several
lessons learned from the deployment. Section~\ref{sec-conclusions} 
outlines future work and concludes.

