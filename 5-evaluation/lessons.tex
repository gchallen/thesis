\section{Lessons Learned}
\label{evaluation-sec-lessons}

% MDW: I wanted to reword this since the tone was a bit lofty ("we spent all
% this time/money so you can save some of yours...") 
Sensor network deployments, particularly in remote areas, involve significant
cost in terms of time and equipment. Failures of hardware and software can
have a negative impact on the uptake of this technology by domain science
experts. Our experiences at Reventador have yielded a number of valuable
lessons for future sensor network deployments. 

%\begin{itemize}

%\item 
{\bf 1. Ground truth and self-validation mechanisms are critical:} We did not
initially consider colocating several of our wireless sensors with existing
data loggers in order to establish ground truth. This would have clearly
aided our analysis, though we were fortunate to locate one of our sensors
near (but not immediately adjacent to) the RVEN station.
%validate the system output we should have planned to colocate trusted, wired
%data loggers alongside several of our stations. This would have made the
%time synchronization and data validity analyses much easier.  
In addition, self-validation mechanisms are needed to provide detailed
information on the health and accuracy of the data recorded by the network.
The periodic ``heartbeat'' messages that we built into our system proved
essential to remotely tracking system operation.

%\item 
{\bf 2. Coping with infrastructure and protocol failures:} As discussed
previously, the sensor nodes themselves were the most reliable components of
the system. Even without classifying the 3-day network outage as an
infrastructure failure, this downtime was far exceeded by outages caused by
power failures at the base station. 
% MDW: Gotta keep in mind that many folks reading this paper are quite senior
% and will have "seen it all" - let's not be preachy.
We did not devote enough attention to assuring the reliability of the base
station and radio modem infrastructure, assuming it would be a trivial matter
of plugging into wall power. This single point of failure was more fragile
than expected.

Additionally, several pieces deployed software, including Deluge and FTSP,
exhibited failures in the field than we not had expected given our laboratory
experiments.  These failures both speak for and show the limitations of
careful, pre-deployment testing.  We were fortunate to be able to correct
protocol errors in the field and during post-processing, but the risk of
uncorrectable problems will lead us towards more rigorous testing and
analysis in the future.

%\item 
{\bf 3. Building confidence inside cross-domain scientific collaborations:}
It is important when working with domain scientists to understand their
expectations and plan carefully to meet them. There is a clear tension
between the desire of CS researchers to develop more interesting and
sophisticated systems, and the needs of domain science, which relies
upon thoroughly validated instrumentation. Pushing more
complexity into the sensor network can improve lifetime and
performance, but the resulting system must be carefully validated
before deployment to ensure that the resulting data is scientifically
accurate.

% MDW: Here "us" refers to all authors of the paper, including Jeff and
% Jonathan. Gotta be careful.
Good communication between CS and domain scientists is also critical.  During
the deployment, the seismologists were eager to see the collected signals,
which were initially in an unprocessed format with timing errors as described
earlier. From the CS perspective, the early data provided evidence of
successful data collection, but from the geophysics perspective it
highlighted failures in the time synchronization protocol. It took a great
deal of effort after the deployment to build confidence in the validity of
our data.

%\end{itemize}

\section{Conclusions and Future Work}
\label{evaluation-sec-conclusions}
\label{evaluation-sec-future}
\label{evaluation-sec-futurework}

As sensor networks continue to evolve for scientific monitoring, taking a
domain science-centric view of their capabilities is essential.
%While most aspects of our system were developed with geophysical monitoring
%in mind, there is a natural tension between the needs of domain scientists
%and the desire to push the technology envelope. 
In this paper, we have attempted to understand how well a wireless sensor
network can serve as a scientific instrument for volcano monitoring.  We have
presented an evaluation of the data fidelity and yield of a real sensor
network deployment, subjecting the system to the rigorous standards expected
for geophysical instrumentation.

We find that wireless sensors have great potential for rapid and dense
instrumentation of active volcanoes, although challenges remain including
improving reliability and validating the timing accuracy of captured signals.
The network was able to detect and retrieve data for a large number of
seismic events, although our event detection parameters require tuning to
capture more signals.
% MDW: I may be misinterpreting the data in
% figs/robustness/nodesalive/node-downtime.dat. From what I can tell, we had
% a average of 3.63% downtime of the nodes themselves, and 31.23% downtime of
% the node-level and basestation-level outages.  Taking 31.23-3.63 I get
% 27.6%. 
In terms of reliability, base station outages affected the network about
27\%~of the time during our deployment, with a single software failure
causing a 3-day outage. However, nodes appeared to exhibit an uptime of 96\%,
which is very encouraging. Clearly, work is needed to improve the robustness
of the base station and system software infrastructure.

% 24 Aug 2006 : GWA : This seems to just have been tossed here and sounds
% awkward.
%For those events that were detected, our reliable data collection 
%protocol had a 90th percentile event yield of 94\%.

Our most difficult challenge was correcting the timing in the captured
signals and validating timing accuracy. A comparative analysis against a
GPS-synchronized standalone data logger shows very good correlation: 23~out
of 28~events correlated against the Reftek broadband station exhibited lag
times within the expected 47~ms window.  Across the sensor array, only 2~out
of~124 P-wave arrivals fall outside of an expected velocity envelope,
suggesting that our timing rectification is very consistent.  This is further
reinforced by linear regression of acoustic wave arrival times with $R^2$
values of greater than $0.99$.  Finally, preliminary analysis of the recorded
seismic signals is consistent with expected volcanic activity, exhibiting
differences between explosion-related activity and deep-source events.

{\bf Future directions:} Our group is continuing to develop sensor networks
for volcano monitoring and we expect to conduct future deployments. Our
eventual goal is to design a large (50 to 100 node) sensor array capable of
operating autonomously for an entire field season of three~months.

A primary concern for future work is reducing power consumption to extend
network lifetimes. Although we did not experience power-related failures of
sensor nodes, we were fortunate that the deployment logistics permitted us to
change batteries as needed. The highest power draw on our platform is the
sampling board, which cannot be powered down since we must sample
continuously. One approach is to perform more extensive signal analysis on
the sensor nodes to reduce the amount of data that must be transmitted
following an event. However, geophysicists are accustomed to obtaining
complete signals, so we must balance network lifetime with signal fidelity. 

In addition, we are interested in exploring novel approaches to programming
large sensor arrays to perform collaborative signal processing. Domain
scientists should not have to concern themselves with the details of sensor
node programming. We plan to develop a high-level programming interface to
facilitate more rapid adoption of this technology. 

% MDW: We might be able to keep this but I think it's been said
% in the event-detection section.
%
%{\bf Lesson \#3:} {\em Establishing ground truth is more challenging
%than first expected.} Although we deployed several wired seismic stations
%near our array using their data to establish ground truth.  As discussed in
%Section~\ref{FIXME}, using the data collected by the Reftek data loggers to
%validate the eruption detector proved suprisingly difficult. We believe that
%this was primarily due to differences in site, instrument and data logger
%response. It was certainly aggravated by the timing problems that rendered
%data collected by one of the wired stations suspect. Future deployments will
%have to expend more energy to ensure the reliability of the stations set up
%to establish ground truth. In particular, in the future we intend to not only
%colocated wired and wireless stations but to to split the output of a single
%sensor into colocated stations, giving us identical signals which would
%greatly facilate fidelity analysis.

