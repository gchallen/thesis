\section{Related Work}
\label{idea-sec-related}

Previous work has addressed the problem of energy load balancing in contexts
such as sensor coverage, role assignment, and energy-aware routing. Other
efforts in sensor networks have focused on reducing the power consumption at
individual nodes without considering energy distribution. Many of these
efforts are specific to a particular application or component and do not
provide a service like IDEA that can be used by a variety of applications. 

A number of existing systems such as Odyssey~\cite{odyssey-osr99},
PowerScope~\cite{powerscope-wmcsa99} and more recently
Cinder~\cite{cinder-mobiheld09}, have addressed measuring or adapting to
energy variations on battery-powered devices, primarily to support mobile
applications. This naturally produces a difference in approach from IDEA,
since IDEA targets networks consisting of multiple nodes but treated as a
single entity. Since nodes are collaborating we can enable more sharing and
ask nodes to sacrifice for each other, whereas mobile device users would
likely be upset if they discovered that their phone was running low on power
because it was trying to improve the lifetime of a stranger's phone located
nearby.

Quanto~\cite{quanto-osdi08} provides a framework for tracking and
understanding energy consumption in embedded sensor systems. The existence of
systems like Quanto was a primary motivation for IDEA, since the visibility
distributed resource tracking provides creates an opportunity to adapt to
changes in availability across the network. Currently IDEA requires that
components model their own energy consumption, which may be difficult for
components with complex behavior. We are exploring integrating Quanto into
IDEA to provide more precise tracking of energy at runtime, which could
eliminate the need for component-specific modeling and ease the process of
integrating applications with IDEA.

Eon~\cite{eon-sensys07} performs similar energy tracking and forward
projection but focuses on single-node, not network-wide adaptations.
SORA~\cite{sora-nsdi05} focuses on decentralized resource allocation based on
an economic model in which nodes respond to incentives to produce data or
perform specific tasks, with each node trying to maximize its profit for
taking a series of actions. While SORA, using correctly set prices, could
produce similar network-wide behavior to that enabled by IDEA, the connection
between prices and the behavior of the network is not completely clear. IDEA
simplifies the problem of global network control through the energy objective
function which directly expresses the application's goal.

Some work on energy-aware routing~\cite{ShahRabaey2002,381685} has addressed
equitable energy distribution within the network by probabilistically
choosing between multiple good paths between each source and sink pair.
LEACH~\cite{leach} and other similar approaches attempt to distributed energy
in an entirely decentralized way, using local heuristics to do so.

EnviroMic~\cite{enviromic} is a distributed acoustic storage system for
sensor networks. When EnviroMic nodes hear an acoustic event, a leader is
elected to assign recording tasks to nodes in the group. As storage space is
limited, EnviroMic attempts to push data to quiet sections of the network
with unused storage, balancing storage consumption across the network. Both
of these tasks involve choosing from a set of nodes that can perform the same
storage task, and so EnviroMic could be integrated with IDEA allowing the
energy overheads of data transfers to be considered.

The IDEA architecture emerged from our own prior work on energy management
for wireless sensor networks, including Lance~\cite{lance-sensys08},
Pixie~\cite{pixie-sensys08}, and Peloton~\cite{peloton-hotos09}. Lance
focused specifically on the problem of bulk data-transfer using resource
vectors and centralized control. By balancing the value and distributed cost
of retrieving sampled signals we enable near-optimal performance.  Pixie
proposed an operating system and programming framework for sensor network
nodes that promotes resources to a first-class primitive, using tickets to
manage resource consumption and brokers to enable specialized management
policies. Pixie does not consider the energy impact of a node on other nodes.

Peloton proposed an architecture for distributed resource management in
sensor networks combining state sharing, vector tickets to represent
distributed resource consumption and a decentralized architecture in which
nodes serve as ticket agents managing the resource consumption of themselves
and on behalf of nearby nodes. IDEA shares many features with Peloton and can
be viewed as the beginnings of an implementation of the Peloton design, with
state sharing to enable energy decision making and every node serving as a
ticket agent for itself but considering the distributed impact of its own
local state.
