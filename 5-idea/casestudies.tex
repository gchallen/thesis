\section{Case Studies}
\label{idea-sec-casestudies}

Throughout the rest of the paper we demonstrate the utility of the
information IDEA provides using three examples, each showing how components
that determine energy load distribution can tune their behavior using IDEA.
We introduce each example here, describing their implementation and
interaction with IDEA in detail. Section~\ref{idea-sec-evaluation} presents
results demonstrating the performance improvements that IDEA delivers for
each application.

\subsection{LPL Tuning}
\label{idea-subsec-lpltuning}

Low-power listening is a technique that enables radio duty-cycling without
requiring that nodes arrange fixed transmission schedules. It is well-suited
for environments where network topologies and traffic patterns are
highly variable, since these variations challenge duty-cycling techniques
that assume a priori knowledge of traffic patterns.

When using LPL, nodes poll the radio channel at a fixed rate, listening for
packets addressed to them. The radio is shut off when not polling or
sending packets. To send a packet to another node the sender must know that
node's polling interval, and repeatedly sends the packet with reduced MAC
backoffs until either the packet is acknowledged, ending the packet train and
indicating a successful transmission, or the length of the packet train
reaches the length of the receiver polling interval, at which point the
transmission fails.

The choice of LPL polling rate at a given node affects the continuous energy
drain required to periodically poll the channel as well as the cost to other
nodes to communicate with the given node. Assuming we model the radio as
drawing $I_{listen}$ and $I_{transmit}$ mA of current in listen and transmit
modes, respectively, then, given an interval between radio checks of $\gamma$
sec, the current draw required to poll the channel is $\frac{1}{\gamma} \cdot
t_{check} \cdot I_{listen}$, where $t_{check}$ is the time the radio must
remain on to detect channel activity. The cost to transmit a packet to a node
using an LPL interval of $\gamma$ is, on average, $\frac{\gamma}{2} \cdot
I_{transmit}$. We can observe then that increasing $\gamma$ or polling the
channel less frequently \textit{reduces} the current draw on the receiving
node while \textit{increasing} the communication cost on sending nodes.

On the CC2420 the receive and sends costs $I_{listen}$ and $I_{trasmit}$ are
similar, and the radio can rapidly leave and return to a low-power state so
$t_{check}$ is short, on the order of 10 ms, allowing the continuous receive
cost to be minimized. As a point of comparison, using a $0.5$ second check
interval produces a current draw of 0.37 mA while requiring 4.35 mAs to send
a single packet. Put another way, sending a single packet requires as much
energy as polling the channel for over 11 seconds.

Adjusting LPL intervals offers a way of changing the energy consumption for
communication between two nodes, and an opportunity for IDEA to tune the
intervals to match the availability of energy within the network. To develop
intuition about the tuning process, we consider a simple example where Node 1
is transmitting packets to Node 2. If Node 1 has a lot of energy while Node 2
has little, then Node 2 should poll the channel slowly and let Node 1 pay the
high per-packet penalty. On the other hand, if Node 2 has a lot of energy
while Node 1 has little, then Node 2 should poll the channel rapidly,
increasing its own energy consumption but reducing the per-packet cost to
Node 1.

IDEA allows us to build a component to tune the LPL parameters on each node
adaptively. Our local state space $s_n = \left\{s_n^5, s_n^6, \ldots,
s_n^{10} \right\}$ where $s_n^j$ corresponds to polling at intervals of $2^j$
on node $n$. For each state $s^j_n$, we construct the projected energy load
vector $\bar{L}(s^j_n)$ out of two components: one measuring the receive cost
to node $n$, the other measuring the transmission cost to other nodes to send
to node $n$. The receive cost on node $n$, $\bar{r}_n$, has only a single
component for node $n$, $r_n^n(s_n^j) = \frac{1024}{2^j} \cdot 0.010
\textrm{sec} \cdot 19.7$ mA, where $0.010$ sec is the check interval and $19.7$ mA is the
radio receive current. The transmission cost to nodes sending to node $n$,
$\bar{t}_n$, has components of the form $t^i_n(s_n^j) = \frac{1}{2} \cdot
\frac{2^j}{1024} \textrm{sec} \cdot \delta(i,n) \cdot 17.4$ mA, where $\delta(i,n)$
is the rate at which node $i$ is sending packets to node $n$ and $17.4$ mA is
the radio transmission current. We construct the total energy load vector
$\bar{L}_n(s^j_n)$ as the component-wise sum of $\bar{r}_n$ and $\bar{t}_n$,
and pass this information to IDEA to evaluate each state.

When the LPL tuning component switches states, it must propagate this
information to nearby nodes that might be sending it data. We use the ability
of IDEA to propagate component state to disseminate this information. The
tuning component intercepts outgoing transmissions, queries IDEA for the
correct LPL interval to use for the given destination, and sets the packet's
LPL interval accordingly.

Changing the LPL interval also effects the total throughput possible over the
link, which provides the component-specific measure of desirability, although
the relationship is complicated by the ability of LPL to bunch transmissions
to amortize the cost of awakening the receiver. For our evaluation we chose
to set the tradeoff factor $\alpha=1$ and optimize only for energy, since the
throughput of the link was not a limiting factor at the data rates we tested.

Finally, low-power probing (LPP) approaches available in
Contiki~\cite{contiki} and made possible by BackCast~\cite{backcast-hotnets}
improve on the LPL approach by using receiver-initiated probing to eliminate
the high channel contention caused by LPL's packet trains. However, they
produce similar energy consumption patterns to LPL and could be tuned in the
same way. We have focused on tuning LPL parameters due to LPL's availability
in the standard TinyOS distribution, but are exploring LPP-based approaches
as future work.

\subsection{Energy-Aware Routing}

Our second example shows how to integrate IDEA with an existing routing
protocol, namely the Collection Tree Protocol (CTP)~\cite{ctp-sensys09}. CTP
is a spanning-tree routing protocol that is a standard component in
TinyOS~\cite{tinyos-asplos00}.  In CTP, each node selects its parent in the
spanning tree based on the {\em expected number of transmissions} (ETX) to
reach the sink. This is an additive metric intended to limit queue occupancy
at nodes along each routing path and maximize packet delivery rates. Although
ETX can be directly converted to an energy measure (assuming the energy costs
to transmit along a link are known), CTP does not explicitly consider energy
availability in its routing decisions.

We integrate IDEA with CTP to create {\em ICTP}, an energy-aware
load-balancing routing protocol that combines the use of ETX with IDEA's
energy objective function.  As described in
Section~\ref{idea-subsec-energyobjectivefunctions}, we parameterize the tradeoff
between pure ETX and pure energy objective using the weighting factor
$\alpha$.  When $\alpha = 1$ the minimum ETX path is always used and ICTP
behaves identically to unmodified CTP. When $0 < \alpha < 1$, potential
parents with path ETX $<$ minimum ETX $\cdot \frac{1}{\alpha}$ will be
considered, with the one producing the best energy objective score chosen.
When $\alpha = 0$, ETX is not considered at all and parent selection is
performed entirely on the basis of energy.  Hence, $\alpha$ indirectly
controls the degree of path stretch that is induced by energy awareness. 

In order to build routes, CTP must periodically broadcast the current parent
and ETX to neighboring nodes.  ICTP adds additional information to these
broadcasts, specifically the {\em expected power}, or {\em EPX}, for
transmissions to the node's parent. This information increases the size of
the broadcast packet sent by ICTP slightly, but does not appreciably affect
the energy consumption of the protocol's own state sharing, since the cost to
transmit a packet using LPL is a function of the receiver's polling interval,
not the packet size.

The local state space $s_n = \left\{s_n^{p_1}, s_n^{p_2}, \ldots, s_n^{p_k}
\right\}$ is defined by the node $n$'s neighbors $p_n = \left\{p^1, p^2,
\ldots, p^k \right\}$, each of them a prospective parent. CTP uses four-bit
wireless link estimation~\cite{Fonseca07} to estimate the ETX to each
neighbor, which ICTP multiplies by the power-per transmission to produce the
EPX to each neighbor, $EPX(n, p_n^i)$. Through ICTP state dissemination node
$n$ also learns the EPX from each neighbor to their current parent,
$EPX(p_n^i, \textrm{parent}(p_n^i))$. We have modified CTP to measure the
traffic rate $\delta(n)$, which is the number of packets per given interval
that node $n$ is forwarding to the sink. This is a function of both its own
packet generation rate and of the traffic induced by nodes upstream that it
is routing for. Given these parameters the projected energy load vector
$\bar{L}(s_n^{p_i})$ has two components: $L_n = \textrm{EPX}(n, p_n^i) \cdot
\delta(n)$ and $L_{p_i} = \textrm{EPX}(p_n^i, \textrm{parent}(p_n^i)) \cdot
\delta(n)$. Based on this information, IDEA chooses the best neighbor as the
node's parent.

Depending on the energy objective function chosen ICTP responds to variations
in load and charging rates in different ways. For the following discussion we
assume that the application uses the \textit{maximize first-node lifetime}
objective function described in
Section~\ref{idea-subsec-energyobjectivefunctions}, and so is willing to trade off
reduced charging rates or lifetimes at nodes that are not the network's
lifetime bottleneck in order to increase the lifetime of the node projected
to die first. Routing trees by their very nature concentrate load near the
base station, which we assume is powered. Without considering variances in
non-routing load or charging rates ICTP will attempt to balance load across
nodes that can communicate directly with the base station, arranging the
routing tree considering both the number of nodes upstream from each of the
base station's neighbors and the quality of their link to the base station.

ICTP also responds to spatial variations in charging rates by building a tree
that is sensitive to where in the network energy is available. ICTP will
route around shadows in the network, or build routing backbones using
quickly-charging nodes or nodes whose batteries are full while attempting to
push nodes low on batteries into leaf roles, reducing or eliminating their
routing responsibilities.

Because ICTP reacts to changes in energy available by potentially choosing
routes with larger ETX, small values of $\alpha$ can begin to effect the
achieved packet delivery rate. We were able to find values of $\alpha$ that
produced significant performance improvements while leaving the delivery rate
unaltered. CTP has a persistent retransmission policy which assists us in
achieving good performance.

\subsection{Distributed Localization}

The third case study illustrates how to use IDEA to control discrete, rather
than continuous, network behavior. We consider a system designed to perform acoustic
source localization. Several previous systems have explored this application
in different contexts, including urban sniper
localization~\cite{shooter-localization} and localizing animals based on
mating calls~\cite{girod-marmots}.  Using IDEA, it is possible to carefully
manage the energy load at each sensor node to prolong battery lifetime while
maintaining high localization accuracy.

Acoustic source localization involves calculating the location of an acoustic
source by collecting arrival times at several stations and performing a
back-azimuth computation. We assume a dense sensor network deployment, so
that an acoustic event is detected by many sensors. We also assume that for
each event, any set of four sensors that heard the event can correctly
perform the localization to within the application's error tolerance. 

A centralized approach to localization requires nodes to transmit data to a
base station where the computation is performed. Because we assume that nodes
cannot accurately compute an arrival time by only considering their own
sampled data, they must transmit a sizeable amount of data to the base
station to implement the centralized strategy, with the bulk data transfer
required producing a significant load on the nodes that heard the event as
well as nodes required to route data. This approach also does not scale well
as the size of the network increases.

To avoid the overheads of centralization we want to perform the localization
inside the network. However, the cost to transmit signals and perform the
computation are still high, so it is important that localization be done in a
way sensitive to the availability of energy within the network.

When an event occurs, the goal is to select a single \textit{aggregator} node
and three \textit{signal provider} nodes from the set of nodes that detected
the event. The signal providers will transmit a portion of the acoustic
signal to the aggregator, which performs the localization computation using a
time-of-arrival and angle-of-arrival computation~\cite{Niculescu03adhoc}.
For each event we expect multiple valid aggregator and signal provider sets
to exist, each with its own energy consumption signature. We refer to a
selection of four such nodes as a \textit{localization plan}. 

Nodes that heard the signal participate in a leader election process, seeded
by the value of the IDEA energy objective function for each proposed
localization plan. Each candidate aggregator computes the energy objective
function for the localization plan or plans that they are the aggregator for.
If more than three nodes within a single hop of an aggregator heard the
event, then the aggregator will have multiple plans to consider. The
aggregator chooses the local plan with the best score and broadcasts a
message advertising that score, which is propagated to all nodes that heard
the event. If the aggregator does not hear a broadcast with a better score,
it assumes that it won the leader election and proceeds to perform the
localization as planned. 
