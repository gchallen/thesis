\chapter{Conclusion}
\label{chapter-conclusion}

While the last century was marked by a new understanding of our world at
microscopic scales, the next will be driven by holistic understanding of
large-scale systems. Wireless sensor networks composed of tens or hundreds
of low-power, resource-constrained devices, can aid scientific exploration by
providing high-fidelity data at scales more difficult to achieve with
traditional instrumentation. This dissertation has helped demonstrate the
promise of this new technology to enable the macroscope and macroscopic
science.

We began by describing how we tackled a single, high data-rate scientific
application: volcano monitoring. Through a series of field deployments
working closely with seismologists, we acquired the domain knowledge
necessary to build a system meeting their needs. After a month-long
deployment at Reventador volcano in 2005, we performed a careful and detailed
analysis of our system's performance. Our study showed that wireless sensor
network technology was able to meet their data fidelity needs and pioneered a
new post-hoc time rectification protocol designed to deal with protocol
failures we observed in the field.

After validating the core capability of our system to deliver data to the
scientists' specification, we addressed energy and bandwidth limitations with
a system called Lance. Lance optimizes reliable signal extraction by
considering both the energy cost and the value of the data to the
application, using a novel online heuristic to approach the performance of an
offline-optimal solution.

Next we addressed the problem of unequal energy distribution using a system
called IDEA. IDEA disseminates information about each nodes current battery
and load rates and uses this to enable energy-aware operation. System
components can use the IDEA service to determine how their own local state
impacts the operation of the network as a whole, and use this feedback to
pick states that help improve application performance.

Both of these architectural solutions are steps forward for scientific sensor
networks, but more work is needed to fully enable the distributed sensor
macroscope. We hope that these efforts will help lead the way towards a full
realization of de Rosnay's vision and a better understanding of the world
around us.
