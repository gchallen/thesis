\chapter{Lessons Learned and Future Work}
\label{chap-lessons}

\section{Future Work}

\XXXnote{GWA: From the Lance Sensys'08 paper.}

The principles guiding Lance's design also lead to several limitations we
hope to address in future work.  Lance's linear policy modules are easy to
use and compose, although it remains unclear whether more complex
interactions between policy modules are needed. Finally, we hope to study the
use of more sophisticated node-level data processing, including feature
extraction, adaptation to changing energy availability, and data
summarization.  The complications introduced by these features must be
balanced against maintaining the simplicity of our current design.

\XXXnote{GWA: From IDEA MobiSys'10 paper.}

As future work we are interested in addressing the problem of cross-component
interaction to be able to optimize the operation of several IDEA components
running in the network simultaneously. This is complicated by the fact that
there is likely to be dependencies between components that cause decisions
made by one to affect others. As an example, the LPL intervals used by a node
would effect the power cost to use the link seen by the routing protocol.  In
addition we are investigating ways to model the impact of node failure on
other nodes. Many sensor network protocols will try to work around nodes
leaving the network or going offline, but this repair process is costly and
causes load within the network to shift.

To conclude, we have described the IDEA architecture in detail, motivated its
use through three examples, and demonstrated that for each example IDEA can
improve performance by better managing distributed energy resources. We have
also discussed the process of developing an application-specific energy
objective function and shown how this can improve the performance of a
localization application while maintaining application fidelity.
\vfill\eject
