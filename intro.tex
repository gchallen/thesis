\chapter{Introduction}
\label{chap-intro}

An emerging application for wireless sensor networks is their use in medical
care. Many medical applications place new demands on sensor network designs.
They often involve variable data rates, multiple receivers, and mobile
nodes. Most existing sensor network designs do not adequately support these
requirements, focusing instead on aggregating small amounts of data from nodes
at fixed locations. Therefore a gap exists between existing sensor network
architectures and the requirements of medical applications.

In this dissertation, we bridge the gap by making three contributions: we
propose CodeBlue medical sensor network architecture, TinyADMR ad-hoc multicast routing protocol, and LiveNet passive monitoring infrastructure.
With CodeBlue, we address the challenges of providing a flexible 
and implementable system architecture for mote-based medical
sensor networks. TinyADMR addresses challenges in multicast routing with the
existence of low quality radio links, memory and bandwidth limitations.
LiveNet addresses the challenges for evaluating deployed medical sensor
networks by reconstructing network dynamics without introducing additional
overhead to the network. We evaluate CodeBlue, TinyADMR and LiveNet
with an indoor sensor network testbed and with a disaster drill deployment.

\section{Medical sensor networks}

The convergence of several new hardware technologies, including MEMS, low
power processors, and low power radio, has enabled a new class of computing
platform: wireless sensors. Wireless sensors combine the capability of
sensing, computation, and wireless communication into a single physical
package, enabling a wide range of applications. Over the past decade, sensor
networks research community has explored the use of wireless sensors in
military target tracking and environmental monitoring such as habitats,
forests, volcanoes, civil infrastructures, factory equipments, and home energy
usage.

One common characteristic of most conventional sensor networks is that they
are intended for deployments of stationary nodes that transmit data at
relatively low rates, with a focus on best-effort data collection at a central
base station. Although the sensor networks research community has made
tremendous progress in perfecting networking and operating system support in
this space, we argue that there is a lack of adequate architectural
support for one emerging and important application domain: medical care.

In a hospital or clinic, outfitting every patient with tiny,
wearable wireless vital sign sensors allows doctors, nurses and other
caregivers to continuously monitor the status of their patients.  In
an emergency or disaster scenario, the same technology would enable
medics to more effectively care for large numbers of casualties.
First responders could receive immediate notifications on any changes
in patient status, such as respiratory failure or cardiac arrest.
Wireless sensors can augment or replace existing wired telemetry
systems for many specific clinical applications, such as physical
rehabilitation or long-term ambulatory monitoring.

A typical scenario involves many patients wearing sensors that monitor basic
vital signs such as heart rate or blood oxygen saturation. A small number of
medical personnel are responsible for continuously monitoring the patients
using devices such as laptops or PDAs. Using the real-time information
retrieved from the network, the medical team can treat
patients who need urgent care more efficiently.


\subsection{Application requirements}

Medical sensor networks have several different characteristics than
conventional sensor network applications. The sensors are worn by patients and the
data sinks are mobile computers carried by doctors or nurses so that the nodes
in the network are not stationary. Unlike typical sensor networks, there are
likely to be more than one data sinks in the network. Nodes can also join or
leave the network dynamically. Furthermore, the data traffic pattern is
on-demand and dynamic. Different doctors may be interested in real-time status
of different groups of patients and the groups can potentially overlap. There may
also be interests for different types or resolutions of sensor data.
Conventional sensor network architectures are inadequate in such scenarios
because the system cannot assume fixed sensor queries, sampling rate and
static traffic pattern. 


%% data may have different priorities associated with it

\subsection{Resource limitation}

In addition to meeting a different set of application requirements, medical sensor
networks, as conventional sensor network systems, face the challenges from
severe
resource limitations of the low power sensor platforms. To be practical, the
sensor nodes must be light-weight, low power and low cost so that they can be
wearable, have long enough battery lifetime and be affordable. This entails
the use of low power radio and low power processor.  For example, the mote
platform, MicaZ~\cite{micaz}, that we use for our medical sensor networks has
4~KB of RAM, a 4~MHz 8-bit CPU and a low power IEEE 802.15.4 radio with
maximum transmit power at 0 dBm. The implication of such a limited platform is
that the software running on the mote cannot be overly computationally
intensive or require large amount of memory. Yet, the system must be responsive enough
to handle complex
interactions with the environment such as sampling medical sensors and
performing radio communication in time. As a result, we not only have to find
a new architecture that fulfills the application requirements, it must also be
designed carefully so that it is light weight enough to fit into the platform
but complex enough to fulfill the application demands. 

\section{Systems challenges for medical sensor networks}

We outline a set of challenges identified through the process of designing,
implementing, deploying, and evaluating a medical sensor network. First of
all, there needs to be a new network architecture to efficiently support the
type of queries and traffic patterns for medical sensor networks. Secondly,
to support the new architecture, it is necessary to provide multicast routing
capability on resource-limited sensor nodes. Third, studying a deployed
medical sensor network requires a non-trivial network monitoring infrastructure.

\subsection{A new network architecture}

In contrast to conventional sensor network applications, medical monitoring
must support a large number of patient sensors deployed in a hospital or
disaster site, a variety of data rates, and multiple mobile receiving
devices, such as PDAs carried by doctors or medics. Moreover, medical
monitoring cannot make use of traditional
in-network aggregation since it is not generally meaningful to combine data
from multiple patients. The goal is instead to efficiently deliver information
to the correct receivers. Therefore, there is a significant gap between
existing sensor network designs and the requirements of medical monitoring. 

Supporting such broad range of requirements demands a new approach to
sensor network design. To bridge the technology gap, we propose CodeBlue, an
architecture for medical sensor networks that provides a set of protocols and
services tailored for this application domain. CodeBlue includes
a publish/subscribe communication model, a rich query interface for periodic
and triggered collection of sensor data, a dynamic sensor discovery protocol,
and a programmatic external interface based on Web Services standards.
CodeBlue provides a high-level interface to the network, making it easy to
support new sensor types and applications that tie into real-time medical
sensor data.

\subsection{Multicast on resource limited sensor nodes}

By reviewing the application requirements of medical sensor networks, it
becomes clear that an ad-hoc multicast routing protocol is more adequate
than the conventional many-to-one data collection tree routing approach.
Fortunately there are existing multicast protocols designed for ad-hoc mobile
networks. However, after the attempt to implement one such protocol, ADMR, on
our resource limited sensor platform, we have discovered several challenges
that were not foreseen by the original designers. In order to achieve
acceptable packet delivery ratio, we have to use different routing metrics that
involves the use of link quality information. We have also
investigated the challenges to scale the network within the memory and
bandwidth limitations on the sensor platform.

\subsection{Deployment support}

To find out whether a design is practical for realistic use, one needs to
deploy the network and evaluate the performance in a realistic environment.
For this purpose, we have conducted a deployment study of the CodeBlue
network. It is necessary to record system behavior in order to study
the correctness and performance of the network during the deployment. Under severe
resource constraints on CPU, memory and bandwidth, it is not practical to
record such information on the sensor nodes locally or transmit the logged
information over the radio. To overcome this problem, we have designed and
built LiveNet, an entirely passive monitoring infrastructure for
reconstructing dynamics of deployed sensor networks.  LiveNet is passive and
therefore does not require changing the application code or incur any
additional resource consumption in the network. Yet, it is able provide
valuable information for evaluating and debugging the deployed network.

\section{Dissertation outline}

The remainder of the dissertation is organized as follows.

Chapter 2 considers background and related work. This chapter introduces 
existing work in sensor networks and ad-hoc networking research community that 
are related to medical sensor networks. We also provide an overview of target
hardware platforms of this thesis.

Chapter 3 presents the requirements for a large class of medical sensor
network applications and a new software architecture to meet these
needs. We first propose the CodeBlue network architecture by presenting the
system goals and overall architecture design. We then describe individual
components
to discuss the design trade-offs in detail. Our key contributions include
proposing publish/subscribe as the appropriate networking abstraction for
a medical sensor network and designing CodeBlue Query (CBQ) interface, which
provides a programmatic interface to the medical sensor network.

Chapter 4 presents our approach to achieving efficient publish/subscribe
communication in sensor networks with mobile nodes, based on the
ADMR~\cite{admr-mobihoc01} {\em ad-hoc\/} multicast routing
protocol. We first describe the challenges encountered when designing and 
implementing the multicast layer to support publish/subscribe communication 
model for CodeBlue. We identify the
challenges from unreliable wireless radio links, memory constraints and
bandwidth constraints. We also explore and evaluate different route discovery
and expiration strategies. The key contributions include two new 
routing metrics that exploit link quality information and identifying
scalability limitations from memory and bandwidth constraints.

Chapter 5 describes LiveNet passive monitoring infrastructure. This chapter 
covers the architecture of LiveNet that involves passive sniffers, trace
merging algorithm and a set of analyses that are used to analyze CodeBlue
deployments. We also present the results of an extensive validation study of 
LiveNet's performance and accuracy with a set of controlled experiments with 
an indoor testbed, MoteLab.

Chapter 6 covers implementation and detailed performance evaluation of
CodeBlue network architecture.  We present evaluation of CodeBlue both on a
sensor network testbed and with a disaster drill deployment.  The testbed
experiments evaluate CodeBlue in scalability, fairness, latency, jitter and
effect of mobility. Then we present a deployment study that was part of a disaster
drill simulating a bus accident. Using data from LiveNet, we are able to
analyze the traffic load, network hotspots, routing paths, and query yield for
this deployment.

Chapter 7 contains lessons learned from the experience of 
designing, implementing, deploying and evaluating medical sensor networks and
provide potential future research directions. 

Chapter 8 concludes.
