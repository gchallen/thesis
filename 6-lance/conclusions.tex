\section{Conclusions and Future Work}
\label{lance-sec-conclusions}
\label{lance-sec-conclusion}
\label{lance-sec-future}
\label{lance-sec-futurework}

Lance is intended to address the limited energy and bandwidth resources in
sensor networks by allowing applications to target resource usage at the
highest {\em value} data collected by the network, subject to a lifetime
target.  We have shown that the Lance architecture permits a wide range of
application-specific resource management policies to be constructed atop
several simple underlying mechanisms. Our results show that Lance achieves
near-optimal data retrieval under a range of energy and bandwidth
limitations, as well as varying data distributions. The analysis of our
deployment at Tungurahua shows how Lance can be effective in a field
setting. 

The principles guiding Lance's design also lead to several limitations we
hope to address in future work.  Lance's linear policy modules are easy to
use and compose, although it remains unclear whether more complex
interactions between policy modules are needed. Finally, we hope to study the
use of more sophisticated node-level data processing, including feature
extraction, adaptation to changing energy availability, and data
summarization.  The complications introduced by these features must be
balanced against maintaining the simplicity of our current design.

\subsection*{Acknowledgments}

The authors are indebted to Jeffrey B. Johnson of New Mexico Tech and 
Hugo Yepes of the Instituto Geof\'{i}sico, Escuela Polit\'{e}cnica
Nacional, Ecuador, for their invaluable assistance with the Tungurahua 
volcano field deployment. We also wish to thank our shepherd, Nirupama
Bulusu, for her help with the final version of this paper. This work was 
supported by the National Science Foundation (grant numbers CNS-0546338 
and CNS-0519675) and a generous gift from Microsoft Corporation.

