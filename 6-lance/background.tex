\section{Background and Motivation}
\label{lance-sec-motivation}

% Mention general structure: Local storage of signals coupled with 
% a reliable bulk-transfer protocol

Wireless sensor networks are becoming more common for applications
that focus on reliable collection of raw signals at relatively high sampling
rates, as opposed to in-network aggregation of low-data-rate samples.
These applications generally make use of extensive offline analysis 
to study the collected data, and it is often infeasible
or impossible to perform this computation within the network itself.
Even in cases where it is possible to shift computation to the network, a
system designer may wish to extract raw data occasionally for calibration
or testing.  Examples of such applications include structural health
monitoring~\cite{netshm-spots06,ggb-ipsn07,wimms-lynch06}, acoustic
sensing~\cite{vango,vigilnet,girod-ipsn07,enviromic}, distributed camera
networks~\cite{cyclops}, and geophysical monitoring~\cite{volcano-osdi06}.

These systems typically record data to flash at each sensor node 
and make use of a reliable bulk-transfer protocol to collect data at 
a base station. Given that the network is capable of sampling data at
a higher rate than it can be downloaded, it is not possible to 
collect the complete signal from all nodes. 
The system is therefore forced to make
a decision about what data to collect and what to throw away. In most
cases this decision is application-specific: for example, a
volcanologist may be chiefly interested in a specific type of seismic
tremor, and a biologist may be looking for acoustic signatures of a 
specific species of marmot~\cite{girod-ipsn07}. The implication is that
the system must be able to determine the intrinsic {\em value} of
a given signal to determine whether resources should be devoted to 
storing and downloading that signal.

Previous approaches have involved simple mechanisms tailored for 
specific applications. For example, in the
NetSHM~\cite{netshm-spots06,netshm-emnets05} structural monitoring 
system, data collection is 
triggered manually following an excitation of the structure.
Sentri~\cite{ggb-ipsn07} has been used for vibration monitoring at
the Golden Gate Bridge; it is unclear from the paper how sampling and
communication are triggered, but reported experimental results
suggest manual operation. The Reventador volcano monitoring 
system~\cite{volcano-osdi06} used a simple triggering algorithm to detect 
seismic events and initiate reliable transfer to the base station. 
The Intel Predictive Maintenance
system~\cite{intel-northseasensys} performs high-data-rate sampling
staggered over infrequent, periodic collection periods to extend
system lifetime. 
All of these systems involve a tight binding of the {\em mechanisms} 
used to manage storage and bandwidth with their respective 
application-specific {\em policies}.

The typical approach to download management is a FIFO model where downloads
occur in a round-robin fashion across the network once a trigger occurs. In
general, new data may be sampled and stored to flash while a download is
taking place.  Therefore, the trigger frequency, download cycle duration and
the number of nodes in the network all effect the amount of data captured by
such an approach.  For example, the Flush~\cite{flush-sensys07} protocol
achieves only 8~Kbps for a reliable transfer over multiple hops; the
Fetch~\cite{volcano-osdi06} and Straw~\cite{ggb-ipsn07} protocols fare
somewhat worse. The RCRT protocol~\cite{rcrt-sensys07} is designed for a case
where all nodes are transmitting simultaneously to the sink, although this
approach severely limits the obtained per-node throughput. As a result, when
incoming data rates exceed download capacity, FIFO download management can
produce excessive delays between data acquisition and retrieval.

Lance assumes that sensor nodes contain adequate flash storage to buffer
signals prior to download.  While the popular TMote~Sky platform supports a
relatively small 1~MB flash, more recent sensor designs~\cite{shimmer} have
several GB~of flash, and we expect this trend to continue. Rather than
focusing on per-node storage, our primary concern is with limitations on
bandwidth and energy.

Our goal is to develop a general-purpose approach to bandwidth and energy
management that complements a reliable data-collection protocol. Such a
system should have several key properties.  First, it should be {\em
customizable}, allowing different applications to specify their own policies
for storage management and bandwidth prioritization. Second, the system
should target a range of optimization goals. Examples include maximizing
overall data priority, bounding energy consumption, maximizing temporal or
spatial coverage of the collected data, or achieving fairness across sensor
nodes. Third, the system should be decoupled from a specific routing
protocol, reliable collection protocol, or sensor node platform, making it
possible to leverage the system in different settings. 

%As a concrete example, consider a network sampling triaxial accelerometer 
%data at 1~kHz at 16~bits/sample. To download 60~sec of
%this signal at 500~bytes/sec would require 
%(1000 * 3 * 2 * 60)/500 = 720~sec {\em per node}.
%Even assuming transfers from all nodes can be performed in parallel at
%this rate, during this time over 4~MBytes of new data would 
%be sampled, and with a 1~MByte flash we have no choice but to discard 
%most of the data. The fundamental limitation of the FIFO approach is that 
%it makes no attempt to prioritize which data is stored or downloaded. 

{\bf Related work:}
Several systems are related to Lance but differ substantially in their
goals and assumptions.  EnviroMic~\cite{enviromic} is a system
designed to support distributed acoustic recording by leveraging the 
collective storage resources of multiple sensor nodes. 
EnviroMic focuses on storage management and load balancing, and
assumes that data will be manually retrieved from sensor nodes
following the deployment. Unlike Lance, EnviroMic is not intended 
for applications with real-time data needs.

ICEDB~\cite{zhang2007icedb} supplies a delay-tolerant and priority-driven
query processor for the CarTel~\cite{cartel} system. While ICEDB
considers bandwidth limitations, it does not consider energy as a
constraint. ICEDB provides SQL extensions allowing queries to assign
both inter- and intra-stream priorities, which are used by the query
processor to manage bandwidth and storage resources. ICEDB also uses a
similar node-level summarization technique to that used by Lance. 

VanGo~\cite{vango} provides an architecture for collecting and processing
high-resolution sensor data on resource-constrained nodes.  VanGo focuses on a
programming model based on a linear filter chain and implementing efficient
signal-processing operations with limited computational power.
WaveScope~\cite{wavescope} and Flask~\cite{flask-tr} are languages for stream
processing applications.  These systems are largely complementary to Lance,
and could be used to process signal data prior to collection, although our
focus is on collecting {\em raw} sensor data from large networks.  These
systems do not attempt to optimize data collection under varying energy and
bandwidth constraints. 

%The general problem, therefore, is how to best leverage limited sensor 
%network resources to obtain the most ``valuable'' data for an end user.
%
%To support different applications, we wish to customize the approach 
%with varying policies for assigning ``value'' to data, as well as for
%...

% discuss problems with simple round-robin download and FIFO storage
% management:
%  - loss of data due to lengthy download times (may take longer
%     to download data than it takes to sample, esp. over multihop
%     paths - give numbers)
%  - no consideration of devoting more storage or b/w to different
%     signals when we have a spectrum of 'utility' values
%       - give example of weak, uncorrelated data to strong,
%          correlated data
%       - or having preference for certain nodes over others based on
%          placement or SNR.

% Need general approach to this problem that can be tailored for
% different application settings. The goals of such a system:
%   - flexible and customizable
%   - support range of optimization targets: maximizing overall
%      utility, maximizing temporal or spatial coverage, 
%      achieving fairness, etc.



