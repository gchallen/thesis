\section{Introduction}

Many sensor network applications involve the acquisition
of high-resolution signals using low-power wireless sensor nodes. Examples
include monitoring acoustic, seismic, and vibration waveforms in
bridges~\cite{ggb-ipsn07}, industrial equipment~\cite{intel-northseasensys}, 
volcanoes~\cite{volcano-osdi06}, and animal
habitats~\cite{girod-ipsn07,enviromic}.  These systems all attempt to acquire
high data-rate (100~Hz or higher), high-fidelity data across the network,
subject to severe constraints on radio bandwidth and energy usage.

Given these constraints, it is typically not possible to acquire
continuous waveforms from all nodes.  As a result, applications strive
to acquire the most ``interesting'' signals, such as a marmot call or
earthquake, and avoid wasting resources on ``uninteresting'' signals.
Currently, these resource-management decisions are made on an {\em ad
hoc} basis for each application, often resulting in suboptimal solutions
that can consume excessive bandwidth or lose data.  We argue that all
of these applications would benefit from a general approach to managing
resources that optimizes the application-specific value of the
data acquired by the network.

% Define challenges here
Optimizing reliable data acquisition requires a coordinated approach
to managing both limited energy capacity and severely constrained
radio bandwidth. Depending on the sampling rate and
resolution, downloading signals may take longer than real time;
while low-power 
sensor node radios obtain single-hop throughput of about 100~Kbps, the
the best reliable protocols achieve less than 
8~Kbps for a single transfer over multiple hops~\cite{flush-sensys07}. 
Likewise, to achieve long lifetimes in the field, the energy cost of 
downloading a signal from the network must be carefully considered. 
The fundamental challenge is how to best direct these limited network
resources to acquire the most valuable data to the application.

This paper presents {\em Lance}, a general approach to bandwidth and energy
management for reliable signal collection in wireless sensor networks. In
Lance, each node acquires data at potentially high rates. For each
application data unit (ADU), each node generates a concise ADU {\em summary},
which is periodically sent to the base station and used to compute a ADU {\em
value}.  Lance computes an ADU {\em download schedule} based on these values,
and uses a reliable transfer protocol to download ADUs according to this
schedule.

Energy usage and battery lifetime are major concerns for long-term
sensor network deployments. 
Lance incorporates a {\em cost estimator} that
predicts the energy cost for reliably
downloading each ADU from the network. We describe a novel energy cost
estimation algorithm that uses information on the network topology to 
determine the energy cost at the sensor node hosting the ADU as well
as intermediate nodes impacted by the multihop transfer protocol. 
Information from the cost estimator is used to adjust the download
schedule for ADUs, allowing Lance to target a battery lifetime for 
the network by load-balancing download operations in a manner that
adheres to an energy schedule.

Lance incorporates a general framework for managing bandwidth and
energy that decouples the mechanism for prioritizing resource allocation from
the application-specific policies used to assign ADU values.
This is accomplished through user-supplied {\em policy modules} that
permit a range of sophisticated prioritization policies to be
tailored for specific applications. Policy modules allow Lance to
target a broad range of optimization metrics, including node-local and
network-wide value maximization, lifetime targeting, 
and acquiring temporally-
or spatially-correlated data from across the network. 
Policy modules allow the network's behavior to be significantly altered
at the base station, without reprogramming the sensor nodes themselves.

The contributions of this paper are as follows. First, we present the
Lance architecture in detail, which is the first system to provide 
a value-driven bandwidth and energy management framework for 
high-data-rate sensor networks. Second, we describe Lance's policy 
modules, which offer a clean separation of policy and mechanism that 
allows the system to be tailored to a broad range of applications. 
We focus on one application in detail: using Lance to maximize data 
quality for a network of seismic and acoustic sensors deployed at 
an active volcano. Third, we show through detailed simulation 
measurements that Lance achieves {\em near optimal} efficiency 
(greater than 96\% in most cases) under a range of data distributions 
and resource limitations. 
Fourth, we present results from 
an eight-node field deployment of Lance at Tungurahua Volcano, 
Ecuador, demonstrating the system's performance in a real 
field setting and the flexibility of policy modules for 
altering the network's operation following deployment. 

The rest of this paper is organized as follows.
Section~\ref{sec-motivation} presents motivation and related
work. We describe the architecture of Lance in detail in
Section~\ref{sec-architecture} and discuss the use of application-specific
policy modules in Section~\ref{sec-policies}.  Section~\ref{sec-casestudy}
presents a case study of several applications that can make use
of the Lance architecture.  We briefly describe our prototype in
Section~\ref{sec-implementation}.  Section~\ref{sec-evaluation}
presents a detailed evaluation while Section~\ref{sec-deployment}
presents results from our field deployment. Finally,
Section~\ref{sec-conclusions} discusses future work and concludes.

