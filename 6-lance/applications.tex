\section{Application case studies}
\label{sec-applications}
\label{sec-casestudy}

In this section, we present several case studies of how Lance can
benefit different application domains. We describe our use of Lance in
a geophysical monitoring sensor network in detail, and discuss its use for
limb motion analysis, structural monitoring, and animal habitat
monitoring. All of these applications share the challenges that Lance was 
designed to address: the need for reliable data collection with 
limited bandwidth and storage. 

\subsection{Geophysical monitoring}

Wireless sensor networks can greatly benefit geophysical monitoring
applications, such as seismic and acoustic data collection at fault
zones~\cite{ucla-seismic} and volcanoes~\cite{volcano-osdi06}. Low-power
wireless sensors are highly desirable in these settings, where sensors must
be carried to a deployment site by hand and often cannot be manually serviced
once deployed. 

The goal is to deploy a wireless array of sensors for 
monitoring both seismic and infrasonic (low-frequency acoustic)
signals, including earthquakes, tremor, and volcanic eruptions. 
Depending on the level of seismic or infrasonic activity, the network 
might record dozens of events an hour, with each event typically lasting 
less than 60~sec (although long-period tremor events can last minutes 
or hours). 

As an example, in our previous work on volcano monitoring at
Reventador~\cite{volcano-osdi06}, each sensor node continuously sampled two
channels of seismic and acoustic data at 100~Hz per channel with a resolution
of 24~bits/sample. Raw data was stored to the node's flash memory as a
circular buffer.  An event-detection algorithm running on each node would
send an event report to the base station whenever an interesting seismic
signal was detected. If the base station received enough event reports within
a short time window, it would initiate a round-robin download of the last
60~sec of data from each node.  Although this system was successfully
deployed, it exhibited several deficiencies which led to a significant loss
of data~\cite{volcano-osdi06}. 

The first problem is that the decision used to download a given signal 
was based on a simplistic binary approach, based on the event-detection 
algorithm running on each node. As a result, the system could not 
prioritize certain events over others. The event-detector logic used 
a simple threshold scheme, and as reported in~\cite{volcano-osdi06}, 
the threshold was set too low, causing the network to trigger on less 
than 5\% of the actual seismic events.

%{\bf FIFO storage management:}
%Each sensor node can only buffer a few minutes of data. Moreover, 
%download cycles can exhibit very high latency, typically requiring
%several minutes for downloading 60~sec of data from a single node.
%To avoid overwriting data during a download cycle, the system disabled 
%sampling during downloads, causing loss of data during transfers.

The second problem was that following each trigger, the network 
initiated a {\em nonpreemptive} download from every node in the
network in a round-robin fashion. This policy caused 
the system to devote resources to downloading small precursor earthquakes 
that immediately preceded larger eruptions~\cite{volcano-osdi06}. 
As a result, many such larger events were not captured. 

Finally, our previous system made no attempt to manage energy. 
As a result, the expected lifetime of the
network is only about a week (using D-cell batteries), necessitating
frequent battery changes over a long deployment.
Clearly, this system could benefit from a prioritized approach to
download management that also considers energy costs to increase
lifetime.

\subsubsection{Adaptation to Lance}

To address these problems, we reimplemented our previous volcano
monitoring system using Lance. Many of the components of the original system,
such as multihop routing, time synchronization, reliable download
protocol, and flash storage interface, remained unchanged. 
The node-level event detector was replaced by an ADU summarization function,
as described below. The base station code for responding to
correlated events was replaced with Lance's optimizer and
policy modules. Our deployment of the completed system at Tungurahua
volcano in August 2007 is discussed in Section~\ref{sec-deployment}.

%\begin{figure*}[t]
%\begin{center}
%\begin{tabular}{ll}
%\raisebox{0.35in}{\parbox{0.05in}{\textbf{(a)}}} & 
%\includegraphics[width=0.95\hsize]{./figures/DataFigure/TOPDATA.pdf}
%\vspace{-0.05in}\\
%\raisebox{0.31in}{\parbox{0.05in}{\textbf{(b)}}} & 
%\includegraphics[width=0.95\hsize]{./figures/DataFigure/EWMA.pdf}
%\vspace{-0.05in}\\
%\raisebox{0.32in}{\parbox{0.05in}{\textbf{(c)}}} & 
%\includegraphics[width=0.95\hsize]{./figures/DataFigure/RSAM.pdf}
%\end{tabular}
%\end{center}
%\caption{\small{\bf Comparing Two Node Utility Calculators:} 
%Shown in relation to the 6~hour data trace, subplot (a), are the output of
%the two candidate node utility calculators described in
%Section~\ref{sec-casestudy-nuc}, with dotted lines identifying the 50
%highest-utility data units computed by each detector. Subplot (b) shows the
%EWMA detector, which is good at detecting event onsets; (c) shows the RSAM
%detector, which offers a better indication of the overall strength of
%volcanic activity. Most of the 50 top ADUs identified by the RSAM detector
%occur during long-period events, such as volcanic tremor. Since the stated
%goal of the volcano monitoring application was to extract short-period events
%we chose the EWMA detector for the majority of our evaluation.}
%\label{fig-nucplot}
%\end{figure*}

\subsubsection{Summarization functions}
\label{sec-casestudy-nuc}

The original system was intended to detect correlated seismic
events from across the network and download data from all nodes,
regardless of whether every node detected the event. This was based
on a simple event detector that computes two exponentially-weighted
moving averages (EWMA) of the seismic signal, with different gain
settings; one EWMA represents the short-term average and the other
the long-term average. When the ratio between these two averages
exceeds a threshold, an event detection message is sent to the base 
station. Subsequent triggers are suppressed for a short duration
afterwards.

This policy is straightforward to implement in Lance by using the ``ratio of
two averages'' as the node-level summarization function.  Rather than
performing thresholding at the node level, we report the maximum ratio over
the ADU as its value, allowing Lance to prioritize different events. The base
station's policy modules are configured as shown in
Section~\ref{sec-example-policies}, using a chain of {\tt filter}, {\tt
correlated}, and {\tt spacespread} to implement the equivalent of the event
triggering policy used in the original system. Note that the Lance version of
the system differs from the original in that download management is
value-driven rather than FIFO. Also, Lance can download ADUs from different
events out of order, avoiding the nonpreemptive download problems of the
earlier system.

While our original system was designed to capture short earthquakes,
were also interested in determining whether Lance could be used to
capture different types of volcanic activity. For this, we make use of
the Real-Time Seismic Amplitude Measurement (RSAM)~\cite{rsam}, which
computes the average seismic amplitude over a given time window. 
Intuitively, RSAM measures 
the total amount of ground shaking caused by earthquakes and tremor,
and is often used by volcano
observatories to characterize the overall level of seismic activity.
%A related measure is the Seismic Spectral-Amplitude Measurement (SSAM), 
%which computes RSAM separately over several frequency bands. 

Different summarization functions and policy modules can be used to implement
a wide range of geophysical monitoring systems with Lance. For example, a
hazard monitoring system could be configured to periodically report RSAM
values for all sensor nodes and download only the strongest events for
further analysis. By limiting downloads to those ADUs with RSAM above some
threshold, energy can be saved. In contrast, a scientific study that wishes
to perform earthquake localization~\cite{aki-richards-80} or tomographic
inversion~\cite{lees-lindley-94} would prefer to download only small
earthquakes with clearly delineated onsets, which can be used to determine
the velocities of seismic waves. Likewise, a researcher studying explosive
events would prefer to download only seismic events with a corresponding
infrasonic component, since non-explosive earthquakes should not generate any
infrasound.

% MDW: I think you do NOT want to even talk about the "37 events
% identified by a seismologist" since it is not relevant here, and
% makes it sound like RSAM is the "wrong"  utility metric. 
%Figure~\ref{fig-nucplot} shows each candidate utility calculated over 
%the 6~hour
%time period used for our analyses. Each graph also identifies, for each
%candidate detector, the top 37-events, as determined by the assigned utility
%values. Though neither detector is perfect, the {\tt EWMA} detector does a better
%job of meeting the scientific expectations, correctly identifying 17 out of
%the 37 scientist-picked events, as compared with only 1 for the {\tt RSAM}
%detector. We believe that this is because the {\tt RSAM} detector will report
%multiple high-utilities during long-period events such as tremor, which this
%application was designed to avoid. Although {\tt RSAM} may be a more
%general-purpose metric and ultimately capable of detecting more types of
%seismic activity, due to the focus on short-period events we chose the {\tt
%EWMA} detector for the evaluations that follow.

% MDW: I am commenting this out since it overlaps too much with the
% Pixie paper.
%\subsection{Motion analysis}
%
%Another application domain that we are exploring is motion
%analysis of patients with movement disorders, such as Parkinson's
%Disease~\cite{parkinsons-embs07}. In this context, up to ten sensor nodes
%equipped with triaxial accelerometers and gyroscopes are placed on the
%patient's limb segments (two each on the arms and legs plus one each
%on the torso and waist), collecting high-resolution data at rates up to 100~Hz
%or more. The goal is to capture data from the body
%sensor network during periods of dyskinesia (abnormal movements) or
%bradykinesia (slowness of movement) associated with the disease. 
%The base station will typically 
%be a laptop located in the home, and as such will experience a wide 
%variation in bandwidth to the body sensor network (including
%disconnections), depending on the patient's location.
%
%%Use of low-power wireless sensors keeps the size and weight
%%of each device down: for example, the wearable sensor node described
%%in~\cite{parkinsons-embs07} measures $44 \times 20 \times 13$~mm and
%%weighs just 10~g. 
%%While the sensor network is not spatially distributed, and
%%all nodes are within a single radio hop of each other, the data rates
%%greatly exceed the radio channel bandwidth: a single node will 
%%consume more than a quarter of the best-case radio capacity, assuming
%%no protocol overhead or retransmissions. 
%
%We are currently developing a sensor network for motion analysis,
%making use of Lance to drive the storage and bandwidth management.
%Each sensor node computes a series of high-level {\em features} from
%the raw sensor data, such as peak amplitude, maximum entropy, and
%RMS. The node prioritization function assigns higher priority to 
%features appearing to represent abnormal movement. The raw signal 
%is also stored as separate ADUs with lower priority than the features,
%allowing Lance to restrict downloads of the raw data to periods
%with a strong radio link to the sensors. During periods of
%disconnection, nodes will buffer ADUs for later transmission; the
%wearable sensors we are using support a large (up to 2~GByte) flash
%memory for this purpose. Priority modules at the base station can 
%estimate the available bandwidth to the body sensors, based on radio
%link quality, and prioritize downloads accordingly. Although this
%system is still under development, Lance's core mechanisms and policy
%modules appear to be a good fit for this application.

\subsection{Other application domains}

We believe that Lance can be used to benefit many applications that make use
of high-resolution signals delivered over a bandwidth-limited wireless
network. These applications require high data rates, precluding continuous
data collection, and rely on classification techniques to determine which
signals to download. Two examples are given below.

{\bf Structural monitoring:}
Structural monitoring systems collect vibration
waveforms from a building, bridge, or other structure in order to
study structural properties and seismic response.
In previous systems~\cite{netshm-emnets05,ggb-ipsn07}, data collection
has been triggered manually or on a simple periodic schedule. Instead,
Lance can be used to prioritize signals following an earthquake or 
forced excitation of the structure, similar to the EWMA and RSAM
functions described earlier. To save energy, the system could choose a
subset of nodes from which to download data to achieve a good spatial
distribution across the structure. The size of the subset could be
chosen depending on the strength of the excitation. In addition, 
policy modules can be used to perform periodic downloading of ADUs 
from each sensor for calibration, as well as to determine whether 
each sensor is still functioning properly.

{\bf Animal habitat monitoring:}
Habitat monitoring applications that deploy high-bandwidth sensors, such as
microphones or cameras, are good candidates for prioritized data extraction. 
An example application may attempt to download interesting audio signals
facilitating offline species classification or 
localization~\cite{girod-ipsn07}. The summarization function could
involve either a triggered event detector, an audio waveform
classifier, or motion detector from a series of camera images~\cite{cyclops}.

At the base station, policy modules can use offline knowledge of node
positions to modify the initial ADU value. One approach might enhance
spatial coverage by prioritizing data collection from nodes nearby the
source of the signal. Another could reject noise by deprioritizing signals
detected by only one node. For example, if fewer than three nodes
report an audio event, it is impossible to perform acoustic
localization and Lance need not waste bandwidth on the signal.
Policy modules can take other metrics into account as well, 
such as the SNR of the recorded signal or the time of day (e.g.,
reducing confidence in camera images taken at night).

